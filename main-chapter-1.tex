\chapter{The Problem And Its Background}
\thispagestyle{empty}
Itinerary planning is an important process that can enhance the experience of tourists in their travel such as how well the places interest the user and the flow in which the places are traversed. In the case of the province of Bulacan, this study created a mobile personalized itinerary generator as a possible solution to optimizing existing itineraries and providing rich and personalized itineraries of Bulacan tourism, based on the interests of the user. This chapter introduces the background, objectives, significance, and scope of the study.
 
\section{Background of the Study}

In the midst of advancement in technology and sciences, culture remains a vital part of many people’s lives. Tourism, being a significant contributor to preservation of culture and local development, relies heavily on how well travel experiences are delivered. One of the most crucial aspects of this process is itinerary planning, a practice ensuring the maximization of multiple points of interest (POIs) efficiently without the unnecessary inconvenience. The ability to create structured travel plans does not only influence convenience but also impacts tourist satisfaction and chance of returning or recommending the destination to others.

Sustainable and authentic trends in travel experiences align with the Philippines’ Department of Tourism’s (DOT) initiatives to develop unique cultural and natural attractions \parencite{DOT2023}. Despite these efforts, historically active provinces like Bulacan still has some of its tourist destinations neglected, proven by how some culturally relevant sites like Barasoain Church, often attract more visitors than others like the Immaculate Conception Cathedral, with 2.6 times more visitors, despite being in the same municipality \parencite{Canet2025S}. Bulacan’s tourist arrival reports also show that certain municipalities acquire an uneven number of visitors for the first three quarters of the year 2025 based on the data provided by the  Provincial History, Arts, Culture, and Tourism Office (PHACTO). A problem that affects the development of Bulacan’s tourism, resulting in an overlook of economic opportunities for lesser-known municipalities.

These conditions highlight the need for smarter ways to plan trips, systems that can pull together all sorts of information and offer suggestions perfectly tailored to what each person likes. This is especially important as more and more people want to explore and truly experience a destination's uniqueness, which often demands a level of planning which outdated methods often fail to deliver \parencite{Halder2022}. The Bulacan Provincial Government is actively promoting their attractions through Bulacan’s Pamana Pass, a checklist of certain sites, but it only acts as a guide. Technology can help people utilize it more, and make trips more satisfying ensuring tourism spots grow in a way that benefits everyone.

Creating itineraries under many constraints has become increasingly automated through a mix of both optimization algorithms and content selection processes to resolve the scheduling and relevance aspects of itinerary recommendation \parencite{Jewpanya2025,Liu2024}. However, such systems are often limited to specific regions and bound by the capabilities of the model used or lack of data \parencite{Cui2025,papadakis2024,Yulfihani2024}. Despite the drive to develop systems that simplify the complicated problems of manual itinerary planning, existing travel apps still struggle to effectively recommend itineraries as reports from \textcite{SkiftState2022} highlights that applications may offer destination suggestions but lack the ability to offer suggestions based on user preferences. Their reports also show the demand for a personally curated service, since 30 percent of travelers responded that they are “more likely to use a travel agent now than before the pandemic”. Although there are systems that already cater to this problem, a common weakness they have is overcomplexity \parencite{Postnikova2024}. General-purpose platforms also generally do not align with the Philippines' strategic imperative to build a deeper, distinct Filipino experience \parencite{SkiftState2022,DOT2023}. These weaknesses expose the need for a simple, localized itinerary recommendation system, particularly in areas with uneven destination popularity like Bulacan. This study adopts Adaptive Genetic Algorithm with Dynamic Mutation and Crossover Probabilities (AGAM) developed by \textcite{Yochum2020} as a recommender algorithm to ensure suggestions does not merely list POIs, but instead consider user preferences.

Another problem of planning a travel itinerary arises when considering other constraints beyond personalization. A challenge that involves not only selecting POIs but also determining the optimal visiting order, persisting despite advances in recommendation and optimization methods \parencite{papadakis2024,Porras2022,Yulfihani2024}. These problems demonstrate that classic ways of planning itineraries lead to unsatisfactory results, implying the inherent value in creating automated systems for planning itineraries more efficiently. From a mathematical perspective, this problem is similar to the Traveling Salesman Problem (TSP) \parencite{Wu2020}. To address such complexity, this study uses an additional optimization algorithm that caters to TSP, which will help optimize the given routes of POIs.

These limitations, taken together, imply that manual travel planning methods in Bulacan remain inefficient due to uneven attraction popularity, lack of a personalized and optimized routing system, and the absence of accessible tools that integrate both user preferences and route optimization, which points to a need for a practical, user-centered, and localized solution.

This study developed Lakad to address the identified problems and provide a tourist utility app for exploring destinations in the province of Bulacan. The system was developed to deliver two core features: a personalized itinerary generation, helping tourists better plan their trips according to their specifics, and an optimization algorithm to ensure routes are arranged in the best way possible. By combining the ideas of integrating personalized preference modeling and route optimization, Lakad seeks to provide a simplified and localized solution. Ultimately contributing to DOT’s strategic goals by promoting Bulacan’s cultural identity and mitigating imbalance in the tourism destinations.


\section{Statement of the Problem}

The lack of systems and applications for itinerary generation in Bulacan makes it much harder for tourists to traverse the wonders the province has to offer. As such, the main objective of this study is to develop a mobile personalized itinerary generator for promoting tourist locations in Bulacan as well as providing optimized paths in the itinerary, allowing the tourist to further enjoy their trip. Specifically, this study aims to answer the following questions: 

\begin{enumerate}
    \item Which algorithm performs best in terms of run time and solution quality for solving the Traveling Salesman Problem (TSP) among the following:
    \begin{enumerate} [label*=\arabic*.] 
        \item Ant Colony Optimization (ACO),
        \item Genetic Algorithm (GA),
        \item Simulated Annealing (SA),
        \item Particle Swarm Optimization (PSO),
        \item Elephant Herding Optimization (EHO),
        \item Grey Wolf Optimizer (GWO), and
        \item Hovering Scouts and Foraging Flocks Pied Kingfisher Optimizer (HSFFPKO)?
    \end{enumerate}
    \item How can the proposed system be developed with the following functionalities:
    \begin{enumerate} [label*=\arabic*.] 
        \item  Tourist Spot Exploration,
        \item  Personalized Itinerary Generation,
        \item  Itinerary Optimization,
        \item  Itinerary Navigation,
        \item  Itinerary Management, and 
        \item  Admin Management?
    \end{enumerate}
    \item How acceptable is the developed system based on the criteria defined in the Technology Acceptance Model?
    \begin{enumerate} [label*=\arabic*.]
        \item Perceived usefulness,
        \item Perceived ease of use,
        \item Attitude towards using, and
        \item Behavioral intention?
    \end{enumerate}
    \item How well does the developed system meet the ISO/IEC 25010:2023 requirements?
    \begin{enumerate} [label*=\arabic*.]
        \item Functional Suitability,
        \item Performance Efficiency,
        \item Compatibility,
        \item Interaction Capability,
        \item Reliability,
        \item Security,
        \item Maintainability,
        \item Flexibility, and
        \item Safety?
    \end{enumerate}
\end{enumerate}

\section{Significance of the Study}
This study developed an optimized mobile itinerary planner designed to promote tourism in Bulacan as a destination for tourists while providing functional and efficient navigation support for locals and commuters. The system functions as a tool that promotes the province's tourism industry, highlights its attractions, and makes trip planning easy by combining efficiency and personalization.

\textbf{Tourists.} The system provides personalized and optimized itineraries according to the user’s interest and preferred distance. In this way, they can enjoy both well-known and lesser-known parts of Bulacan's natural, historical, and cultural heritage while making the most of their time.

\textbf{Local Establishments.} The system gives them more chances to get their products and attractions seen because they can be included in the itineraries. This kind of advertising gets more customers involved and helps local businesses grow.

\textbf{Local tourism sector.} The system provides a tool to tourism offices and local tourism operators that enhances destination promotion, encourages tourist engagement, and improves the efficiency of travel planning.

\textbf{Local Communities.} The system provides residents and local communities with assistance in moving around the province and discovering nearby attractions. By improving mobility, the system helps develop the community and allows all people to appreciate the Bulacan’s cultural and historical sites more.

\textbf{Commuters.} The system provides an alternative tool designed to enhance daily travel efficiency for commuters within Bulacan. It brings practical convenience to people who rely on mobile guidance during regular transport because it offers optimized routing and accessible location information.

\textbf{Future Researchers.} This research can be a reference for researchers who would want to improve or further develop the tourism recommender system. The use of AGAM and TSP algorithms provides a foundation that others may extend, compare with different algorithms, or further improve by adding other features.

Overall, this study is significant because it supports development of tourism, sustainable development, and culture enhancement in Bulacan, and provides an example of how optimized and technology-driven solutions may enhance travel and navigation experiences.


\section{Scope and Limitation of the Study}

This study focuses on the development and evaluation of Lakad, a mobile personalized itinerary generator designed for the province of Bulacan. The research addresses the identified gap in manual itinerary planning and the limitations of existing travel applications by providing a localized solution that promotes the tourism in Bulacan. The application enhances the travel experience of tourists by providing functionalities such as itinerary optimization, personalized itinerary generation, tourist spot searching, itinerary navigation, and itinerary management. By integrating these features, the system makes traveling within Bulacan more convenient, efficient, and engaging for visitors.

Lakad is limited to the tourist destinations within Bulacan. The Points of Interest (POIs) included in the application are natural attractions, historical sites, cultural landmarks, and recreational areas, which are verified by the Provincial History, Arts, Culture, and Tourism Office (PHACTO). Pasalubong Center were also included in the application. Restaurants are not included as POIs due to their large number and the potential bias towards bigger establishments. 

The system was developed using the React Native framework, which enables cross-platform compatibility. However, deployment is limited to Android operating systems due to resource and cost constraints. iOS development requires access to macOS hardware and an Apple Developer account, which were not feasible for this study. Future iterations of Lakad may consider iOS implementation to expand platform reach.

For mapping and navigation services, the system integrates Mapbox API, which supports driving, walking, and cycling directions. Public transportation routing is not included, as Mapbox does not provide transit or multi-modal transportation data. Geographic coordinates and location data were sourced from Mapbox and PHACTO official records.

The itinerary generation feature applies Adaptive Genetic Algorithm with Dynamic Mutation and Crossover Probabilities (AGAM) that balance multiple objectives including POI ratings, user-defined categorical interests, and travel distance constraints. The fitness function incorporates weighted factors to generate personalized recommendations aligned with user preferences. For route optimization, the system employs the algorithm determined to be most suitable based on a comparative analysis of seven metaheuristic algorithms, evaluated using standardized TSPLIB instances. The evaluation criteria include solution quality, measured by average relative error, and runtime efficiency. The algorithm ranking was performed using the Simple Additive Weighting (SAW) method with equal weights assigned to both criteria.

The system generates single-day itineraries constrained by user-defined parameters such as maximum travel distance and maximum number of POIs. Multi-day tour planning is not supported in the current implementation. Users may pause and resume itinerary navigation, and stops are automatically marked as visited based on proximity detection or manual confirmation. The system does not integrate external booking services such as accommodation reservations, and ticket purchases.

The system’s quality and acceptability was quantitatively evaluated using two standard instruments, Technology Acceptance Model (TAM) and ISO/IEC 25010:2023. The evaluation for the end-users utilize TAM to assess perceived usefulness, ease of use, and intention to use towards the application. In addition, the system was assessed against the ISO/IEC 25010:2023 software quality standards, covering functional suitability, performance efficiency, compatibility, interaction capability, reliability, security, maintainability, flexibility and safety.




