\chapter{The Problem And Its Background}
\thispagestyle{empty}
Itinerary planning is an important process that can enhance the experience of tourists in their travel such as how well the places interest the user and the flow in which the places are traversed. In the case of the province of Bulacan, this study seeks to propose a mobile personalized itinerary generator as a possible solution to optimizing existing itineraries and providing rich and personalized itineraries of Bulacan tourism, based on the interests of the user. This chapter introduces the background, objectives, significance, and scope of the study.
 
\section{Background of the Study}

In the midst of global advancement in technology and sciences, culture remains a vital part of many people’s lives around the world, and so are the habits they have in connection to it. Tourism, being a significant contributor to economic progress, p  reservation of culture, and local development, relies heavily on how well travel experiences are organized and delivered. One of the most crucial aspects of this process is itinerary planning, a practice which ensures that tourists can maximize their time while visiting multiple points of interest (POIs) efficiently, and enjoy what a place can offer without the unnecessary inconvenience. The ability to create a structured travel plan does not only influence convenience but also impacts tourist satisfaction and chance of returning or recommending the destination to others.

According to \textcite{Unwto2023} and \textcite{Gecikli2024}, the tourism sector has been undergoing a transformation marked by the growing interest in sustainable practices in tourism such as authentic cultural immersion and the exploration of lesser-known destinations. Post-pandemic recovery gave an emphasis on responsible tourism and a shift away from mass tourism towards more personalized and meaningful travel experiences. This trend is evident in the rise of modern tourism markets such as ecotourism, adventure tourism, and cultural heritage tourism, which often focus on underexplored regions. Indigenous tourism in regions such as Fiji, Australia, and Aotearoa New Zealand are gaining popularity as they offer the experiencing of cultural values of guardianship and community wellbeing, leading to authentic cultural exchanges while advancing self-determined and sustainable local development \parencite{Scheyvens2021}. While at the same time, the advances in itinerary-planning technologies enables travellers to customize travel routes according to their individual interests, thereby aligning with the global shift toward meaningful and locally-rooted tourism experiences \parencite{Halder2022}.

In the Philippines, the Department of Tourism (DOT) has been actively promoting lesser-known destinations and cultural heritage sites to better diversify tourism offerings and distribute its economic benefits across the country. Global trends focusing on sustainable and authentic travel experiences align with DOT’s initiatives that include developing new tourism circuits, investing in infrastructure in emerging destinations, and implementing digital marketing campaigns to highlight the unique cultural and natural attractions. In many provinces, recent campaigns have emphasized community-based tourism and the promotion of their local festivals, which aims to provide tourists with deeper cultural immersion while also supporting local economies \parencite{DOT2023}. This approach is rooted from the Connectivity, Convenience, and E(quality) strategies of DOT’s National Tourism Development Plan (NTDP) to strengthen the value of Filipino identity in every tour, which ensures that Philippine tourism offers authentic experiences with the richness and diversity of Filipino cultures across thousands of islands in the country.

Bulacan, even though a historically active province in the Philippines, still has some of its historical and cultural destinations neglected despite the province’s rich heritage, proven by how some culturally relevant sites like Barasoain Church and the Basilica Minore de Immaculada Conception both in Malolos, attract and often have more visitors than others like the Meyto Shrine of Calumpit, which respectively had 78, 30, and 10 visitors, collected among 100 respondents \parencite{Canet2025S}. Thus, promotion of tourism, and by extension, the development of tools that can improve it is a vital aspect to the continued revival of the province’s tourism sector. And, as further supported by its Provincial History, Arts, Culture, and Tourism Office (PHACTO), Bulacan’s tourist arrival reports show that the municipalities where the aforementioned churches are located had 315,929 and 103,953 visitors respectively for the first three quarters of the year 2025. The Provincial Government of Bulacan launched the Pamana Pass in 2023, an initiative that promotes the tourism engagement of the province \parencite{Velasco2023}. The Pass initiative, although aims to strengthen the relevance and popularity of the destinations, lacks the capability of offering an optimized itinerary plan as the order of visitation is left to be decided by the tourist.

The exploration of new tourism frontiers, especially with the desire for unique, personal journeys, highlights the need for smarter ways to plan trips. Systems that can pull together all sorts of information, offering suggestions perfectly tailored to what each person likes. This is especially important as more and more people want to explore beyond the usual tourist traps and truly experience a destination's uniqueness, which often demands a level of planning that outdated methods just can't deliver \parencite{Halder2022}. By using technology to help people understand these special places, they can make their trips more satisfying and ensure these tourism spots grow in a way that benefits everyone. In this regard, itinerary planners and POI recommendation systems are such of the many utilities that can improve this.

Itinerary planning is a process where someone thoughtfully organizes a set of destinations into a detailed schedule in order to maximize their time while taking into account specific constraints that can affect their experience. Nowadays, creating itineraries has become increasingly automated through recommendation systems, which can even generate schedules based on many constraints \parencite{Jewpanya2025,Liu2024}. Some of the modern-day recommendation systems are often a mix of both optimization algorithms and content selection processes. This method resolves not only the scheduling aspect of itinerary recommendation, but also the relevance of the destination suggested to users. However, such itinerary recommendation systems are limited to specific regions, often bound by the capabilities of the model or lack of data that can be used to identify which destinations are to be included \parencite{Cui2025,papadakis2024,Yulfihani2024}.

The development of itinerary recommendation systems are driven by the need to simplify the complicated problems of manual itinerary planning. However, many existing travel apps struggle with effective itinerary optimization as reports from Skift’s, a leading source for travel news and research, Skift Research highlights that many of those applications offer destination suggestions but lack the ability to offer suggestions based on user preferences, which leads to less personalization of travel plans. This report also shows the demand now for a personally curated service, since 30 percent of travelers responded that they are “more likely to use a travel agent now than before the pandemic” \parencite{SkiftState2022}. Although there are many systems that are already developed to cater to this problem, a common weakness they have lies in the overcomplexity of their systems \parencite{Postnikova2024}. Furthermore, general-purpose platforms often fail to provide the necessary level of curation and destination discovery required to highlight local, and unique cultural experiences, and they generally do not align with the Philippines' strategic imperative to build a deeper, distinct Filipino experience \parencite{SkiftState2022,DOT2023}. These weaknesses expose the need for a simple, localized itinerary recommendation system, particularly in areas with uneven destination popularity like Bulacan.

Planning a travel itinerary manually is a demanding task, as many tourists must differentiate attractions to fit their interest while at the same time meeting limitations in terms of time and location accessibility. \textcite{Hendrawan2024} stated that an increase in POIs in combination with limited resources makes it difficult to create itineraries, and while travel agents offer assistance, such services are high cost and inaccessible to many tourists. \textcite{Porras2022} also argued that planning itineraries is fundamentally difficult, for it not only involves choosing POIs themselves but also finding an optimal visiting order under numerous constraints, such as personal interests, transportation, and time constraints. These problems demonstrate that classic ways of planning itineraries tend to lead to incomplete or unsatisfactory schedules, which there is value in creating automated systems to plan itineraries more efficiently. From a computational perspective, this problem is closely related to the model of Traveling Salesman Problem (TSP), which is known to have solutions that become exponentially more complex as the number of destinations or POIs increases \parencite{Wu2020}. And even when functional apps are deployed, itinerary generation has to carefully consider constraints within user satisfaction metrics, which remains a significant task despite advanced recommendation and optimization methods \parencite{papadakis2024,Yulfihani2024}. These limitations point to the need for a practical, user-centered, and localized solution.

To address these identified challenges, a proposed itinerary recommendation system, LAKAD, aims to navigate these problems and offer a tourist utility app for exploring destinations in the province of Bulacan. The system will be developed to feature core utilities that can enhance the travel experience of tourists. Among these features is an itinerary optimization, where the system ensures that travel routes are arranged in the most efficient way possible, minimizing time and cost while still maximizing convenience. The system also aims to provide a personalized itinerary generation for users to tailor travel plans according to their own preferences. Additionally, a tourist spot searching feature will be implemented to allow users to freely and easily browse Bulacan’s offered attractions and destinations, which includes heritage sites to natural sceneries. An itinerary navigation feature will help users arrange their trip destinations into a structured schedule according to their desired length of the travel, in addition to features that can guide tourists during their trip like providing directions and other similar functions. Lastly, to help users stay organized, an itinerary management feature will allow users to arrange and keep track of their chosen destinations.

\section{Statement of the Problem}

The lack of systems and applications for itinerary generation  in Bulacan makes it much harder for tourists to traverse the wonders the province has to offer. As such, the main objective of this study is to develop a mobile personalized itinerary generator for promoting tourist locations in Bulacan as well as providing optimized paths in the itinerary, allowing the tourist to further enjoy their trip. Specifically, this study aims to answer the following questions: 

\begin{enumerate}
    \item Which algorithm performs best in terms of run time and solution quality for solving the Traveling Salesman Problem (TSP) among the following:
    \begin{enumerate} [label*=\arabic*.] 
        \item Ant Colony Optimization (ACO),
        \item Genetic Algorithm (GA),
        \item Simulated Annealing (SA),
        \item Particle Swarm Optimization (PSO),
        \item Elephant Herding Optimization (EHO),
        \item Grey Wolf Optimizer (GWO), and
        \item Hovering Scouts and Foraging Flocks Pied Kingfisher Optimizer (HSFFPKO)?
    \end{enumerate}
    \item How can the proposed system be developed with the following functionalities:
    \begin{enumerate} [label*=\arabic*.] 
        \item  Itinerary Optimization,
        \item  Personalized Itinerary Generation,
        \item  Tourist Spot Searching,
        \item  Itinerary Navigation, and
        \item  Itinerary Management?
    \end{enumerate}
    \item How acceptable is the proposed system based on the criteria defined in the Technology Acceptance Model?
    \begin{enumerate} [label*=\arabic*.]
        \item Perceived usefulness,
        \item Perceived ease of use,
        \item Attitude towards using, and
        \item Behavioral intention?
    \end{enumerate}
    \item How well does the proposed system meet the ISO/IEC 25010:2023 requirements?
    \begin{enumerate} [label*=\arabic*.]
        \item Functional Suitability,
        \item Performance Efficiency,
        \item Compatibility,
        \item Interaction Capability,
        \item Reliability,
        \item Security,
        \item Maintainability,
        \item Flexibility, and 
        \item Safety?
    \end{enumerate}
\end{enumerate}

\section{Significance of the Study}
This study aims to develop an optimized mobile itinerary planner designed to promote tourism in Bulacan as a destination for tourists while providing functional and efficient navigation support for locals and commuters. The system functions as a tool that promotes the province's tourism industry, highlights its attractions, and makes trip planning easy by combining efficiency and personalization.

\textbf{Tourists.} The system provides personalized and optimized itineraries according to the user’s interest and preferred distance. In this way, they can enjoy both well-known and lesser-known parts of Bulacan's natural, historical, and cultural heritage while making the most of their time.

\textbf{Local Establishments.} The system gives them more chances to get their products and attractions seen because they can be included in the itineraries. This kind of advertising gets more customers involved and helps local businesses grow.

\textbf{Local tourism sector.} The system provides a tool to tourism offices and local tourism operators that enhances destination promotion, encourages tourist engagement, and improves the efficiency of travel planning.

\textbf{Local Communities.} The system provides residents and local communities with assistance in moving around the province and discovering nearby attractions. By improving mobility, the system helps develop the community and allows all people to appreciate the Bulacan’s cultural and historical sites more.

\textbf{Commuters.} The system provides an alternative tool designed to enhance daily travel efficiency for commuters within Bulacan. It brings practical convenience to people who rely on mobile guidance during regular transport because it offers optimized routing and accessible location information.

\textbf{Future Researchers.} This research can be a reference for researchers who would want to improve or further develop the tourism recommender system. The use of AGAM and TSP algorithms provides a foundation that others may extend, compare with different algorithms, or further improve by adding other features.

Overall, this study is significant because it supports development of tourism, sustainable development, and culture enhancement in Bulacan, and provides an example of how optimized and technology-driven solutions may enhance travel and navigation experiences.


\section{Scope and Limitation of the Study}

This study focuses on the development of a mobile personalized itinerary generator, designed specifically for the province of Bulacan, to address issues in manual trip planning and limitations of existing travel apps by offering a simpler, more user-centered tool that highlights the province’s cultural and historical attractions. The application will be an Android-based mobile app designed to generate personalized itineraries for tourists in Bulacan. The system aims to enhance the travel experience of tourists by providing functionalities such as itinerary optimization, personalized itinerary generation, tourist spot searching, itinerary navigation, and itinerary management. By integrating these features, the system seeks to make traveling within Bulacan more convenient, efficient, and engaging for visitors.

The scope of the POIs is limited to natural, historical, cultural, and heritage tourist destinations of Bulacan along with the numerous “Pasalubong Centers” within the province that will be obtained from, and validated by the  Provincial History, Arts, Culture, and Tourism Office (PHACTO). Tourist destinations outside Bulacan will not be considered. The system will not include restaurants as POIs due to their large number and there will be unfairness if small and unestablished restaurants are not included. The itinerary generation focuses on single-day travel sequences and users may pause or resume their visit, the system will not produce itineraries intended for multi-day tour planning. Furthermore, the system does not include public transportation routing because Mapbox, which is the mapping service used only to support walking, cycling, and driving directions, and does not provide transit or multi-modal transportation route information. Itinerary generation and recommendations will be based on weights assigned to each vertex or POI, which serve as inputs for the optimization process. Data for mapping and location will be sourced from platforms such as Mapbox and other publicly available datasets. The system will exclude the implementation of additional services such as accommodation booking, ticket reservations, or guided tour arrangements. The respondents will consist of tourists who are currently visiting Bulacan, regardless of whether they live within the province or in another location. These individuals will be the end-user of the system and participate in the evaluation to provide feedback.

The application will be developed using the React Native framework, which allows for cross-platform development and implementation. However, the mobile application will be developed exclusively for the Android operating system due to feasibility and cost considerations, as development for iOS requires a MacBook and an Xcode developer account, both of which require further funding of the project. The entirety of the study which includes the planning, drafting, writing, development, evaluation, and polishing of the paper will be carried out within an eight-month timeline.

Furthermore, personalized itinerary recommendation will utilize Adaptive Genetic Algorithm with Dynamic Mutation and Crossover Probabilities (AGAM), a model based on the Orienteering Problem (OP). The model will be used to generate itineraries. AGAM considers multiple objectives such as POI ratings, user-defined interests, and POI popularity to recommend and generate itineraries. For route optimization, the system will employ the best algorithm determined by a comparative analysis, applied within the framework of the TSP to yield the shortest possible routes for the itineraries. This ensures that the itineraries generated are both optimized and meaningful to the user’s preferences.

Additionally, this study involves testing and comparing the performance of selected algorithms used to optimize routes. The comparative analysis is limited only to the algorithms chosen by the researchers. The dataset that will be used to compare the algorithms will be obtained from existing graphs from TSPLIB, and then the results will be listed accordingly. To evaluate the chosen algorithms, the researchers will use the Friedman Test to find significant differences. Then the Nemenyi Post-Hoc Test will be used to identify which particular algorithms are different from the others.

The system’s quality and acceptability will then be quantitatively evaluated using two standard instruments, Technology Acceptance Model (TAM) and ISO/IEC 25010:2023. The evaluation for the end-users will utilize TAM to assess perceived usefulness, ease of use, and intention to use towards the application. In addition, the system will be assessed against the ISO/IEC 25010:2023 software quality standards, covering functional suitability, performance efficiency, compatibility, interaction capability, reliability, security, maintainability, flexibility and safety.



