\chapter{The Problem And Its Background}
\thispagestyle{empty}
Itinerary planning is an important process that can enhance the experience of tourists in their travel such as how well the places interest the user and the flow in which the places are traversed. In the case of the province of Bulacan, this study seeks to propose a mobile personalized itinerary generator as a possible solution to optimizing existing itineraries and providing rich and personalized itineraries of Bulacan tourism, based on the interests of the user. This chapter introduces the background, objectives, significance, and scope of the study.
 
\section{Background of the Study}

In the midst of advancement in technology and sciences, culture remains a vital part of many people’s lives. Tourism, being a significant contributor to preservation of culture and local development, relies heavily on how well travel experiences are delivered. One of the most crucial aspects of this process is itinerary planning, a practice ensuring the maximization of multiple points of interest (POIs) efficiently without the unnecessary inconvenience. The ability to create structured travel plans does not only influence convenience but also impacts tourist satisfaction and chance of returning or recommending the destination to others.

Sustainable and authentic trends in travel experiences align with Philippines’ Department of Tourism’s (DOT) initiatives to develop unique cultural and natural attractions \parencite{DOT2023}. Despite these efforts, historically active provinces like Bulacan still has some of its tourist destinations neglected, proven by how some culturally relevant sites like Barasoain Church, often attract more visitors than others like the Immaculate Conception Cathedral, with 2.6 times more visitors, despite being in the same municipality \parencite{Canet2025S}. Bulacan’s tourist arrival reports also show that certain municipalities acquire an uneven number of visitors for the first three quarters of the year 2025 (Provincial History, Arts, Culture, and Tourism Office (PHACTO), 2025). A problem that affects the development of Bulacan’s tourism, resulting in an overlook of economic opportunities for lesser-known municipalities.

These conditions highlight the need for smarter ways to plan trips, systems that can pull together all sorts of information and offer suggestions perfectly tailored to what each person likes. This is especially important as more and more people want to explore and truly experience a destination's uniqueness, which often demands a level of planning which outdated methods often fail to deliver \parencite{Halder2022}. The Bulacan Provincial Government is actively promoting their attractions through Bulacan’s Pamana Pass, a checklist of certain sites, but it only acts as a guide. Technology can help people utilize it more, and make trips more satisfying ensuring tourism spots grow in a way that benefits everyone.

Creating itineraries under many constraints has become increasingly automated through a mix of both optimization algorithms and content selection processes to resolve the scheduling and relevance aspects of itinerary recommendation \parencite{Jewpanya2025,Liu2024}. However, such systems are often limited to specific regions and bound by the capabilities of the model used or lack of data \parencite{Cui2025,papadakis2024,Yulfihani2024}. Despite the drive to develop systems that simplify the complicated problems of manual itinerary planning, existing travel apps still struggle to effectively recommend itineraries as reports from \textcite{SkiftState2022} highlights that applications may offer destination suggestions but lack the ability to offer suggestions based on user preferences. Their reports also show the demand for a personally curated service, since 30 percent of travelers responded that they are “more likely to use a travel agent now than before the pandemic”. Although there are systems that already cater to this problem, a common weakness they have is overcomplexity \parencite{Postnikova2024}. General-purpose platforms also generally do not align with the Philippines' strategic imperative to build a deeper, distinct Filipino experience \parencite{SkiftState2022,DOT2023}. These weaknesses expose the need for a simple, localized itinerary recommendation system, particularly in areas with uneven destination popularity like Bulacan. This study adopts Adaptive Genetic Algorithm (AGAM) developed by \textcite{Yochum2020} as a recommender algorithm to ensure suggestions does not merely list POIs, but instead consider user preferences.

Another problem of planning a travel itinerary arises when considering other constraints beyond personalization. A challenge that involves not only selecting POIs but also determining the optimal visiting order, persisting despite advances in recommendation and optimization methods \parencite{papadakis2024,Porras2022,Yulfihani2024}. These problems demonstrate that classic ways of planning itineraries lead to unsatisfactory results, implying the inherent value in creating automated systems for planning itineraries more efficiently. From a mathematical perspective, this problem is similar to the Traveling Salesman Problem (TSP) \parencite{Wu2020}. To address such complexity, this study uses an additional optimization algorithm that caters to TSP, which will help optimize the given routes of POIs.

These limitations, taken together, implies that manual travel planning methods in Bulacan remain inefficient due to uneven attraction popularity, lack of a personalized and optimized routing system, and the absence of accessible tools that integrate both user preferences and route optimization, which points to a need for a practical, user-centered, and localized solution.

This study proposes LAKAD to address the identified problems and provide a tourist utility app for exploring destinations in the province of Bulacan. The system will be developed to deliver two core features: a personalized itinerary generation, helping tourists better plan their trips according to their specifics, and an optimization algorithm to ensure routes are arranged in the best way possible. By combining the ideas of, and integrating personalized preference modeling and route optimization, LAKAD seeks to provide a simplified and localized solution. Ultimately contributing to DOT’s strategic goals by promoting Bulacan’s cultural identity and mitigating imbalance in the tourism destinations.


\section{Statement of the Problem}

The lack of systems and applications for itinerary generation  in Bulacan makes it much harder for tourists to traverse the wonders the province has to offer. As such, the main objective of this study is to develop a mobile personalized itinerary generator for promoting tourist locations in Bulacan as well as providing optimized paths in the itinerary, allowing the tourist to further enjoy their trip. Specifically, this study aims to answer the following questions: 

\begin{enumerate}
    \item Which algorithm performs best in terms of run time and solution quality for solving the Traveling Salesman Problem (TSP) among the following:
    \begin{enumerate} [label*=\arabic*.] 
        \item Ant Colony Optimization (ACO),
        \item Genetic Algorithm (GA),
        \item Simulated Annealing (SA),
        \item Particle Swarm Optimization (PSO),
        \item Elephant Herding Optimization (EHO), and
        \item Grey Wolf Optimizer (GWO)?
        % \item Hovering Scouts and Foraging Flocks Pied Kingfisher Optimizer (HSFFPKO)?
    \end{enumerate}
    \item How can the proposed system be developed with the following functionalities:
    \begin{enumerate} [label*=\arabic*.] 
        \item  Itinerary Optimization,
        \item  Personalized Itinerary Generation,
        \item  Tourist Spot Searching,
        \item  Itinerary Navigation, and
        \item  Itinerary Management?
    \end{enumerate}
    \item How acceptable is the proposed system based on the criteria defined in the Technology Acceptance Model?
    \begin{enumerate} [label*=\arabic*.]
        \item Perceived usefulness,
        \item Perceived ease of use,
        \item Attitude towards using, and
        \item Behavioral intention?
    \end{enumerate}
    \item How well does the proposed system meet the ISO/IEC 25010:2023 requirements?
    \begin{enumerate} [label*=\arabic*.]
        \item Functional Suitability,
        \item Performance Efficiency,
        \item Compatibility,
        \item Interaction Capability,
        \item Reliability,
        \item Security,
        \item Maintainability,
        \item Flexibility, and 
        \item Safety?
    \end{enumerate}
\end{enumerate}

\section{Significance of the Study}
This study aims to develop an optimized mobile itinerary planner designed to promote tourism in Bulacan as a destination for tourists while providing functional and efficient navigation support for locals and commuters. The system functions as a tool that promotes the province's tourism industry, highlights its attractions, and makes trip planning easy by combining efficiency and personalization.

\textbf{Tourists.} The system provides personalized and optimized itineraries according to the user’s interest and preferred distance. In this way, they can enjoy both well-known and lesser-known parts of Bulacan's natural, historical, and cultural heritage while making the most of their time.

\textbf{Local Establishments.} The system gives them more chances to get their products and attractions seen because they can be included in the itineraries. This kind of advertising gets more customers involved and helps local businesses grow.

\textbf{Local tourism sector.} The system provides a tool to tourism offices and local tourism operators that enhances destination promotion, encourages tourist engagement, and improves the efficiency of travel planning.

\textbf{Local Communities.} The system provides residents and local communities with assistance in moving around the province and discovering nearby attractions. By improving mobility, the system helps develop the community and allows all people to appreciate the Bulacan’s cultural and historical sites more.

\textbf{Commuters.} The system provides an alternative tool designed to enhance daily travel efficiency for commuters within Bulacan. It brings practical convenience to people who rely on mobile guidance during regular transport because it offers optimized routing and accessible location information.

\textbf{Future Researchers.} This research can be a reference for researchers who would want to improve or further develop the tourism recommender system. The use of AGAM and TSP algorithms provides a foundation that others may extend, compare with different algorithms, or further improve by adding other features.

Overall, this study is significant because it supports development of tourism, sustainable development, and culture enhancement in Bulacan, and provides an example of how optimized and technology-driven solutions may enhance travel and navigation experiences.


\section{Scope and Limitation of the Study}

This study focuses on the development of a mobile personalized itinerary generator, designed specifically for the province of Bulacan, to address issues in manual trip planning and limitations of existing travel apps by offering a simpler, more user-centered tool that highlights the province’s cultural and historical attractions. The application will be an Android-based mobile app designed to generate personalized itineraries for tourists in Bulacan. The system aims to enhance the travel experience of tourists by providing functionalities such as itinerary optimization, personalized itinerary generation, tourist spot searching, itinerary navigation, and itinerary management. By integrating these features, the system seeks to make traveling within Bulacan more convenient, efficient, and engaging for visitors.

The scope of the POIs is limited to natural, historical, cultural, and heritage tourist destinations of Bulacan along with the numerous “Pasalubong Centers” within the province that will be obtained from, and validated by the  Provincial History, Arts, Culture, and Tourism Office (PHACTO). Tourist destinations outside Bulacan will not be considered. The system will not include restaurants as POIs due to their large number and there will be unfairness if small and unestablished restaurants are not included. The itinerary generation focuses on single-day travel sequences and users may pause or resume their visit, the system will not produce itineraries intended for multi-day tour planning. Furthermore, the system does not include public transportation routing because Mapbox, which is the mapping service used only to support walking, cycling, and driving directions, and does not provide transit or multi-modal transportation route information. Itinerary generation and recommendations will be based on weights assigned to each vertex or POI, which serve as inputs for the optimization process. Data for mapping and location will be sourced from platforms such as Mapbox and other publicly available datasets. The system will exclude the implementation of additional services such as accommodation booking, ticket reservations, or guided tour arrangements. The respondents will consist of tourists who are currently visiting Bulacan, regardless of whether they live within the province or in another location. These individuals will be the end-user of the system and participate in the evaluation to provide feedback.

The application will be developed using the React Native framework, which allows for cross-platform development and implementation. However, the mobile application will be developed exclusively for the Android operating system due to feasibility and cost considerations, as development for iOS requires a MacBook and an Xcode developer account, both of which require further funding of the project. The entirety of the study which includes the planning, drafting, writing, development, evaluation, and polishing of the paper will be carried out within an eight-month timeline.

Furthermore, personalized itinerary recommendation will utilize Adaptive Genetic Algorithm with Dynamic Mutation and Crossover Probabilities (AGAM), a model based on the Orienteering Problem (OP). The model will be used to generate itineraries. AGAM considers multiple objectives such as POI ratings, user-defined interests, and POI popularity to recommend and generate itineraries. For route optimization, the system will employ the best algorithm determined by a comparative analysis, applied within the framework of the TSP to yield the shortest possible routes for the itineraries. This ensures that the itineraries generated are both optimized and meaningful to the user’s preferences.

Additionally, this study involves testing and comparing the performance of selected algorithms used to optimize routes. The comparative analysis is limited only to the algorithms chosen by the researchers. The dataset that will be used to compare the algorithms will be obtained from existing graphs from TSPLIB, and then the results will be listed accordingly. To evaluate the chosen algorithms, the researchers will use the Friedman Test to find significant differences. Then the Nemenyi Post-Hoc Test will be used to identify which particular algorithms are different from the others.

The system’s quality and acceptability will then be quantitatively evaluated using two standard instruments, Technology Acceptance Model (TAM) and ISO/IEC 25010:2023. The evaluation for the end-users will utilize TAM to assess perceived usefulness, ease of use, and intention to use towards the application. In addition, the system will be assessed against the ISO/IEC 25010:2023 software quality standards, covering functional suitability, performance efficiency, compatibility, interaction capability, reliability, security, maintainability, flexibility and safety.



