\chapter{Theoretical Framework}
\thispagestyle{empty}
This chapter combines significant findings from various literature and studies from tourism, exploration of the traveling salesman problem (TSP) in itinerary planning, and methods for recommendation for itinerary generation. This comprehensive review serves as the basis for the development of the Lakad application by identifying relevant factors such as optimization algorithms for TSP in itinerary planning and factors to consider when creating personalized recommendation features of the system.

\section{Relevant Theories}
Key features of Lakad rely on established theories of mathematics. This section examines the foundational theories and explains how they are used and related to the concept of itinerary optimization and personalized generation.

\subsection{Graph Theory}
Graph Theory is a branch of mathematics that is still relatively young but is developing quickly. The foundation of graph theory goes back to the 18th century, when Leonhard Euler solved the Königsberg Bridge Problem, which is often thought of as the first problem in the field.  Euler’s work introduced this approach of representing real-life connections through vertices and edges, which forms the basis of all network studies and relations. In the early 20th century, mathematicians like Kuratowski, Wagner, and Whitney made the field bigger by studying planar graphs and coming up with basic ideas that later became Graph Minor Theory, which connected graph structures to geometry and topology. The proof of the Four Color Theorem in 1976 made graph theory even more popular and showed how useful it could be for solving complex mathematical and computational problems \parencite{Carmesin2022}.

A graph is a set of vertices and edges, where each edge connects exactly two vertices. Topologically, a graph can be thought of as a one-dimensional simplicial complex \parencite{Carmesin2022}. A graph $G=(V,E)$ is composed of a set of vertices $V$ and a set of edges $E$. There are two types of graphs, directed graphs, which have edges that can only go in one direction, and undirected graphs, where edges allow movement in both directions. Additionally, they can be weighted, it means that each edge has a number that represents different factors like cost, time, and distance.

In this study, the concept of graph theory was applied to represent the points of interest (POI) as nodes and the distance between every connected vertex as the edges. This graphical representation of cities and their distances is one of the common applications of graph theory and allows access to powerful algorithms and methods for optimization.

\subsection{Traveling Salesman Problem}

This is a specific problem under the field of graph theory. According to \textcite{Pop2024}, the Traveling Salesman Problem (TSP) has been in the history of combinatorial optimization since 1930, and was properly provided a mathematical formulation by Merrill M. Flood. TSP became very popular since it is a widely investigated optimization problem, often serving as a benchmark for modern optimization algorithms. The problem states that, given a set of cities, the salesman’s goal is to find the shortest possible route that visits each city exactly once.  According to \textcite{Wu2020}, TSP is considered a non-deterministic polynomial time problem or simply an NP-hard problem, meaning it is hard to solve efficiently and finding the exact solution takes a lot of time. Hence, one of the best way to solve it is through heuristic algorithm, where the best solution is not necessarily the optimal but the solution that takes reasonable of amount of time to solve \parencite{Sabery2023}. 

A Study by \textcite{Violina2021}, analyzes two exact algorithms to solve TSP, the brute force algorithm and the branch and bound algorithm. They conclude that solving TSP with exact algorithm results in optimal outcome but less effective. On the other hand, independent studies by \textcite{Wu2020} and \textcite{Wadi2025}, utilizes metaheuristic to solve TSP, and found that it is much better than using exact algorithms. It can easily be seen how the idea of TSP can be applied to itineraries. Each location of itineraries will be nodes and the distances between them are the edges. The goal of the system for the itinerary is to find the route which covers the least distance \parencite{Pop2024}. These concepts of TSP are used in the route optimization feature of the system, where the system will find the optimal route for the generated itinerary.  

\subsection{Orienteering Problem}
The Orienteering Problem (OP) is a routing problem that aims to maximize the total rewards collected along a route within a given travel budget \parencite{yu2022}. Similar to this, \textcite{morandi2024} explains OP as the task of planning a route for one vehicle that has to follow a travel budget. The problem is depicted as a graph where the roads have distances and the locations have points or reward values. The goal in this problem is to create a path that starts and ends at the depot, stays within the travel limit, and collects as many rewards as possible from the places visited. Similarly, it originated from a sport of the same name, where participants visit check-points with pre-determined scores, in an attempt to maximize their total score within a specific time \parencite{Lim2019}. \textcite{pvenivcka2019} described OP as a type of problem in operations research, introduced to it in 1984 by Tsiligirides. They then explained that OP is a problem that combines the combinatorial optimization problem Knapsack Problem (KP) with TSP. These two work separately as TSP, given a set of customers, seeks to find the sequence in which to visit selected customers, constrained within a budget, while also minimizing the tour path. The subset of selected customers is selected by the KP, where it maximizes the collected profit based on the selection of customers to be visited within the budget constraints. This was used in the itinerary generation feature of the system where the user provides input for the time or distance constraint to generate an itinerary for the user. This is supported by two independent studies by \textcite{Yochum2020} that utilize Adaptive Genetic Algorithm with Dynamic Mutation and Crossover Probabilities and, \textcite{Tenemaza2020} by modeling the Tourist Trip Design Problem (TTDP) as a Time-Dependent Orienteering Problem with Time Windows (TDOPTW) in their itinerary planning systems.

\subsection{Simple Additive Weighting}

Simple Additive Weighting (SAW), is one of the most widely used methods for Multi-Criteria Decision Making (MCDM). It is used to evaluate and rank a set of alternatives based on multiple and often conflicting criteria. SAW works by calculating a single performance score for each alternative by summing the weighted performance ratings of those alternatives across all criteria \parencite{Hamed2023}. 

The implementation of SAW follows a structured process to transform diverse criteria into a comparable format. To aggregate different criteria, the data must be normalized first, this is essential because decision matrices often contain data with different units of measurement \parencite{Hamed2023}. \textcite{Malefaki2025} suggests that min-max normalization technique is the most appropriate for SAW, as it offers robustness regarding small variations in initial data while remaining sensitive to significant shifts in trends. Another critical component of SAW is the assignment of weights to each criterion, which reflects their relative importance in the decision-making context. These weights must sum to 1 (or 100 percent). Weighting can be derived mathematically from the data itself  (objective weighting), or subjectively chosen based on the preference, knowledge, and experience of the decision-maker \parencite{Odu2019}. The final score for each alternative is calculated by summing the products of the normalized ratings and their respective weights. The alternative with the highest or lowest total score, depending on the objective, is considered the optimal solution \parencite{Hamed2023}. 

In the context of this study, SAW was used to select the best algorithm for route optimization. The criteria for selection include accuracy and speed. Each algorithm was evaluated based on these criteria, and the one with the best overall score was chosen for implementation in the Lakad application.

\subsection{Technology Acceptance Model}
The Technology Acceptance Model (TAM) is one of the well-known theories in explaining and predicting users’ acceptance of information systems. Fred D. Davis developed this theory in 1989 to understand the psychological factors affecting individual’s acceptance and usage of technology. According to \textcite{Marikyan2025TAM}, the model is based on the Theory of Reasoned Action (TRA), which highlights how an individual’s attitude and subjective norms influence their behavioral intention. By concentrating on the variables that influence users’ attitudes toward using a particular system, TAM modifies this framework to fit the technological environment.

The model provides a theoretical framework for assessing how likely it is that technology will be adopted in different situations. It has been one of the most widely applied models due to its simplicity and strong explanatory capacity to pinpoint the primary psychological factors influencing users’ adoption of new technologies. Through this framework, researchers and developers can analyze how users perceive a system and identify where modification can be made to enhance acceptance and satisfaction \parencite{Marikyan2025TAM}. TAM postulates that the acceptance of technology is primarily determined by an individual’s behavioral intention (BI), which is shaped by two cognitive responses: Perceived Usefulness (PU) and Perceived Ease of Use (PEOU) \parencite{Marikyan2025TAM}. According to \textcite{Aburbeian2022}, these two constructs are considered the most critical variables influencing the use or rejection of new technology.


\subsection{Software Product Assessment Framework (ISO/IEC 25010:2023)}

ISO/IEC 25010:2023 is part of the Systems and Software Quality Requirements and Evaluation (SQuaRE) series, and it identifies the models to be used for describing and evaluating the quality of software and system products. It provides common terminology and concepts that enable developers, consumers, and evaluators to specify quality requirements and to determine whether a product meets the requirements. The standard emphasizes that both functional and non-functional aspects of performance should be reflected in the defined characteristics and sub-characteristics used in the evaluation of the quality of the product \parencite{ISO25010Web,ISOIEC25010_2023}.

The model states that the product quality model consists of nine primary characteristics: functional suitability, performance efficiency, compatibility, interaction capability, reliability, security, maintainability, flexibility, and safety. Each characteristic is made up of sub-characteristics which describe quantifiable components of quality, like operability, scalability, coexistence, correctness, and completeness. To keep up with changing technological contexts, the 2023 revision includes revised terminology, replacing interaction capability for usability and flexibility for portability. This system ensures that from a theoretical basis, software products are evaluated consistently to meet user expectations for performance, reliability, and safety throughout their life cycle.


\section{Review of Related Literature}
This section presents studies relevant to the development of the developed application. It covers works on Bulacan tourism, optimization techniques for itinerary generation, and related approaches in personalized travel planning. By reviewing these studies, the foundation for the system was established and the selection of suitable methods for itinerary generation was justified.

\subsection{State of Tourism in Bulacan}

According to \textcite{Canet2024P}, tourism development in Bulacan is highly influenced by accessibility, infrastructure, promotion, and the preservation of cultural and natural heritage. While Bulacan shows great potential as a tourist destination because of its rich history and heritage, the issues such as weak promotion, limited facilities, and transportation continue to limit its growth. Following this, a study by \textcite{Canet2025S}, investigates why some historical and cultural destinations in Bulacan remain neglected despite the province's rich heritage. Their study reveals that popular landmarks like Barasoain Church and Basilica Minore de Immaculada Conception, both in Malalos City, attract the most visitors. On the other hand, attractions in other municipalities such as Meyto Shrine in Calumpit and Francisco Balagtas Museum in Balagtas, showed lower visitations. This imbalance reflects the uneven promotion of Bulacan’s tourist sites.

This uneven promotion aligns with the findings of \textcite{Canet2024P} that one of the issues is the weak promotion. Furthermore, \textcite{Canet2025S} examined the promotional strategies used by tourism offices and found that online promotions, social media campaigns, and partnership with bloggers and vloggers were the most effective methods of engaging younger audiences. While traditional approaches such as brochures, festivals and community-driven activities still play a role, digital strategies are better suited for Gen Z and millennial demographic. They emphasized the importance of collaboration among local government units, schools, communities, and tourism stakeholders to ensure consistent promotion and sustainable tourism development.

The local government of Bulacan has also recognized the importance of tourism in the province. In 2023, the Provincial Government of Bulacan launched the Bulacan Pamana Pass, a tourism initiative that focuses on local culture, heritage conservation, and community engagement. The Pass consists of twenty historical and cultural landmarks in the province. The landmarks include Barasoain Church, Casa Real de Malolos, the Malolos Cathedral, and the historic train stations in Guiguinto and Meycauayan, offering a structured route and incentives for completion \parencite{Velasco2023}. While the Heritage Pass strengthens heritage preservation and promotional efforts, it does not provide optimized travel sequencing for visitors, leaving tourists to plan routes on their own. This gap underscores the need for systems that not only identify attractions but also generate efficient itineraries tailored to tourists’ time and travel constraints.

\textbf{Digital Adoption of Tourism in Bulacan.} Digital strategies also play a critical role in Bulacan’s tourism promotion, \textcite{Canet2023SanRafael}, studied how social media content is disseminated to the users in San Rafael, Bulacan and found that 57 percent agreed that social media helped them determine their next choice of destination and 61 percent agreed that social media is a good strategy to make more tourist locations known to people. In line with this, \textcite{DelaCruz2022} also surveyed 100 tourists that visited Dona Remedios Trinidad, Bulacan to determine the effectiveness of using social media in promoting a tourist destination and asked questions ranging from how social media influences traveler’s decisions to the problems faced by tourists when using social media as a guide for tourism. They found that most travelers agreed that social media does influence the choice of destination of travelers. While problems reported by tourists is the spread of false or misleading information, different expectations, and inaccurate photos / videos to different tourist destinations. Clearly, the importance and role of social media in making tourist locations well-known is irreplaceable, aligning with the findings of \textcite{Canet2025S}. However, it is also prone to misleading information, highlighting the need for verified and curated information of tourist attractions in Bulacan.

\subsection{Meta-heuristic Algorithm for Routing}

Route optimization can be modeled as a TSP, where it seeks the shortest possible route that visits a set of cities (nodes) exactly once. There are many algorithms that provide the solution to the TSP, one of those algorithms is the exact method. According to \textcite{Violina2021}, exact methods such as Brute Force (BF) and Branch and Bound (BB) algorithms are used to solve TSP. These methods involve evaluating each possible solution one at a time until all of them are explored, then comparing each solution and then selecting the smallest one. The exact algorithm guarantees the best solution, however it becomes inefficient when the number of cities becomes large. 

Another approach in solving TSP is through metaheuristic algorithms. These are often referred to as a Nature-inspired Optimization Algorithm, it is designed to find the optimal or near-optimal solutions to complex problems by mimicking the behavior of natural systems, processes, or phenomena \parencite{Sabery2023}. Generally, metaheuristic algorithms begin with an initial solution and follow a set of heuristic principles to explore the search space and iteratively improve it. These guidelines help the algorithm avoid local optima and locate globally optimal or nearly optimal solutions by directing the search process toward promising areas of the search space. 

In line with this, \textcite{BarbCiorbea2023} compared an exact method, Mixed Integer Linear Programming (MILP), with a metaheuristic approach, Ant Colony Optimization (ACO), for solving the TSP with time window constraints. The results of the exact approach shown is advantageous because it guarantees the best solution; however, its runtime becomes very high as the number of cities increases. For larger instances, the heuristic approach produced solutions much more quickly, although without the guarantee of global optimality. Despite this, both algorithms were able to achieve the best result in the tested scenario. Moreover, the metaheuristic demonstrated an additional advantage by aiming to minimize waiting duration at nodes when waiting could not be avoided. These findings emphasize that while exact methods provide precision, metaheuristic methods are better suited for larger and more time-sensitive problems \parencite{BarbCiorbea2023}. This demonstrates that while exact methods are strictly optimal with respect to accuracy, their exponentially growing time complexity renders them impractical to apply to large-scale or time-constrained situations. 

The goal of most metaheuristic algorithms is to find an optimal path to an instance of the TSP. The increased robustness was also found in the study of \textcite{Hossain2024} where they compared new optimization algorithms against old optimization algorithms for solving the TSP and concluded that the new algorithms demonstrated greater effectiveness than the old algorithms in medium-scale instances. The new algorithms used in the study which are the Artificial Bee Colony (ABC), Grey Wolf Optimization (GWO), and the Salp Swarm Algorithm (SSA) performed better routes on average compared to the classic algorithms such as the Genetic Algorithm (GA), Ant Colony Optimization (ACO), and Simulated Annealing (SA). This means that on average, the three newer algorithms produced more consistent results than the classic three. However, they also stated that although newer algorithms produced better outputs, traditional algorithms such as GA and ACO remained to demonstrate strengths for specific instances. While newer and complex algorithms outperform traditional ones in accuracy and robustness, their higher computation times may pose challenges for resource-constrained environments such as mobile devices.

On the other hand, a different comparative study by \textcite{Wadi2025} highlighted the capability of ACO and Particle Swarm Optimization (PSO) to provide a fast execution speed with minimal solution quality reduction, albeit PSO being parameter-sensitive. They also examined another swarm-based approach in the form of Elephant Herding Optimization (EHO), which consistently outperformed both ACO and PSO by achieving lower optimal costs while maintaining short execution times. Their study demonstrated the practicality of the swarm-based approaches in large-scale instances, over exact methods. There are also algorithms that focus on faster execution time at the cost of less solution quality to the route planning. These fast algorithms, however, are most suitable for hardware-related limitations such as processing power in mobile devices.

A more recent nature-inspired optimization technique related to route planning is the Hovering Scouts and Foraging Flocks Pied Kingfisher Optimizer (HSFFPKO). This algorithm enhances the original Pied Kingfisher Optimizer by adding two mechanisms that promote a better balance between exploration and exploitation in searching for a solution. The hovering scout component increases early-stage exploration by probing new regions of the solution space, while the foraging flock mechanism improves exploitation toward guiding candidate solutions to better areas. HSFFPKO showed strong performance in optimization on the CEC 2017 benchmark suite and on various engineering design problems by showing faster convergence and high solution quality compared with several swarm-based metaheuristics \parencite{DelaCruz2025HSFFPKO}.

Each discussed metaheuristic algorithm demonstrates their strengths and caveats. For instance, in the study of \textcite{Hossain2024}, new algorithms (ABC, SSA, GWO) generally have better convergence and scalability than older algorithms (GA, ACO, SA). In terms of speed and accuracy, SA and PSO show fast runtime but sacrifice accuracy. However, the algorithms in the study of \textcite{Hossain2024} and \textcite{Wadi2025} were tested under different tests. Hence, to find the best algorithm among them, each must be evaluated under the same test. Furthermore, the review also shows that the exact algorithm is impractical due to the time it takes to find the exact solution, thus excluding it in the algorithm to be evaluated. The best performing algorithm in terms of speed and accuracy was the algorithm implemented in the study’s developed mobile application.


\subsection{Itinerary Recommendation and Generation}

An itinerary is defined as a structured sequence of selected places, also called Point of Interest (POI), from a set of destinations in an area. Itineraries often include activities that are offered by the POI. Every itinerary is planned accordingly under many constraints, especially in scheduling the order of visitation. These constraints include but are not limited by the route distance, time-allocation, and popularity. By planning an itinerary, tourists can utilize it to serve as a practical roadmap for their travels. These itineraries are not only useful to tourists, but are also an important application of what is called optimization algorithms and recommendation systems. Researchers often describe it as a tourist trip design problem (TTDP), a combinatorial planning task that finds the optimal visiting order of POIs under many constraints \parencite{Ruiz-Meza2022systematic}. 
	
A recommendation system is often included in itinerary generators as explored in the study of \textcite{Otaki2025} where they create a Travel recommender systems (TRSs) that recommend the most relevant itineraries for the users. One of the popular recommendation methods is the Content-based Filtering (CBF). It uses a feature list of items and compares it with items preferred by a specific user previously. The items that match in similarity are recommended to the user. CBF works by storing user profiles based on item features which are most commonly preferred by the user. These features are used to map the similarity of one item with another by similarity equation. Then, it compares each item’s features with the user profile and recommends it based on the degree of similarity \parencite{Raghuwanshi2019}.  The other popular recommendation method is through Collaborative Filtering (CF). This method is used to predict a user’s preferences or opinions by using the collective information of other users of the system. It basically analyzes the similarity between user’s preferences and provides a recommendation that way \parencite{Li2023,Widayanti2023}. However, both CBF and CF suffer from a cold start problem where it requires a large enough data of users.
	
Another approach for itinerary recommendation is based on the Orienteering Problem (OP). According to \textcite{Lim2019} Tour recommendation has its roots in the OP and similar variants where a key feature is that they do not incorporate any personalization for individual users. However, a study by \textcite{Yochum2020} proposed an Adaptive Genetic Algorithm with dynamic crossover and mutation probabilities (AGAM) to handle personalized multi objective itinerary planning. Their approach integrates mandatory POIs, popularity ratings, visit duration, travel time, and costs into a weighted fitness function. By using data from platforms like TripAdvisor, GoogleMaps, and Flickr, AGAM adapts during evolution to avoid stagnation and produce diverse solutions. They tested the model with a dataset for six cities namely Budapest, Edinburgh, Toronto, Glasgow, Perth, and Osaka. AGAM was compared against two baseline heuristics, MaxN a greedy approach that prioritizes the maximum number of POIs, and MaxP also a greedy approach that prioritizes POIs with the highest rating. The results show that while MaxN and MaxP performed better at including mandatory POIs, AGAM generated richer itineraries with higher-rated POIs. AGAM shows more effective use of time budgets, and improved user enjoyment, but at the expense of higher cost and fewer mandatory POIs. This study shows that an algorithmic approach can provide personalized itinerary for tourists.

The algorithmic approach of \textcite{Yochum2020} highlights adaptability through genetic algorithms and diverse real world data sources to produce personalized itineraries and AGAM prioritizes richer and higher rated itineraries even at the cost of efficiency. Additionally, this approach solves the shortcomings of CBF and CF such as: CBF doesn’t have much control over route-level constraints like travel time, distance and budget. Due to the shortcomings of CF and CBF and since the proposed system is a standalone mobile application with no external server, an algorithmic approach is the desired approach for TRS. Hence, the algorithmic approach of \textcite{Yochum2020} is the framework to be implemented on the proposed system because it combines recommendations and itinerary generation through the AGAM framework.

\subsection{Mobile Itinerary Generator}

Itinerary generators have been adopted in mobile devices, and various studies have consistently highlighted the role of smartphones in automating and personalizing trip planning. A study by \textcite{Estilo2023} developed a mobile application named Destamp, it is a comprehensive planning tool that unifies itinerary generation, budget tracking, and recommendations for Iloilo city.  Destamp uses Genetic Algorithm combined with CBF for recommendations. In contrast, \textcite{Loh2022} developed an itinerary planner with smart trip generation and social platform that takes a more generic tourism perspective.  Loh’s app includes itinerary management, auto build route function, and optional manual editing. \textcite{Rahman2025} work on iGuid Ipoh positioned the app as a smart-tourism solution for a heritage city, integrating itinerary management with location-based services (LBS) and interactive maps to promote local attractions.

In terms of personalization, \textcite{Yulfihani2024} developed a tourism recommendation system for Batang Regency in Indonesia. Their system compares different categories based on the user's preference making the recommendation more tailored fit for the user. For \textcite{Estilo2023}, user’s preferences are expressed through financial constraints and attraction choices within Iloilo City. \textcite{Rahman2025} iGuide Ipoh focuses more on contextual and spatial personalization via LBS: the app helps users discover nearby attractions and manage trips while in the destination, but relies less on sophisticated preference modeling and more on location, maps, and curated city content. Loh’s system offers personalization through attraction categories and trip parameters (e.g., dates, starting points), then auto-generates a route that users may rearrange, thereby balancing automation with post-generation control \parencite{Loh2022}.

These studies complement each other, for instance the study of \textcite{Estilo2023} lacks itinerary management, which \textcite{Loh2022} and \textcite{Rahman2025} included in their study. Additionally, \textcite{Yulfihani2024}, although providing a great tourism recommendation system, didn't include interactive map capabilities just like the other mentioned studies. Lakad sought to implement these features to fill the shortcomings with each mentioned study. By unifying personalized recommendations, route optimization, itinerary management, and map-based navigation; Lakad positions itself as an improved and more holistic mobile solution for travel planning.


\section{Review of Related Studies}

This section presents different related studies that feature: itinerary optimization, where in a group of locations is selected and the order of visitation becomes an output; and personalized itinerary recommendation, where based on the user’s preferences and other factors, an itinerary is generated.

\subsection{Trippit: An Optimal Itinerary Generator}

Trippit by \textcite{Nguyen2019} addresses a fundamental problem in itinerary planning which is the optimization of the itinerary itself which is a fundamental limitation of Google Maps. Although Google Maps supports multiple stops, it leaves the sequencing to the user and often resulting in inefficient trip planning. Manual planning takes time and significant effort to optimize in terms of total distance traveled and how long each location to visit for. The study also stressed the fact that most travelers would often switch between Google Maps and Yelp for planning itinerary. The system eliminates the difficulty in travel planning which makes vacation trips more time efficient and enjoyable to travelers. 

The system was developed using the Waterfall method where each design phase is completed first before beginning the next. An android application was developed using React Native library and called network APIs such as the Google Places API and Foursquare Places API to calculate the optimized itinerary provided by the user. Users can input point of interests (POI) in which the application then retrieves the information from the respective APIs and calculates and outputs the optimized tour based on a greedy algorithm. Their system also calculates the best time to visit each input.
	
Trippit underwent several testing phases: primary testing, where they compared the results of the system to a known small list of itineraries; unit and integration testing, where they tested the logic and the correctness of the user interface; and alpha testing, with 10 users and was validated using a user satisfaction survey, in which most of the alpha users were satisfied with the resulting itinerary.  However, through the selection of the algorithm  and device came its limitation, which is that the application can only process up to 12 points of interest (POI). Its evaluation is also a concern as it didn’t use any standard evaluation methods like ISO/IEC 25010:2023 standard or the Technology Acceptance Model (TAM). Although a small prototype, Trippit can be considered a good enough system as it was able to produce good itineraries with a simple smartphone application without the use of advanced tools like Google OR.

However, the system focuses solely on itinerary optimization and does not include any recommendation features. Users must already know which POIs they want to visit and input them into the system. There is no mechanism to suggest POIs based on user preferences or popular attractions. Additionally, the greedy algorithm used may not always yield the globally optimal itinerary, especially as the number of POIs increases. The limitation of processing only up to 12 POIs also restricts its applicability for longer trips with more destinations. Future improvements could include integrating recommendation algorithms to suggest POIs and exploring more advanced optimization techniques to handle larger sets of locations.

\subsection{Optimization of Tourism Destination Recommendations in Batang Regency Using Content-based Filtering}

\textcite{Yulfihani2024} created a tourism recommendation system for Batang Regency in Indonesia. They applied Content-Based Filtering (CBF) for their recommendation system, using tourist preferences as their data source. Tourists interact with the system by adding destinations to a wishlist. The system then uses the wishlist to learn the tourist’s preferences. The recommendation factors category similarity where it compares the categories of wishlist items like nature, culture, culinary, and leisure; and location similarity by calculating the distance between destinations using the Haversine formula, which finds the shortest distance between two points on Earth using latitude and longitude.  The total similarity then calculated with 70 percent category similarity and 30 percent location similarity. The system ranks all available attractions by their total similarity score and the top 10 destinations are recommended to the tourist. The system was tested in different scenarios such as single item in wishlist, multiple items from same category with same/different locations, and multiple items from mixed categories. They use precision, recall, and F1 score as a metric for the system. 

The system was designed with a three-tier architecture that includes Backend, Web Admin Panel, and Mobile Application. The backend was built using Laravel (PHP framework) and MySQL, it acts as the hub for all operations including user authentication, processing recommendations using the CBF algorithm, and managing communication between the mobile app, admin panel, and database. The Web admin panel is for administrators and also developed using Laravel, it enables admins to manage tourist destinations, categories (nature, culture, culinary, etc.). Any changes made in the admin panel updates the backend database which instantly reflects to the mobile apps. The Mobile application was developed using Android Studio and Kotlin, it connects the backend using RESTful APIs via Retrofit. The app provides tourists with Login/registration, wishlist management, personalized recommendation based on wishlist and CBF algorithm, viewing details of attractions, and Filtering by categories and proximity.

The system provides highly accurate and relevant recommendations for tourists, especially in simple preference scenarios. The Content-based filtering approach worked well for aligning recommendations with tourists interests, and achieving a very high F1 score of 0.965. However, performance dropped slightly in complex scenarios like when the wishlist items mixed with different categories and locations. The Authors recommend future improvements such as expanding the dataset, using hybrid models (CBF + collaborative filtering), and adding user feedback mechanisms to further refine recommendations. 

However, their study is solely focused on the recommendation aspect of the system, and not on the itinerary optimization. While the system can recommend destinations based on user preferences, it does not optimize the sequence of visits or consider travel constraints like time and distance. Additionally, they didn’t include an evaluation for the system like  ISO/IEC 25010:2023 standard or the Technology Acceptance Model (TAM), only the evaluation for the Content-based filtering was provided.

\subsection{Visit Planner: A Personalized Mobile Trip Design Application based on a Hybrid Recommendation Model}

Visit Planner (ViP) a mobile application prototype developed by \textcite{papadakis2024}. It offers personalized recommendations for an itinerary based on user preference which are either explicitly collected by the application or assessed through the user’s behavior within it. ViP utilizes an Expectation Maximization method to offer the user an itinerary that tailors to their satisfactory needs while taking into consideration time and spatial constraints that concern both the user and the destination. The application currently focuses on the city of Agios Nikolaos in Crete. The system requires a user to create a secured profile in a registration process, wherein demographic information and essential data for the algorithm is collected. After this registration process, the user is also asked to specify their preferences in three different ways of their choice, one is by rating categories of POI, another choice is by stating if they like or dislike the category, and another one is by selecting their most preferred category. These categories are carefully specified by analyzing responses collected from 150 visiting tourists, in addition to local and expert knowledge in Agios Nikolaos.

The architecture of the system consists of main components like the front-end User Interface, back-end database, the middleware for processing the information, the recommendation components, and the itinerary creation components. The front-end of the application is an android app available in Google Play. The back-end is composed of MariaDB databases for the purpose of storing necessary information of users and POIs for smooth operations and functionalities. The middleware, built upon the basis of Spring Boot Framework aims to process the controlling flow between the system components and perform the functionalities. It works by receiving queries from the front-end, and then sending those to the recommendation algorithms. The recommendation and itinerary creation components comprises four several recommendation algorithms to tender the needs of the user, where each user is assigned a different algorithm for their cases in performance evaluation wise. ViP integrates multiple, novel recommendation algorithms which can adapt to different user profiles. 

The algorithms include model-based collaborative filtering that uses synthetic coordinate based system for recommendations (SCoR), hierarchical content-based similarity measures, a hybrid content-based recommendation using Weighted Extended Jaccard approach combined with the second one, and Bayesian recommendation algorithm. Any of these algorithms is applied to generate personalized POI suggestions. And once candidate POIs are retrieved, an expectation-maximization-based itinerary creation component sequences them into a feasible route by maximizing user satisfaction function constrained under temporal and spatial factors.


\subsection{I-AIR: Intention-aware travel itinerary recommendation via multi-signal fusion and spatiotemporal constraints}

\textcite{Cui2025} developed I-AIR (Intention-Aware Itinerary Recommendation), a system based on deep learning that aimed to generate personalized and practical traveling itineraries. The two primary components of the system are an Itinerary Construction Policy, which organizes selected Points of Interest (POIs) into a logical and sequential trip, and an Intention-Aware POI Scoring Model, which assesses the relevance of possible POIs based on user preference and contextual cues. To model user preferences, I-AIR applied a Transformer-based sequential encoder that identified time patterns from check-in history, an explicit feedback encoder that synthesized long-term preferences from ratings and likes, and a Graph Convolutional Network (GCN) that represented POI relationships through co-visitation patterns. These components were integrated via a multi-signal fusion layer to generate customized POI relevance scores. In optimizing the route, the system applied a greedy algorithm that picked the top-scoring viable POI at each step while removing possibilities that violate prohibitions such as travel time, wait time, business hours, and total time budget. The model was trained using a combination of next-POI prediction and explicit feedback reconstruction on datasets including user trajectories, ratings, dwell times, and POI attributes.
 
During evaluation, the system was tested on eight real word datasets, including four each of theme parks and city tourism. These datasets included detailed information such as user check-in records, wait times at attractions, and tourists’ reviews, and provided implicit and explicit feedback signals. The evaluation measures applied were precision, recall, and F1-scores, and comparative baseline systems included were Greedy-Popular (GPop), Greedy-Near (GNear), PersQ, EffiTourRec, DCC-PersIRE, BERT-Trip, and DLIR. Results showed that I-AIR outperformed the other systems significantly across all datasets, with higher precision in both theme park and city tourism scenarios.

\subsection{Destamp: A Mobile Application for Generating Budget-Based Itineraries to Enhance Travel Planning}

\textcite{Estilo2023} have proposed a very similar system to the study’s proposed application where itinerary generation and itinerary optimization are the main core features. The system of \textcite{Estilo2023} is a mobile application that uses GA to generate an itinerary for the user. However, unlike Lakad, the GA used in the study is univariate and does not consider other factors that may affect the user’s experience on the generated itineraries such as the user’s categorical preferences and POI ratings. The main objective function implied in the study is the minimizing of the itinerary travel budget depending on the maximum budget provided by the user and fitting the opening hours of each POI to the specified starting time of the user.

Destamp was developed using the AGILE methodology and the React Native framework for developing cross-platform mobile applications. Mapbox service was also used to provide mapping functions such as routing and interactive map features. While the system is split into two sections for the architecture: frontend and backend. The frontend handles all the graphical user interface and interactions which sends requests to the backend for the processing and logic such as itinerary generation.

The application has many similar features to Lakad such as itinerary generation and live map interaction. Users can set multi-day itineraries as well as allocate budgets for the itinerary to be generated. Each POI in the app contains the map location, budget cost, opening hours, and contact information. Business owners can also use the app to submit their own shops and businesses to show up as POI in Destamp, making the whole POI list accessible and dynamic. 

While very similar to Lakad, Destamp is only bounded to the places in Iloilo City. It also implements a freemium model to its features where most features such as maximum itineraries are locked behind a subscription. It also uses GA for its itinerary generation but is lacking the option for the itinerary optimization through TSP model as their optimization refers to pair-wise optimization through Dijkstra algorithm instead. There is also a clear gap in itinerary management where itineraries cannot be edited and must be regenerated for any changes. Furthermore, the evaluation instrument of their study is not clearly defined, and the evaluators they considered consisted of ten respondents, in which they concluded that GA actually does improve travel recommendations.

Destamp limits their users from fully utilizing their application. This is different from Lakad which allows users full access to all the available features mentioned in Destamp such as itinerary generation and optimization. The itinerary generation of Lakad is based on the AGAM framework of \textcite{Yochum2020}, and applying TSP algorithms for route optimization. Finally, users can also fully manage their own itineraries allowing them to edit the POIs included in existing itineraries.

\section{Synthesis of the Review}

The reviewed works establish valid grounds for the developed mobile application’s technical framework. Concepts such as Graph Theory, the Traveling Salesman Problem, and Orienteering Problem provide the mathematical foundation for itinerary generation and route optimization. While exact algorithms guarantee the best solution, they are often impractical for mobile devices due to high computation times. Consequently, metaheuristic algorithms have been shown to outperform exact algorithms as explored in the study of \textcite{Hossain2024} and \textcite{Wadi2025}, offering a viable solution for mobile constraints. Furthermore, regarding personalization, literature suggests that while Content-Based Filtering is effective, it suffers from the “cold-start” problem and often provides the same recommendation as they are often deterministic in nature. In contrast, algorithmic approaches, particularly the AGAM by \textcite{Yochum2020}, offer greater flexibility by allowing the fitness function to account for multiple objectives such as user preferences, ratings, and time constraints. Therefore, AGAM is an applicable method for itinerary recommendation compared to content selection methods.

Mobile applications like Trippit, Destamp, and I-AIR proved the feasibility of smartphone-based route generation. However, the review highlights significant fragmentation in the current landscape as some features of existing mobile applications are mutually exclusive to each other. Some focus solely on budget (Destamp), while others prioritize simple routing without personalization (Trippit) or lack of comprehensive itinerary management features. Additionally, many existing studies fail to apply  evaluation frameworks like ISO/IEC 25010:2023 or the Technology Acceptance Model (TAM). This limits the reliability of their usability claims. Lakad fills these gaps by providing a verified generator specifically for Bulacan tourism.

Finally, the literature highlights the needs for this system for Bulacan to exist. Although digital strategies are crucial for promotion as stated by \textcite{Canet2025S}, but \textcite{DelaCruz2022} pointed out that reliance on social media often leads to misinformation. While the Bulacan government has initiated programs like the Pamana Pass, there is currently no provincial-level system that automates and optimizes personalized itinerary planning. Existing Philippine-based studies, such as those by \textcite{Estilo2023} are limited to specific cities (Iloilo city) and do not address the provincial scope required for Bulacan. Therefore, Lakad addresses these gaps by integrating verified local data with advanced optimization algorithms to create a personalized tourism experience in Bulacan.

\section{Conceptual Framework}
This section explains the key inputs, processes, and outcomes in development of the study using an Input-Process-Output (IPO) model.

\begin{figure}[H]
    \centering
    \caption{Input-Process-Output (IPO) Model}
    \includegraphics[width=\textwidth]{IPO.png}
    \label{fig:conceptualframework}
\end{figure}

\subsection{Input}
The input contains the review of the methods used by related studies and systems, which was used as foundation in the study. These methods consist of different algorithms used for solving the TSP and selecting which algorithm is the most applicable in the mobile platform as well as which recommendation model is a fit for itinerary recommendation. The second input is the gathering of tourist spot locations or point of interests (POI). These POIs are locations in Bulacan which can be considered as tourist attractions. This was gathered from and verified by the Provincial History, Arts, Culture, and Tourism Office (PHACTO) to assess the correctness of the POIs included in the system. Development tools refer to the hardware and software tools that was used to develop the application. These include operating systems such as Windows or Linux, integrated development environments (IDE) such as Visual Studio or Android Studio, and design pieces of software like Figma or Adobe Illustrator.

\subsection{Process}
The AGILE methodology was used in the software development of Lakad, specifically the Kanban method. Kanban method is  a  visual  process  management  system that can manage knowledge and work by considering the Just In Time (JIT) delivery approach \textcite{Alaidaros2021}. The first phase of the development was planning and analyzing the different requirements of the system such as flow charts, entity relationship diagrams, and use cases. Afterward, mockup user interface was made to fully outline the look and feel of the application. Building phase was the actual development of the core features of the application and testing and debugging refers to the testing of the correctness of the core features’ implementation. The power of AGILE methodology was used to iteratively cycle between building phase and testing to evaluation by users and professionals to quickly enhance features and get immediate feedback. The system went through rigorous evaluation by professionals through ISO 25010:2023 and through TAM by its intended users.

\subsection{Output}
The expected output of this study is a fully functional mobile itinerary generator and optimization system in the scope of Bulacan. Lakad application was able to optimize the route to be taken for the user’s itinerary as well as recommend personalized itineraries that may interest the user.

\section{Definition of Terms}
This section provides the definition of the terms used in the study, along with the operational aspects of the proposed system, to ensure clarity and a proper understanding of the concepts within the context of the research. 

\begin{description}
    \item[Cold Start.] A problem where an algorithm has not enough necessary input data from users to perform instructions.
    \item[Distance Matrix.] A table of pairwise distances between POIs used by the routing algorithms in the application.
    \item[Itinerary.] Planned sequence of tourist destinations that either reflects the user’s selected preferences or is made as a recommendation by the algorithm.
    \item[Itinerary Generation.] The process of selecting POIs that match the user’s preferences and constraints to form a recommended itinerary.
    \item[Itinerary Management.] A feature that allows users to organize, edit, and monitor their selected destinations and travel schedules.
    \item[Itinerary Navigation.] A feature that provides users with directions and guidance throughout their travel route.
    \item[Itinerary Optimization.] The process of planning the most efficient route or order of destinations for travel, determined through the use of optimization algorithms.
    \item[Pasalubong Centers.] Local shops or markets in Bulacan that sell souvenirs and local products, which can be shown when the user is traveling.
    \item[Personalized Itinerary Recommendation.] A feature that suggests travel routes and destinations made by the algorithm tailored to the individual preferences and constraints of the user.
    \item[Provincial History, Arts, Culture, and Tourism Office (PHACTO).] The government office responsible for promoting and preserving the history, arts, culture, and tourism of Bulacan province. Source of POI data.
    \item[Point of Interest (POI).] A destination or tourist spot specifically preferred by a tourist, such as historical sites, natural attractions, or cultural landmarks that can be included into an itinerary.
    \item[Tourist.] A person traveling from place to place for pleasure or interest. The tourist serves as the end-user who interacts with the mobile application.
    \item[Tourist Spot Searching.] A feature that enables users to browse, locate, and explore tourist destinations within the province.
    \item[Mapbox.] An online service that provides a mapping platform to offer navigation features, interactive maps, and location information for tourist destinations in Bulacan.
    \item[OpenRouteService (ORS).] An online service that provides routing and optimization capabilities, which can be used to calculate the most efficient routes for the itineraries generated by the application. 
    \item[Application Programming Interface (API).] A set of rules and tools that allows the application to access external services, like Mapbox, in order to get the map, routing, and location information required for navigation and itinerary creation.
    \item[Stops.] The specific destinations or POIs included in a user’s itinerary that they plan to visit during their trip. 
\end{description}





