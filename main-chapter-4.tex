\chapter{Presentation, Analysis, and Interpretation of Data}

This chapter presents the results of the study, including a comparative analysis of TSP algorithms,  the developed application, and the evaluation of system acceptability and software quality. The findings are interpreted in the context of the research objectives and questions outlined in Chapter 1.

\section{Comparative Analysis of TSP Algorithms}
This section presents the performance evaluation of the selected metaheuristic algorithms for solving the Traveling Salesman Problem (TSP). The algorithms were tested on four standard instances from the TSPLIB library: burma14, bays29, eil51, and berlin52. The primary goal was to identify the most suitable algorithm for the Lakad mobile application, balancing solution quality (minimizing travel distance) and runtime efficiency (minimizing user wait time).

\subsection{Analysis of Solution Quality (Accuracy)}

The solution quality is measured by the average relative error (\%) compared to the optimal solution for each TSP instance.

\begin{table}[H]
	\centering
	\singlespacing
	\caption{Summary of Average Relative Error (\%)}
	\label{tab:acc} \renewcommand{\arraystretch}{1.2}
	\begin{tabularx}{\linewidth}{XXXXXX}
		\toprule
        \multirow{2}{*}{\textbf{Algorithm}} & \multicolumn{4}{c}{\textbf{TSPLIB Instances}} & \multirow{2}{*}{\textbf{Average}} \\
        \cmidrule(lr){2-5}
         & \textbf{burma14} & \textbf{bays29} & \textbf{eil51} & \textbf{berlin52} &  \\
		\midrule
        GA & 0.00 & 0.03 & 0.38 & 0.00 & 0.10 \\
        SA & 0.00 & 0.42 & 2.77 & 3.15 & 1.59 \\
        ACO & 0.53 & 1.50 &  5.49 &  2.49 & 2.50 \\
        EHO & 0.00 & 3.71 &  20.47 &  23.36 & 11.27 \\
        PSO & 0.99 & 12.74 &  36.88 &   43.46 & 23.77 \\
        HSFFPKO & 1.88 &  20.18 &  127.28 &  95.07 & 61.11 \\
        GWO & 6.32 & 33.67 & 145.85 & 181.36 & 91.80 \\ 
		\bottomrule
	\end{tabularx}
\end{table}

Table \ref{tab:acc} summarizes the results of each algorithm showing their performance across the four TSPLIB instances and their average relative error. The results indicate that the Genetic Algorithm (GA) consistently achieved the highest solution quality, with an average relative error of 0.10\%. Simulated Annealing (SA) and Ant Colony Optimization (ACO) also performed well, with average errors of 1.59\% and 2.50\%, respectively. The Elephant Herding Optimization (EHO), Particle Swarm Optimization (PSO), Hovering Scouts and Foraging Flocks Pied Kingfisher Optimizer (HSFFPKO), and Grey Wolf Optimizer (GWO) showed significantly higher errors, indicating lower solution quality for the tested TSP instances.

\subsection{Analysis of Runtime Efficiency}

Runtime efficiency is critical for the Lakad application, as users expect quick responses when optimizing their travel itineraries. 

\begin{table}[H]
	\centering
	\singlespacing
	\caption{Summary of Average Runtime Efficiency (seconds)}
	\label{tab:runtime} \renewcommand{\arraystretch}{1.2}
	\begin{tabularx}{\linewidth}{XXXXX}
		\toprule
        \multirow{2}{*}{\textbf{Algorithm}} & \multicolumn{4}{c}{\textbf{TSPLIB Instances}}  \\
        \cmidrule(lr){2-5}
         & \textbf{burma14} & \textbf{bays29} & \textbf{eil51} & \textbf{berlin52} \\
		\midrule
        EHO     & 0.17 & 0.27   & 0.44      & 0.45   \\ 
        SA      & 0.49 & 1.04   & 1.82      & 1.90    \\
        GWO     & 0.51 & 1.02   & 1.73      & 1.75    \\
        HSFFPKO & 1.16 & 1.42   & 1.78      & 1.76    \\
        PSO     & 3.36 & 4.77   & 6.30      & 6.84    \\
        ACO     & 2.51 & 9.67   & 28.62     & 29.23   \\
        GA      & 3.79 & 26.31  & 168.15    & 196.59  \\
		\bottomrule
	\end{tabularx}
\end{table}

Table \ref{tab:runtime} summarizes the average runtime (in seconds) for each algorithm across the four TSPLIB instances. While GA achieved the best solution quality, its runtime increased exponentially with the problem size. For the berlin52 instance, GA required approximately 196.59 seconds (over 3 minutes) to complete. In contrast, EHO had the lowest runtime across all instances, with a runtime of $<0.5$ seconds even for the largest instance. SA and GWO also demonstrated low runtimes, while ACO and PSO showed moderate to high runtimes, especially as the problem size increases.

\subsection{Algorithm Ranking}
To select the most suitable algorithm for the Lakad application, a multi-criteria decision analysis (MCDA) approach was used to rank the algorithms based on both solution quality and runtime efficiency, particularly the Simple Additive Weighting (SAW) method. The per instance scores and rankings are presented in Table \ref{tab:rank}. Equal weights ($w _{i}= 0.5$) were assigned to both criteria.

\begin{table}[H]
	\centering
	\singlespacing
	\caption{Final SAW Scores and Rankings of TSP Algorithms}
	\label{tab:rank} \renewcommand{\arraystretch}{1.2}
	\begin{tabularx}{\linewidth}{XXXXXX}
		\toprule
        \multirow{2}{*}{\textbf{Algorithm}} & \multicolumn{4}{c}{\textbf{TSPLIB Instances}} & \multirow{2}{*}{\textbf{Rank}} \\
        \cmidrule(lr){2-5}
         & \textbf{burma14} & \textbf{bays29} & \textbf{eil51} & \textbf{berlin52} &  \\
		\midrule
        SA      & 0.06  & 0.03  & 0.01  & 0.01 & 1st \\
        EHO     & 0.02  & 0.06  & 0.07  & 0.07 & 2nd \\
        ACO     & 0.37  & 0.21  & 0.10  & 0.08 & 3rd \\
        PSO     & 0.52  & 0.28  & 0.14  & 0.14 & 4th \\
        HSFFPKO & 0.30  & 0.33  & 0.44  & 0.33 & 5th \\
        GA      & 0.50  & 0.50  & 0.50  & 0.50 & 6th \\
        GWO     & 0.56  & 0.51  & 0.51  & 0.50 & 7th \\
		\bottomrule
	\end{tabularx}
\end{table}

Based on the SAW scores, Simulated Annealing (SA) emerged as the top-ranked algorithm, offering a strong balance between solution quality and runtime efficiency. Although GA achieved the best accuracy, its long runtime makes it resulted in a consistently high SAW score, placing it in the 6th position. EHO performed well on small instances, but its accuracy degraded on larger instances, making it less reliable for complex itineraries. 

In conclusion, Simulated Annealing (SA) was the selected route optimization algorithm for Lakad. It provides a good trade-off between solution quality and runtime efficiency, ensuring that users receive optimized itineraries in a reasonable time frame, enhancing the overall user experience of the Lakad application.


\section{The Developed System}
The developed system, Lakad, was developed as both a place explorer and itinerary planner with route optimization capabilities. It allows tourists to search tourists spots within Bulacan and plan their itineraries by generating one for them and gives out navigation instructions for travel. The Lakad application has five key features, which are described in the following subsections.

\subsection{Tourist Spot Exploration}

\begin{figure}[H]
    \centering
    \caption{Tourist Spot Exploration}
    \begin{subfigure}[H]{0.3\textwidth}
        \centering
        \fbox{\includegraphics[width=\textwidth]{dev-system/mainpage.jpg}}
        \caption{Place Exploration}
        \label{fig:1a}
    \end{subfigure}
    \hfill 
    \begin{subfigure}[H]{0.3\textwidth}
        \centering
        \fbox{\includegraphics[width=\textwidth]{dev-system/searching2.jpg}}
        \caption{Search Places}
        \label{fig:1b}
    \end{subfigure}
    \hfill 
    \begin{subfigure}[H]{0.3\textwidth}
        \centering
        \fbox{\includegraphics[width=\textwidth]{dev-system/touristspotlist.jpg}}
        \caption{Place List}
        \label{fig:1c}
    \end{subfigure}
\end{figure}

Lakad allows user to explore tourist spots in Bulacan in an interactive manner. Figure \ref{fig:1a} shows a map which has markers for the respective tourist spots with their images. There is also a search bar that allows user to search for tourist spots by name, as shown in Figure \ref{fig:1b}. The user can also view the list of all tourist spots with their images, categories, locations, and ratings, as shown in Figure \ref{fig:1c}. This feature provides users with a comprehensive overview of the available tourist spots in Bulacan.

\begin{figure}[H]
    \centering
    \caption{Tourist Spot Information}
    \begin{subfigure}[H]{0.3\textwidth}
        \centering
        \fbox{\includegraphics[width=\textwidth]{dev-system/information.jpg}}
        \caption{Place Information}
        \label{fig:2a}
    \end{subfigure}
    \hfill
    \begin{subfigure}[H]{0.3\textwidth}
        \centering
        \fbox{\includegraphics[width=\textwidth]{dev-system/review.jpg}}
        \caption{Review and Rating}
        \label{fig:2b}
    \end{subfigure}
    \hfill
    \begin{subfigure}[H]{0.3\textwidth}
        \centering
        \fbox{\includegraphics[width=\textwidth]{dev-system/review2.jpg}}
        \caption{Review and Rating}
        \label{fig:2c}
    \end{subfigure}
\end{figure}

The tourist spot markers on the map are interactive, allowing users to click on them to view detailed information about the tourist spot, including its name, category, location, description, and user reviews and ratings as shown in Figure \ref{fig:2a}. There is also a button that allows users to quickly add the tourist spot to their itinerary. Scrolling down, users can view the reviews and ratings of the tourist spot, as shown in Figure \ref{fig:2b}. This feature provides users with valuable insights from other travelers. Figure \ref{fig:2c} shows more comprehensive reviews and ratings of the tourist spot by clicking the “See all reviews” button. The “All Reviews” shows the reviews and ratings of other users and allows sorting and filtering of the reviews.

\begin{figure}[H]
    \centering
    \caption{Reviews and Ratings}
    \begin{subfigure}[H]{0.3\textwidth}
        \centering
        \fbox{\includegraphics[width=\textwidth]{dev-system/makerating.jpg}}
        \caption{Write Review and Rating}
        \label{fig:3a}
    \end{subfigure}
    \hfil
    \begin{subfigure}[H]{0.3\textwidth}
        \centering
        \fbox{\includegraphics[width=\textwidth]{dev-system/report.png}}
        \caption{Report Review}
        \label{fig:3b}
    \end{subfigure}
\end{figure}

Users of Lakad can also create their own reviews and ratings for the tourist spots they have visited. They can rate the tourist spot using a star rating, write a review, and upload photos of their experience, as shown in Figure \ref{fig:3a}. Additionally, users can report inappropriate reviews to maintain the quality of reviews in the application, as shown in Figure \ref{fig:3b}. This encourages user engagement and helps build a community of travelers sharing their experiences while maintaining a healthy review environment.


\subsection{Personalized Itinerary Generation}

\begin{figure}[H]
    \centering
    \caption{Personalized Itinerary Generation}
    \begin{subfigure}[H]{0.3\textwidth}
        \centering
        \fbox{\includegraphics[width=\textwidth]{dev-system/agamgenerate.jpg}}
        \caption{Generate Itinerary}
        \label{fig:4a}
    \end{subfigure}
    \hfill
    \begin{subfigure}[H]{0.3\textwidth}
        \centering
        \fbox{\includegraphics[width=\textwidth]{dev-system/selectlocation.jpg}}
        \caption{Select Location}
        \label{fig:4b}
    \end{subfigure}
    \hfill
    \begin{subfigure}[H]{0.3\textwidth}
        \centering
        \fbox{\includegraphics[width=\textwidth]{dev-system/selecttype.jpg}}
        \caption{Select Type}
        \label{fig:4c}
    \end{subfigure}
\end{figure}

One of the core features of Lakad is its ability to generate personalized itineraries for users based on their preferences and selected tourist spots. 
The Adaptive Genetic Algorithm with Dynamic Mutation and Crossover Probabilities (AGAM) is the one responsible for generating personalized itineraries, and the required parameters are filled by the user, as shown in Figure \ref{fig:4a}. Users can select their preferred tourist spots location which are based on the districts and municipalities of Bulacan, as shown in Figure \ref{fig:4b}. Next, users must select the type of tourist spots they want to visit, which are based on the categories of the tourist spots, as shown in Figure \ref{fig:4c}. The maximum travel distance and the number of tourist spots to visit are also required parameters for itinerary generation. After filling in the required parameters, users can click the “Generate Itinerary” button to receive a personalized itinerary that optimizes their travel route based on their preferences and selected tourist spots.

\begin{figure}[H]
    \centering
    \caption{Generated Itinerary}
    \begin{subfigure}[H]{0.3\textwidth}
        \centering
        \fbox{\includegraphics[width=\textwidth]{dev-system/swipecard.jpg}}
        \caption{Card View}
        \label{fig:5a}
    \end{subfigure}
    \hfil
    \begin{subfigure}[H]{0.3\textwidth}
        \centering
        \fbox{\includegraphics[width=\textwidth]{dev-system/generated.jpg}}
        \caption{Generated Itinerary}
        \label{fig:5b}
    \end{subfigure}
\end{figure}

After generating the itinerary, a card view of a tourist spot in the itinerary is shown where users can swipe left or right to reject or accept the tourist spot, respectively, as shown in Figure \ref{fig:5a}. If the user accepts the tourist spot, it will be added to their itinerary, and they can view the generated itinerary (Figure \ref{fig:5b}). An autogenerated name for the generated itinerary is also provided, but users can edit it to their preference. The estimated time to complete the itinerary is also shown, and users can set a duration for each tourist spot in the itinerary. This feature allows users to have control over their itinerary and make adjustments based on their preferences and distance constraints.


\subsection{Itinerary Optimization}

\begin{figure}[H]
    \centering
    \caption{Itinerary Optimization}
    \label{fig:6}
    \begin{subfigure}[H]{0.3\textwidth}
        \centering
        \fbox{\includegraphics[width=\textwidth]{dev-system/pre-opt.jpg}}
        \caption{Pre-Optimization Itinerary}
        \label{fig:6a}
    \end{subfigure}
    \hfil
    \begin{subfigure}[H]{0.3\textwidth}
        \centering
        \fbox{\includegraphics[width=\textwidth]{dev-system/post-opt.jpg}}
        \caption{Post-Optimization Itinerary}
        \label{fig:6b}
    \end{subfigure}
\end{figure}

The initial generated itinerary only considers the maximum travel distance, ensuring that the total distance of the itinerary does not exceed the specified limit. However, it does not optimize the travel route between the selected tourist spots. Hence, Lakad incorporates the Simulated Annealing (SA) algorithm to optimize the travel route of the generated itinerary. Figure \ref{fig:6a} shows the initial arrangement of the tourist spots in the itinerary generated by AGAM. The “Optimize” button can then be clicked to optimize the travel route of the itinerary using SA. Figure \ref{fig:6b} shows the optimized arrangement of the tourist spots in the itinerary after applying SA. The first tourist spot in the itinerary is based on the shortest distance from the user's current location, and the subsequent tourist spots are arranged in a way that minimizes the total travel distance. As shown in 
Figure \ref{fig:6}, from 71.326 km to 31.250 km, the total travel distance for this itinerary was significantly reduced. This shows the effectiveness of the SA algorithm. Additionally, optimization process is done in a reasonable time frame.

There are also instances where the initial generated itinerary is the optimal solution, thus no changes are made to the arrangement of the tourist spots in the itinerary after optimization. 

\subsection{Itinerary Management}

\begin{figure}[H]
    \centering
    \caption{Itinerary Management}
    \begin{subfigure}[H]{0.3\textwidth}
        \centering
        \fbox{\includegraphics[width=\textwidth]{dev-system/itinerarymain.jpg}}
        \caption{Itinerary View}
        \label{fig:7a}
    \end{subfigure}
    \hfill
    \begin{subfigure}[H]{0.3\textwidth}
        \centering
        \fbox{\includegraphics[width=\textwidth]{dev-system/createitinerary.jpg}}
        \caption{Create Itinerary}
        \label{fig:7b}
    \end{subfigure}
    \hfill
    \begin{subfigure}[H]{0.3\textwidth}
        \centering
        \fbox{\includegraphics[width=\textwidth]{dev-system/deleteiti.jpg}}
        \caption{Delete Itinerary}
        \label{fig:7c}
    \end{subfigure}
\end{figure}

Users can see and manage their generated itineraries in the “Itineraries” page, as shown in Figure \ref{fig:7a}. The users can view their existing itineraries, including the details of each itinerary such as the number of tourist spots, total travel distance, date created, current progress, and the name of the itinerary. It also has a button to start or continue the itinerary, which will give navigation instructions to the user to travel to the tourist spots in the itinerary. Clicking the “New” button will prompt the user to create a new itinerary manually or smart generate an itinerary using the AGAM, as shown in Figure \ref{fig:7b}. Swiping left on an existing itinerary will show the delete button, allowing users to delete their itineraries, as shown in Figure \ref{fig:7c}.

\begin{figure}[H]
   \centering
    \caption{Manual Itinerary Creation}
    \label{manual}
    \fbox{\includegraphics[width=0.3\textwidth]{dev-system/addstop.jpg}}
\end{figure}

If the user chooses to create a new itinerary manually, they can select the tourist spots they want to visit and search it by name, as shown in Figure \ref{manual}. On the other hand, if the user chooses to smart generate an itinerary, they will be directed to the AGAM itinerary generation page (Figure \ref{fig:4a}) where they can fill in the required parameters and generate a personalized itinerary based on their preferences and selected tourist spots. This feature provides users with flexibility in creating their itineraries, allowing them to either have a personalized itinerary generated for them or create their own itinerary based on their preferences and selected tourist spots.

\subsection{Itinerary Navigation}

\begin{figure}[H]
    \centering
    \caption{Itinerary Navigation}
    \begin{subfigure}[H]{0.3\textwidth}
        \centering
        \fbox{\includegraphics[width=\textwidth]{dev-system/navigationdirection.jpg}}
        \caption{Directions}
        \label{fig:8a}
    \end{subfigure}
    \hfill
    \begin{subfigure}[H]{0.3\textwidth}
        \centering
        \fbox{\includegraphics[width=\textwidth]{dev-system/navigationarrival.jpg}}
        \caption{More Information}
        \label{fig:8b}
    \end{subfigure}
    \hfill
    \begin{subfigure}[H]{0.3\textwidth}
        \centering
        \fbox{\includegraphics[width=\textwidth]{dev-system/navigationmode.jpg}}
        \caption{Travel Mode}
        \label{fig:8c}
    \end{subfigure}
\end{figure}

Lakad also features navigation of an itinerary by showing the direction to the first unvisited tourist spot in the itinerary. Figure \ref{fig:8a} shows the direction by displaying the route from the user's current location to the first unvisited tourist spot in the itinerary. It also shows the estimated time of arrival in the tourist spot and the departure time based on the duration set by the user. Scrolling down, users can view more information such as the entire direction to the tourist spot, the next tourist spot in the itinerary, also a button to mark the tourist spot as visited, as shown in Figure \ref{fig:8b}. Users can also change the travel mode for navigation, which includes driving, walking, and bicycling, as shown in Figure \ref{fig:8c}. The user can choose to avoid tolls and turn on the voice navigation. This feature provides users with a convenient way to navigate their itineraries and ensures that they can easily find their way to the tourist spots they want to visit.  

Additionally, a stop is also marked as visited when the user is detected to be within a certain proximity of the tourist spot. This feature provides users with a seamless navigation experience, allowing them to focus on enjoying their trip while the application automatically tracks their progress through the itinerary.

\subsection{Admin Side}

This is an additional feature of the Lakad application that allows the application to be more manageable and scalable.

\begin{figure}[H]
    \centering
    \caption{Admin Options}
    \label{admin:option}
    \fbox{\includegraphics[width=0.3\textwidth]{dev-system/adminmode.jpg}}
\end{figure}

 The administrator side mainly features the management of the tourist spots that will be shown in the application. Figure \ref{admin:option} shows all the privileges of the administrator such as managing user accounts, managing tourist spots, and managing itineraries.  

\begin{figure}[H]
    \centering
    \caption{Analytics}
    \label{admin:dash}
    \fbox{\includegraphics[width=0.3\textwidth]{dev-system/admindashboard.jpg}}
\end{figure}

The administrator can see the summary and reports of the application in the Analytics page, as shown in Figure \ref{admin:dash}. The summary shows the total number of itineraries, total number of tourist spots, archived tourist spots, and the total kilometers planned in the itineraries. Summary of popular places is also shown along with the least popular places. This feature provides the administrator with insights into the usage of the application and helps them make informed decisions about managing the tourist spots and improving the user experience. 

\begin{figure}[H]
    \centering
    \caption{Manage Places}
    \begin{subfigure}[H]{0.3\textwidth}
        \centering
        \fbox{\includegraphics[width=\textwidth]{dev-system/adminplaces.jpg}}
        \caption{Manage Tourist Spot}
        \label{admin:places}
    \end{subfigure}
    \hfil
    \begin{subfigure}[H]{0.3\textwidth}
        \centering
        \fbox{\includegraphics[width=\textwidth]{dev-system/adminadd.jpg}}
        \caption{Add Tourist Spot}
        \label{admin:add}
    \end{subfigure}
\end{figure}

To keep the application scalable and up-to-date, the administrator can manage the tourist spots in the application by adding new tourist spots, updating existing tourist spots, see Figure \ref{admin:places}. When adding a new tourist spot, the administrator must fill in the required information such as the name, category, location, description, and upload photos of the tourist spot, as shown in Figure \ref{admin:add}. The coordinates of the tourist can either be retrieved manually from Google Maps or other mapping services, or by pinning the location of the tourist spot on the map. This feature allows the administrator to easily manage the tourist spots in the application.

\begin{figure}[H]
    \centering
    \caption{Tourist Spot Management}
    \begin{subfigure}[H]{0.3\textwidth}
        \centering
        \fbox{\includegraphics[width=\textwidth]{dev-system/adminupdate.jpg}}
        \caption{Update Tourist Spot}
        \label{admin:update}
    \end{subfigure}
    \hfil
    \begin{subfigure}[H]{0.3\textwidth}
        \centering
        \fbox{\includegraphics[width=\textwidth]{dev-system/adminanal.jpg}}
        \caption{Tourist Spot Analytics}
        \label{admin:anal}
    \end{subfigure}
\end{figure}

Existing tourist spots can also be updated by the administrator, as shown in Figure \ref{admin:update}. The administrator can update the information of the tourist spot such as the name, category, location, description, and photos. Additionally, the analytics of the tourist spot can also be viewed, which shows the total number of visits, total number of reviews, average rating, rating distribution, visit trends, and number of itineraries that include the tourist spot, as shown in Figure \ref{admin:anal}. This feature provides the administrator with insights into the performance of each tourist spot.

\begin{figure}[H]
    \centering
    \caption{Distance Matrix Management}
    \label{admin:distance}
    \fbox{\includegraphics[width=0.3\textwidth]{dev-system/admindistance.jpg}}
\end{figure}

One of the critical components of the Lakad application is the distance matrix, which contains the distances between all pairs of tourist spots. Figure \ref{admin:distance} shows the distance matrix management page, where the distance matrix can be recomputed when new tourist spots are added. This approach lowers the API usage from Open Route Service, ensuring that the application can scale without incurring high costs from API calls. Since, it will calculate the distances between all pairs of tourist spots, expect that the recomputation process may take a while, especially as the number of tourist spots increases.  

\begin{figure}[H]
    \centering
    \caption{Review Reports}
    \begin{subfigure}[H]{0.3\textwidth}
        \centering
        \fbox{\includegraphics[width=\textwidth]{dev-system/adminreport.jpg}}
        \caption{{}}
        \label{admin:report1}
    \end{subfigure}
    \hfil
    \begin{subfigure}[H]{0.3\textwidth}
        \centering
        \fbox{\includegraphics[width=\textwidth]{dev-system/adminreport2.jpg}}
        \caption{{ }}
        \label{admin:report2}
    \end{subfigure}
\end{figure}

The administrator can also view the reported reviews in the application, as shown in Figure \ref{admin:report1} and Figure \ref{admin:report2}. Users can report inappropriate reviews, and the administrator can view these reported reviews and take appropriate actions such as deleting the review, hide, or dismiss the report. This feature helps maintain the quality of reviews in the application and ensures that users can rely on the reviews and ratings provided by other travelers.

\section{System Evaluation}

This section presents the evaluation results of the Lakad application, focusing on the acceptability of the system among users and the overall software quality. The data collected was processed and analyzed in Jupyter Notebook with the use of Python libraries such as Pandas, Matplotlib, and NumPy. The results are presented in terms of per construct median scores, frequency distributions, mean scores, and standard deviations. 

\subsection{Evaluation of System Acceptability}

The acceptability of the Lakad was evaluated  with a Technology Acceptance Model (TAM) survey, which was distributed to 150 respondents. The distribution was proportionately done across different tourist spots in a selected municipality in Bulacan. 

\begin{table}[H]
	\centering
	\singlespacing
	\caption{System Acceptability results}
	\label{tam1}
	\renewcommand{\arraystretch}{1.2}
    \begin{tabularx}{\linewidth}{p{0.3\linewidth} >{\centering\arraybackslash}X >{\centering\arraybackslash}X p{0.23\linewidth}}
		\toprule 
		\textbf{Construct} & \textbf{Mean Score} & \textbf{Standard Deviation} & \textbf{Interpretation}\\
		\midrule
        Perceived usefulness        & 4.71 & 0.34 & Highly acceptable \\  
        Perceived ease of use       & 4.71 & 0.43 & Highly acceptable \\
        Attitude towards using      & 4.72 & 0.40 & Highly acceptable \\
        Behavioral intention to use & 4.56 & 0.47 & Highly acceptable \\  
        \midrule
         \textbf{Grand TAM Score}           & \textbf{4.67} & \textbf{0.42} & \textbf{Highly acceptable} \\
		\bottomrule
	\end{tabularx}
\end{table}

Table \ref{tam1} summarizes the results of the TAM survey, showing the mean scores and standard deviations for each construct. All the constructs achieved a high mean which was interpreted as highly acceptable, indicating that users found the Lakad application to be useful, easy to use, and had a positive attitude towards using it. Perceived usefulness, perceived ease of use, and attitude towards using all had a mean score of 4.71 or higher, suggesting that the Lakad application effectively meets user needs and expectations. The behavioral intention to use scored slightly lower than the other constructs but still achieved a high mean score of 4.56, indicating that users are likely to continue using the application in the future. The standard deviations for all constructs were relatively low (ranging from 0.34 to 0.47), suggesting that there was a consensus among respondents regarding their perceptions of the Lakad application. Overall, the results indicate that the Lakad application is well-received by users and has a high level of acceptability with a grand TAM score of 4.67 which is interpreted as highly acceptable.

\begin{table}[H]
	\centering
	\singlespacing
	\caption{Per item median and frequency scores of TAM survey}
	\label{tam2}
	\renewcommand{\arraystretch}{1.2}
	\begin{tabularx}{\linewidth}{>{\hsize=3.5\hsize}X >{\hsize=.583\hsize}X >{\hsize=.583\hsize}X >{\hsize=.583\hsize}X >{\hsize=.583\hsize}X >{\hsize=.583\hsize}X >{\hsize=.583\hsize}X}
		\toprule 
		\multirow{2}{*}{\textbf{Items}} & \multirow{2}{*}{\textbf{Median}} & \multicolumn{5}{c}{\textbf{Frequency}} \\
		\cmidrule(l){3-7}
		& & \textbf{1} & \textbf{2} & \textbf{3} & \textbf{4} & \textbf{5} \\
		\midrule
		PU1 (Improvement in Performance) & 5 & 0 & 0 & 3 & 52 & 95 \\
		PU2 (Task Efficiency)            & 5 & 0 & 0 & 3 & 30 & 117 \\
		PU3 (Productivity)               & 5 & 0 & 0 & 1 & 54 & 95 \\
		PU4 (Effectiveness)              & 5 & 0 & 0 & 4 & 38 & 108 \\
		PU5 (Usefulness)                 & 5 & 0 & 0 & 1 & 23 & 126 \\
		\midrule
		PEU1 (Ease of Learning)          & 5 & 0 & 0 & 5 & 43 & 102 \\
		PEU2 (Ease of Use)               & 5 & 0 & 0 & 2 & 39 & 109 \\
		PEU3 (Clarity of Interaction)    & 5 & 0 & 0 & 2 & 37 & 111 \\
		PEU4 (Ease of Becoming Skillful) & 5 & 0 & 0 & 10 & 30 & 110 \\
		PEU5 (Ease of Operation)         & 5 & 0 & 0 & 3 & 27 & 120 \\
		\midrule
		ATU1 (Positive Feelings)         & 5 & 0 & 0 & 3 & 29 & 118 \\
		ATU2 (Enjoyment)                 & 5 & 0 & 0 & 4 & 37 & 109 \\
		ATU3 (Satisfaction)              & 5 & 0 & 0 & 3 & 40 & 107 \\
		\midrule
		BI1  (Intention to Use)          & 4 & 0 & 2 & 15 & 67 & 66 \\
		BI2  (Future Use)                & 5 & 0 & 1 & 9 & 45 & 95 \\
		BI3  (Recommendation)            & 5 & 0 & 2 & 1 & 23 & 124 \\
		\bottomrule
	\end{tabularx}
\end{table}

Table \ref{tam2} shows the per item median scores and frequency distributions for each item in the TAM survey. This shows the underlying distribution of responses for each item. The data is heavily skewed towards the higher end of the scale (4 and 5), indicating that most respondents had a positive perception towards Lakad. In the Perceived Usefulness (PU), all items report a median score of 5. The Usefulness item (PU5) has the highest frequency of 5s, with 126 respondents rating it as 5, indicating that users find the system very useful. For the Perceived Ease of Use (PEOU), all items also report a median score of 5. The system is perceived as easy to operate, learn, and master. Ease of Operation (PEU5) has the highest frequency of 5s, with 120 respondents rating it as 5. For the Attitude Towards Using (ATU), all items also report a median score of 5. Users express high satisfaction, enjoyment, and positive feelings towards using the system. Satisfaction (ATU3) has the highest frequency of 5s, with 107 respondents rating it as 5. The Behavioral Intention to Use (BI) items show slightly more variance, this explains the results in the lower mean score for this construct in Table \ref{tam1}. The Intention to Use item (BI1) has a median score of 4, with a significant number of respondents rating it as 4 (67 respondents) and 5 (66 respondents). This suggests that while many users have a strong intention to use the system, there is also a notable portion that is less certain about their intention to use it. The Future Use item (BI2) has a median score of 5, with 95 respondents rating it as 5, indicating that many users are likely to continue using the system in the future.
 
\begin{figure}[H]
    \centering
    \caption{Respondent Demographics}
    \begin{subfigure}[H]{0.3\textwidth}
        \fbox{\includegraphics[width=\textwidth]{evaluation/travfreq.png}}
        \caption{Travelling Frequency}
        \label{fig:travelfrequency}
    \end{subfigure}
    \hfil
    \begin{subfigure}[H]{0.3\textwidth}
        \fbox{\includegraphics[width=\textwidth]{evaluation/mobexp.png}}
        \caption{Experience in Mobile Applications}
        \label{fig:mobexp}
    \end{subfigure}
\end{figure}

Figure \ref{fig:travelfrequency} shows the distribution of respondents based on their travelling frequency with the majority of respondents (54.7\%) travels occasionally, while 28.4\% rarely travels, and only 16.9\% travels frequently. This may also explain the slightly lower mean score for the BI construct in Table \ref{tam1}, as users who rarely or occasionally travel may not have a strong intention to use the application as reflected in BI1. This also implies that occasional travelers may find the application useful for their travel in the future, as reflected in the frequency distribution of BI2. 

Figure \ref{fig:mobexp} shows that 44.2\% of respondents have Intermediate experience with mobile applications, while 29.9\% have Beginner experience, and 25.9\% have Advanced experience. This supports the high mean score for the PEOU construct in Table \ref{tam1} because whatever the experience level of the respondents with mobile applications, they still find the Lakad application easy to use. 

\begin{figure}[H]
    \centering
    \caption{Frequency distribution of TAM survey responses}
    \label{fig:tam-frequency-distribution}
    \begin{subfigure}[H]{0.3\textwidth}
        \centering
        \fbox{\includegraphics[width=\textwidth, trim=0 0 0 1cm, clip]{evaluation/pu.png}}
        \caption{Perceived Usefulness}
        \label{fig:pu}
    \end{subfigure}
    \hfill
    \begin{subfigure}[H]{0.3\textwidth}
        \centering
        \fbox{\includegraphics[width=\textwidth, trim=0 0 0 1cm, clip]{evaluation/peou.png}} 
        \caption{Perceived Ease of Use}
        \label{fig:peou}
    \end{subfigure}
    \hfill
    \begin{subfigure}[H]{0.3\textwidth}
        \centering
        \fbox{\includegraphics[width=\textwidth, trim=0 0 0 1cm, clip]{evaluation/au.png}}
        \caption{Attitude Towards Using}
        \label{fig:atu}
    \end{subfigure}
    \hfill
    \begin{subfigure}[H]{0.3\textwidth}
        \centering
        \fbox{\includegraphics[width=\textwidth, trim=0 0 0 1cm, clip]{evaluation/bi.png}}
        \caption{Behavioral Intention to Use}
        \label{fig:bi}
    \end{subfigure}
\end{figure}

Figure \ref{fig:tam-frequency-distribution} shows the frequency distribution of responses for each construct in the TAM survey. The distribution is heavily skewed towards the higher end of the scale (4 and 5), indicating that most respondents had a positive perception towards Lakad. This further supports the interpretation of the high mean scores for each construct in Table \ref{tam1}, suggesting that users find the Lakad application to be useful, easy to use, and have a positive attitude towards using it. It also explores the Behavioral Intention to Use construct, which shows a slightly more varied distribution compared to the other constructs, but still indicates a generally positive intention to use the application in the future.

\subsection{Software Quality Evaluation}

The software quality of the Lakad application was evaluated using a survey based on the ISO/IEC 25010:2023 software quality model. The survey was done by IT experts, Software developers, and selected faculty members in the College of Information and Communications Technology (CICT) of Bulacan State University. The survey was distributed to 22 IT professionals. 

\begin{table}[H]
	\centering
	\singlespacing
	\caption{System Quality results}
	\label{iso1}
	\renewcommand{\arraystretch}{1.2}
    \begin{tabularx}{\linewidth}{p{0.4\linewidth} >{\centering\arraybackslash}X >{\centering\arraybackslash}X >{\centering\arraybackslash}X}
		\toprule 
		\textbf{Construct} & \textbf{Mean Score} & \textbf{Standard Deviation} & \textbf{Interpretation}\\
		\midrule
        Functional Suitability & 4.86 & 0.24  & Excellent \\
        Performance Efficiency & 4.77 & 0.39  & Excellent \\
        Compatibility          & 4.66 & 0.50  & Excellent \\
        Interaction Capability & 4.76 & 0.26  & Excellent \\
        Reliability            & 4.82 & 0.36  & Excellent \\
        Security               & 4.83 & 0.32  & Excellent \\
        Maintainability        & 4.68 & 0.32  & Excellent \\
        Flexibility            & 4.75 & 0.31  & Excellent \\
        Safety                 & 4.75 & 0.45  & Excellent \\
        \midrule
         \textbf{Grand ISO/IEC 25010:2023 Score}           & \textbf{4.76} & \textbf{0.36} & \textbf{Excellent} \\
		\bottomrule
	\end{tabularx}
\end{table}

Table \ref{iso1} summarizes the results of the software quality evaluation based on the ISO/IEC 25010:2023 model. All constructs achieved a high mean score ranging from 4.66 to 4.86, which were interpreted as excellent. The highest mean score was for Functional Suitability (4.86), indicating that the Lakad application effectively meets the functional requirements and provides the necessary features for users. Security (4.83) and Reliability (4.82) also scored very high, suggesting that the application is secure and reliable for users. Compatibility had the lowest mean score (4.66) among the constructs, this is explained by the exclusivity of the app to Android devices, which limits its compatibility with other platforms. But since the app is written in React Native, it can be easily ported to other platforms in the future. The standard deviations for all constructs were relatively low (ranging from 0.24 to 0.50), indicating that there was a consensus among respondents regarding the software quality of the Lakad application. Overall, the results indicate that the Lakad application has excellent software quality across all evaluated constructs, with a grand ISO/IEC 25010:2023 score of 4.76 which is interpreted as excellent.  


To provide a more detailed analysis of the software quality evaluation, the per item median scores and frequency distributions for each item in the survey are presented in the following tables. Exploring for deeper understanding of the specific aspects of each construct that contributed to the overall high scores.

All the items in the survey reported a median score of 5, the majority of the respondents rated each item as 5. Figure \ref{fig:iso_freq} shows the per item frequency distribution of the ISO/IEC 25010:2023 survey, which illustrates the distribution of responses for each item in the survey. The letter before the item corresponds to the construct, such that (A) is for Functional Suitability, (B) is for Performance Efficiency, (C) is for Compatibility, (D) is for Interaction Capability, (E) is for Reliability, (F) is for Security, (G) is for Maintainability, (H) is for Flexibility, and (I) is for Safety. 

The per item frequency distribution in Figure \ref{fig:iso_freq} reveals the cause of the high mean scores for each construct. It also shows the strengths and weaknesses of Lakad in terms of software quality. Accountability (F), Functional Appropriateness (A), Testability (G), and Faultlessness (E) achieved the highest frequencies of 5 (Excellent), indicating that the app effectively accomplish the task of planning an itinerary while being reliable. On the other hand, Analysability (G), Replaceability (H), Modularity (G), and Interoperability (C) accumulated the lowest frequencies of 5. Although, those items have the lowest frequencies of 5, most of the response were above average (Very Good). Since Lakad operates exclusively to Android devices, this explains the results in the Interoperability item that the app currently does not work to other devices like iOS. The score in Replaceability suggest that while Lakad shows high Functional suitability score, competing against existing apps like Google Maps and TripAdvisor may be difficult. But since Lakad is written in React Native, it can be easily ported to other platforms in the future, which may improve the score in Interoperability and Replaceability. Overall, the per item frequency distribution provides insights into the specific areas of software quality where Lakad excels and where there may be opportunities for improvement.

\begin{figure}[H]
    \centering
    \caption{Per item frequency distribution of ISO/IEC 25010:2023 survey}
    \label{fig:iso_freq}
    \rule{\textwidth}{0.5pt}
    \includegraphics[width=\textwidth, trim=0cm 2cm 0.5cm 0cm, clip]{evaluation/iso_freq.png}
    \rule{\textwidth}{0.5pt}
\end{figure}


\begin{figure}[H]
    \centering
    \caption{Per construct frequency distribution of ISO/IEC 25010:2023}
    \label{fig:iso_construct}
    \begin{subfigure}[H]{0.3\textwidth}
        \centering
        \fbox{\includegraphics[width=\textwidth, trim=0 0 0 1cm, clip]{evaluation/isoa.png}}
        \caption{Functional Suitability}
        \label{fig:isoa}
    \end{subfigure}
    \hfill
    \begin{subfigure}[H]{0.3\textwidth}
        \centering
        \fbox{\includegraphics[width=\textwidth, trim=0 0 0 1cm, clip]{evaluation/isob.png}}
        \caption{Performance Efficiency}
        \label{fig:isob}
    \end{subfigure}
    \hfill
    \begin{subfigure}[H]{0.3\textwidth}
        \centering
        \fbox{\includegraphics[width=\textwidth, trim=0 0 0 1cm, clip]{evaluation/isoc.png}}
        \caption{Compatibility}
        \label{fig:isoc}
    \end{subfigure}
    \hfill
    \begin{subfigure}[H]{0.3\textwidth}
        \centering
        \fbox{\includegraphics[width=\textwidth, trim=0 0 0 1cm, clip]{evaluation/isod.png}}
        \caption{Interaction Capability}
        \label{fig:isod}
    \end{subfigure}
    \hfill
    \begin{subfigure}[H]{0.3\textwidth}
        \centering
        \fbox{\includegraphics[width=\textwidth, trim=0 0 0 1cm, clip]{evaluation/isoe.png}}
        \caption{Reliability}
        \label{fig:isoe}
    \end{subfigure}
    \hfill
    \begin{subfigure}[H]{0.3\textwidth}
        \centering
        \fbox{\includegraphics[width=\textwidth, trim=0 0 0 1cm, clip]{evaluation/isof.png}}
        \caption{Security}
        \label{fig:isof}
    \end{subfigure}
    \hfill
    \begin{subfigure}[H]{0.3\textwidth}
        \centering
        \fbox{\includegraphics[width=\textwidth, trim=0 0 0 1cm, clip]{evaluation/isog.png}}
        \caption{Maintainability}
        \label{fig:isog}
    \end{subfigure}
    \hfill
    \begin{subfigure}[H]{0.3\textwidth}
        \centering
        \fbox{\includegraphics[width=\textwidth, trim=0 0 0 1cm, clip]{evaluation/isoh.png}}
        \caption{Flexibility}
        \label{fig:isoh}
    \end{subfigure}
    \hfill
    \begin{subfigure}[H]{0.3\textwidth}
        \centering
        \fbox{\includegraphics[width=\textwidth, trim=0 0 0 1cm, clip]{evaluation/isoi.png}}
        \caption{Safety}
        \label{fig:isoi}
    \end{subfigure}
\end{figure}

Figure \ref{fig:iso_construct} shows the per construct frequency distribution of the ISO/IEC 25010:2023 survey. The Functional Suitability, Performance Efficiency, and Flexibility constructs shows a frequency distribution of only 4s and 5s, indicating that all respondents rated these constructs as either Very Good or Excellent. Generally, all constructs show a frequency distribution that is heavily skewed towards the higher end of the scale (4 and 5), indicating that most respondents had a positive perception towards the software quality of Lakad. This further supports the interpretation of the mean scores in Table \ref{iso1} that the Lakad application has excellent software quality across all evaluated constructs. This also shows that the evaluation achieved an overall median score of 5.