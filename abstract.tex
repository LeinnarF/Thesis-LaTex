\begin{abstract}
Manual itinerary planning is inefficient because of poor personalization and the lack of integration with route optimization tools. This study develops Lakad, a personalized mobile itinerary generator for promoting tourism in Bulacan. Lakad utilized an Adaptive Genetic Algorithm with Dynamic Mutation and Crossover Probabilities (AGAM) for preference-based itinerary generation and integrates Simulated Annealing (SA) for further route optimization. The quantitative and applied approach was adopted for the research, which included the comparative evaluation of seven different algorithms for the Traveling Salesman Problem, along with the system development, and assessment based on the Technology Acceptance Model (TAM) and ISO/IEC 25010:2023 standard. Simple Additive Weighting (SAW), detetmined that Simulated Annealing (SA) algorithm has the best balance of accuracy and execution speed for real-time mobile application. The grand mean score of 4.67 for TAM evaluation results collected from end-users showed favorable acceptance of the system. The professional evaluation results using ISO/IEC 25010:2023 showed that experts evaluated the system with a mean of 4.76, implying that it is up with software quality standards. The findings reveal that implementing metaheuristic algorithms together with local tourism information enabled the development of an efficient system to create personalized itineries and support sustainable tourism development in provinces like Bulacan.


\noindent\textit{Keywords: } itinerary generation, route optimization, traveling salesman problem, simulated annealing, adaptive genetic algorithm, Bulacan tourism
\end{abstract}