\begin{abstract}
Manual itinerary planning in Bulacan remains inefficient due to uneven attraction popularity, the absence of personalized routing tools, and the lack of systems that integrate user preferences with route optimization. This study developed Lakad, a mobile personalized itinerary generator designed to promote tourism in the Province of Bulacan. The system utilizes the Adaptive Genetic Algorithm with Dynamic Mutation and Crossover Probabilities (AGAM) for personalized itinerary generation and Simulated Annealing (SA) for route optimization. A quantitative and applied research design was employed, combining a comparative analysis of seven metaheuristic algorithms for the Traveling Salesman Problem (TSP), system development, and formal evaluation using the Technology Acceptance Model (TAM) and ISO/IEC 25010:2023 standards. Using the Simple Additive Weighting (SAW) method with equal weights for solution quality and runtime efficiency, SA emerged as the most suitable algorithm, achieving a strong balance between accuracy and speed for real-time mobile use. The developed application features tourist spot exploration, personalized itinerary generation, route optimization, GPS navigation, and itinerary management. TAM evaluation results indicated highly favorable user acceptance, with respondents finding the system both useful and easy to use. Professional evaluation based on ISO/IEC 25010:2023 confirmed that the system achieved outstanding performance across all software quality criteria. The findings demonstrate that combining metaheuristic optimization with localized tourism data provides a practical, user-centered solution for sustainable tourism development in Bulacan and serves as a promotional tool for both popular and lesser-known destinations.


\noindent\textit{Keywords: } itinerary generation, route optimization, traveling salesman problem, simulated annealing, adaptive genetic algorithm, Bulacan tourism
\end{abstract}