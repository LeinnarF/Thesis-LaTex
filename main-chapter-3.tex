\chapter{Research Methodology}

In this chapter, the researchers will discuss the research design, model adoption, process of developing the system, system evaluation, population and sample, data collection, data processing, and ethical considerations.

\section{Research Design}

This study will employ a quantitative applied research design which uses quantitative methods to provide a structured framework for the development and evaluation of the proposed mobile personalized itinerary generator for tourists and alike. 

\subsection{Applied Research}
The applied research aspect of the study will focus on developing and implementing a system for generating personalized itineraries and optimizing existing itineraries for the interests of the user. The primary objective of this study is to create a mobile application that optimizes itineraries using the ABO algorithm and generates personalized itineraries with travel budget using AGAM algorithm. Through an applied research approach, the features to be implemented in the system will be important for the needs of its intended users which are the tourists and their itinerary needs.

\subsection{Quantitative Research}
The quantitative research aspect of this study will seek to measure and analyze the quality, standards, and acceptability of the system using quantitative methods by collecting user and professional feedback. The researchers will assess the quality and standards using the ISO/IEC 25010 standard by seeking professionals to assess the system. While the acceptability of the system for tourists will be analyzed using the TAM. The survey will gauge the user’s satisfaction with the system using a 10-point rating scale, providing insights to the system’s ease of use and usefulness. Incorporating quantitative approach for the study ensures an empirical and evidence-based support for the quality and acceptability of the proposed system.

\section{Model Adoption}
The models adopted for the system are the traveling salesman problem for the optimization of the itineraries and orienteering problem for the generation of personalized itineraries with allocation of travel distance budget. 

\subsection{African Buffalo Optimization}
The African Buffalo Optimization (ABO) algorithm is a meta-heuristic method inspired by the cooperative behavior of African buffaloes in their search for food. It models how buffalo herds communicate, explore, and exploit their environment to find optimal grazing areas. In optimization problems such as the Traveling Salesman Problem (TSP), ABO can be used to determine the shortest possible route that visits all cities exactly once while minimizing travel cost or distance.

According to \textcite{Odili2022Method}, the algorithm begins by initializing a population of buffaloes, where each buffalo is randomly placed within the solution space. Next, the fitness value of each buffalo is calculated based on the total travel cost of its route. The buffaloes then update their positions using the ABO update rules, which are derived from the buffaloes’ communication signals. These signals guide the herd collectively toward better solutions by balancing exploration (searching new areas) and exploitation (refining known good areas). During this process, buffaloes construct new tours by adding unvisited cities based on their attractiveness and the herd’s accumulated experience. This cycle continues until the stopping criteria, such as a maximum number of iterations or convergence, are met. Finally, the best solution found by the herd is output as the optimized route.

Formally, the algorithm can be expressed using two main update equations. The first is for buffalo’s memory or exploitation behavior:
\[
m _{k}' = m _{k} + lp1 ( bg - w _{k} ) + lp2 ( bp _{k} - w _{k} )
\]

Where $ m _{k} $ represents the buffalo’s previous movement (exploitation term), $ bg $ is the global best fitness representing the herd’s best solution, and $bp _{k}$ is the individual buffalo’s best-known position. The parameters $lp1$ and $lp2$ are learning coefficients that control the influence of herd communication and personal experience, respectively. This equation enables each buffalo to adjust its movement based on both the global herd direction and its own previous success, allowing it to return to a more rewarding position if it strays into less optimal regions. 

The second equation controls the buffalo’s exploration within the search space:
\[
w _{k} ' = \frac{ ( w _{k} + m _{k} ) }{\lambda}
\]

Where $w_k$ represents the buffalo’s exploratory move, $ m _{k} $ represents the exploitation or memory component, and $\lambda \in (0,1]$ is the exploration driver that regulates how strongly the buffalo explores new areas. After updating both memory and position, the algorithm checks if the global best $bg$ has improved. If so, it proceeds to the next iteration; otherwise, it continues refining current solutions. The process repeats until the stopping condition is met (e.g., maximum iteration count or convergence). Once completed, the herd’s best-known position $bg$ is returned as the optimal solution, representing the most efficient tour or path discovered by the algorithm. 


\subsection{Adaptive Genetic Algorithm}

The researchers also chose to adopt a modified variant of the genetic algorithm developed by \textcite{Yochum2020} called Adaptive Genetic Algorithm with Dynamic Mutation and Crossover Probabilities or AGAM for the optimization of recommended itineraries defined in terms of TSP. AGAM addresses itinerary planning as a Multi-Objective Optimization Problem, an optimization technique rooting from orienteering problems. As discussed in their work, \textcite{Yochum2020} developed AGAM to generate optimal and personalized travel itineraries by balancing several objectives such as POI rating, and distance of POIs between the next one. Rooting from GA, AGAM simulates the process of natural evolution and utilizes genetic operators such as selection, crossover, and mutation to iteratively improve the quality of itineraries which, unlike traditional GA that use fixed mutation rates, also employs dynamic adjustments in parameters during the evolutionary process in a way that can maintain population diversity and therefore generate efficient and adaptive solutions.

AGAM works by taking specific factors that can affect the results of a solution. In this paper, the goal is to generate an optimal solution of a recommended itinerary, which lets the model take into account factors similar to the ones proposed by \textcite{Yochum2020}. These factors include mandatory POIs, all POIs, POI distance, and POI rating. The factor referring to mandatory POIs also means that the sequence of each itinerary requires a starting and ending destination. In addition to these factors, AGAM also requires a certain probability that determines the chances of mutation and crossover to happen.

To better understand the factors taken into account in this study, the researchers define them as follows;
The popularity of a POI $q_i$ is given by the visit count of the POI over the number of visits of the most visited POI.
\[P(q_i) = \frac{1}{q_i}\] 

The rating of a POI $q_i$ is given by taking the average rating of all visitors for $q_i$.
\[R(q_i) = \frac{1}{x} \sum_{j=1}^{x} r_{q_i}(u_j)\]

AGAM considers the population set $P = {p_1, …, p_n}$ where each individual $p_i = {q_1, q_2, q_3, …, q_n}$ is a sequence where each $q_i$ is a POI. This means that AGAM treats each POI sequence $p_i$ as an individual gene, hence $P$ as a sequence of itineraries. \textcite{Yochum2020} noted that measuring different factors involves variables with inconsistent units and scales, and that each gene sequence may vary in length depending on the number of POIs selected. They addressed this by applying a normalization method that converts all variable values into a uniform range between 0 and 1, so that the algorithm can fairly compare and combine the factors within a fitness function. This fitness function normalizes the evaluation of the quality of each $p_i$ by integrating the input factors defined earlier. In this way, every itinerary $p_i$ in $P$ obtains different weights given by the function:
\[
f(p_i) = w_{1}Tm(p_i) + w_{2}Tn(p_i) + w_{3}Tr(p_i) + w_{4}Tp(p_i)
\]
Where, $f(p_i)$ is the fitness function score of each itinerary $p_i$. $w_{j}$ is an adjustable weight of each factor which depends on the user’s preference.

$Tm(p_i)$ is the ratio of mandatory POIs, it is calculated as the number of mandatory POIs that in the currently itinerary $p_i$ divided by the total number of mandatory POIs that the user want to visit


$Tn(p_i)$ is the total number of POIs included in the itinerary relative to the maximum number of POIs that can be visited within the user’s $M AXT$. It is obtained by dividing the number of POIs in the itinerary $p_i$ by the maximum possible number of POIs.

$Tr(p_i)$ is the total rating of $p_i$ given by taking the sum of all $q_i$ rating $R(q_i)$ in $p_i$ and dividing it by the best possible rating of all POIs in consideration combined $MAXR = Number of POIs * Maximum Rating$.
\[ \dfrac{\sum_{j=1}^{n} R(q_j) \in p_i}{MAXR}, \text{ such that } n = last POI\]

$Tp(p_i)$ is the total popularity of $p_i$ given by the sum of all $q_i$ popularity $P(q_i)$ in $p_i$ and dividing it by the total popularity count of the all POI $MAXP$.
\[\dfrac{\sum_{n}^{i=1}P(q_i) \in p_i}{MAXP}, \text{ such that } n = last POI\]

Moreover, AGAM’s most important part is the dynamic crossover and mutation probabilities that it uses. This property of AGAM serves as a guide and is helpful in finding the best solution  since it prevents the system from repeatedly computing in a local optima, a region of solutions that  can be optimal but lacks diversity, a kind of deadlock where the system ignores every other possible optimal solution outside the current locality. \textcite{Yochum2020} defined it probabilities as  
\[
PC =
\begin{cases}
pc_1 - \dfrac{(pc_1 - pc_2)(f' - f_{avg})}{f_{max} - f_{avg}}, & f' \ge f_{avg} \\
pc_1, & f' < f_{avg}
\end{cases}
\]

In this case, $PC$ is the probability of crossover. $pc_1$ and $pc_2$ are the input bounds that will limit the crossover. These values are often defined where $pc_1 \geq pc_2$, $f’$ then corresponds to the larger valued fitness score of two selected parents, $f_{max}$ is the largest valued fitness score among the set of populations $P$, and $f_{avg}$ refers to the average fitness score of the population $P$. The next one is for the probability of mutation.
\[
PM =
\begin{cases}
pm_1 - \dfrac{(pm_1 - pm_2)(f' - f_{avg})}{f_{max} - f_{avg}}, & f \ge f_{avg} \\
pm_1, & f < f_{avg}
\end{cases}
\]
where, $pm_1$ and $pm_2$ are also predefined bounds to limit mutations, and prevent over saturation, and $f$ is the fitness score of a parent that has to mutate. Furthermore, the parameters $pc_1$, $pc_2$, $pm_1$, and $pm_2$ are all predefined input bounds selected through parameter tuning, which controls the range of possible crossover and mutation probabilities during the evolution process.

\section{Process of Developing the System}

The development of the system will follow a structured approach that is divided into distinct phases to ensure a systematic and efficient creation of the mobile application. These phases include analysis, planning, development, testing, and review, each contributing to the overall quality and functionality of the final product.

\[
\text{ SDLC Picture here }
\]

\subsection{Analysis}
The analysis phase of the development involves outlining the system’s scope, objectives, and requirements. This is done by reviewing the existing literature in itinerary recommendation and local tourism in the province of Bulacan. This phase establishes the key features of the proposed system, LAKAD, that will integrate five major features designed to provide optimized, personalized, and efficient itinerary planning for tourists exploring Bulacan. Each feature combines techniques, verified local data, and user-centric design principles.

\begin{enumerate}
    \item \textbf{Personalized Itinerary Generation}
    The main feature of the proposed application will be a personalized itinerary generator focusing on the diverse cultural, historical and heritage tourist destinations located across the province of Bulacan. The system will be developed to adapt to user-specific factors that can affect the personalization aspect of an itinerary. This feature aims to cater the personal preferences that a user has while still promoting the exploration of the rich tourism culture of the region.

    \item \textbf{Itinerary Optimization}
    The optimization feature of the system will work in arrangement with the personalized itinerary generator by employing the ABO algorithm. The optimization will further help the maximization of travel distance and constraints to visit every possible destination within the generated itinerary.
    
    \item \textbf{Tourist Spot Searching}
    A tourist spot search feature will allow users to explore Bulacan’s tourist attractions specifically through an interface where they can search through a catalog of  tourist destinations that will be verified by the Provincial History, Arts, Culture, and Tourism Office (PHACTO) of Bulacan. Furthermore, the system will display important and relevant information such as location and short description about the tourist spot.
    
    \item \textbf{Itinerary Management}
    An Itinerary Management feature of the system will allow users to manage and organize itineraries. The users may add, edit, or remove destinations from their itinerary. It will also allow the tracking of visited and unvisited locations so that users are informed of the status of their progress in the itinerary. Users will also be able to save and reload itineraries for future use.
    
    \item \textbf{Itinerary Navigation}
    This feature will assist the user during travel by providing real-time navigation support. Users will be able to track their current location and visualize routes similar to Google Maps. It will displace direction, travel duration, and distances between tourist spots and support live GPS tracking to help users follow their itinerary accurately.

\end{enumerate}

To ensure consistent implementation of the mobile application, the following system requirements for mobile devices will be recommended as shown in Table \ref{sysreq}

\begin{table}[H]
    \centering
    \caption{Recommended System Requirements for Mobile Devices}
    \label{sysreq}
    \renewcommand{\arraystretch}{1.0}
    \begin{tabularx}{\linewidth}{XX}
        \toprule 
        \textbf{Category} & \textbf{Specification} \\
        \midrule
        Operating System & Android 9.0 (Pie) or higher. \\
        Processor (CPU) & Octa-core 1.8 GHz or equivalent. \\
        Memory (RAM) & Minimum 3 GB \\
        Storage & At least 200 MB free space \\
        Display & 720 x 1280p or higher. \\
        Internet Connectivity & Required for fetching map data and API requests. \\
        GPS & For location tracking and navigation. \\
        \bottomrule
    \end{tabularx}
\end{table}

These requirements ensure that the application runs efficiently on most modern mid-range Android smartphones. Devices with higher specifications are expected to experience smoother performance, especially when generating optimized itineraries or handling large datasets of tourist locations.

\subsection{Planning}
The Gantt chart provides a visual timeline of the schedule of the research activity that will be conducted in the process of developing LAKAD. It outlines the sequence of tasks relating to Thesis I and Thesis II, including the flow of each phase of the research from documentation and planning until system development, evaluation, and final defense.

\[
\text{Gantt Chart Here}
\]

The \textbf{Thesis I} output focuses on the research documentation and planning for development of LAKAD. This process begins with the Review of Related Studies (Chapter 2), in which local and foreign studies are gathered, evaluated, and organized to develop the theoretical and conceptual framework. By gathering information in Chapter 2, determining the best algorithm for the system, and the appropriate approach to execute the idea, the researchers were able to establish a solid foundation that would be the basis of the succeeding phases. After completing Chapter 1: Problem and its Background and Chapter 2, the researchers finalize the concepts, models, and methods that will be utilized in Chapter 3: Methodology preparation. Chapter 3 defines the research design, the selected development model, and the desired process on creating the system, as well as the foundation on system evaluation and data gathering that will be conducted on Thesis II.

For \textbf{Thesis II}, the researcher will proceed to the implementation and completion phase. The group will develop and implement the proposed system based on the design created in Thesis I. 

\textbf{System Development and Implementation}. This phase will focus on building and integrating the components of the system based on the planned design. In the development process, the construction of the front-end, database integration, and feature implementation following the chosen development model. Testing and debugging processes will be carried out on a regular basis to ensure proper functionality and performance.

\textbf{System Testing and Evaluation}. This phase is aimed at testing the reliability, usability, and general quality of the system. Various testing procedures will be conducted to ensure that every function works as planned and meets user needs. Feedback will be collected from users and a system analyst to identify areas that need improvement.

\textbf{Data Analysis and Interpretation}. This stage will include data processing and data analysis that are conducted from the system evaluation. Responses collected will be organized, encoded, and statistically analyzed to determine user perceptions as well as the general performance of the system.

\subsection{Development}

This development phase pertains to the actual coding and implementation of the system’s features  inside an android application. The researchers’ will utilize the AGILE methodology, particularly the kanban method for the software development life cycle. This will include weekly meetings and development checks by the researchers to ensure the system is properly developed to its intended uses.

The application will be written in Kotlin programming language which allows fast android development.  The system will also utilize the Jetpack Compose framework for implementing graphical user interfaces and Android Studio IDE for developing the application as shown in table \ref{listtools}. 

\begin{table}[H]
    \centering
    \caption{List of Tools for Development}
    \label{listtools}
    \renewcommand{\arraystretch}{1.0}
    \begin{tabularx}{\linewidth}{XX}
        \toprule 
        \textbf{Name} & \textbf{Description} \\
        \midrule
        Android Studio  & A robust IDE for developing and debugging android applications. \\
        Kotlin          & A powerful programming language for developing android applications as replacement for java. \\
        Jetpack Compose & A library for developing reactive graphical user interfaces in android.\\
        Git             & A software for versioning and managing software projects. \\
        GitHub          & Host for git repositories that allows online collaboration. \\
        \bottomrule
    \end{tabularx}
\end{table}

\subsection{Testing}
For the testing phase, LAKAD will undergo rigorous evaluation to ensure the overall functionality and quality. The process will include system integration and user acceptance testing. System testing will focus on validating each module of the system by using test cases that will evaluate the logic of the system which will identify and resolve potential issues affecting itinerary generation and recommendation features. Integration testing will be conducted to ensure that all components of the system, such as itinerary generation and optimization, management, and user interface, operates cohesively to provide a seamless user experience.


\subsection{Review}
The review phase will focus on evaluating the overall functionality and usability of the system after development is completed. The researchers will gather views and opinions of users, including a system analyst to determine whether the system runs efficiently and serves its desired purposes. Guided by their observation, necessary adjustments and enhancements will be designed to refine the system.

\section{System Evaluation}

\subsection{Evaluation Instrument}
The proposed system will be evaluated using the Technology Acceptance Model (TAM), as it provides a theoretical framework for understanding the process by which users accept and adopt new technologies. TAM is widely applied in system development studies for calculating the level of user acceptance based on their perceptions and intentions for using a system. To evaluate the effectiveness and acceptance of the proposed system, the following key constructs will be evaluated:
\begin{enumerate}
    \item \textbf{Perceived Usefulness}. This construct measures how much users would believe that system usage would improve their task performance. It looks at whether system functionalities like itinerary recommendation, generation, and optimal route provide concrete value and efficiency to the user.
    \item \textbf{Perceived Ease of Use}. This measures the degree of which users find the system easy to learn and operate. It singles out features including interface design, feature clarity, and the convenience of performing tasks.
    \item \textbf{Attitude Toward Use}. This assesses the sentiment, satisfaction, and positive impression towards using the system. It measures how much the user enjoys interaction with the system and whether they find it useful and interesting.
    \item \textbf{Intent to Engage in Usage}. This criterion measures the tendency and likelihood of the user to continue using the system in the future. It measures the intentions of the people to recommend or repeatedly use the system as their preferred tool.
\end{enumerate}

The proposed system will also be evaluated by IT professionals using the ISO / IEC 25010 questionnaire which evaluates the system’s quality and user satisfaction. The following criteria will be used in the questionnaire and are based on the ISO / IEC 25010 standards:

\begin{enumerate}
    \item \textbf{Functional Suitability}. This will focus on how well the proposed system meets the user’s needs. The questions will focus on the system's main features such as itinerary optimization, personalized itinerary generation, and itinerary management.
    \item \textbf{Performance Efficiency}. This will assess the system’s ability to provide fast and efficient processing and its capability to handle large amounts of tasks.  The questions will assess the system’s response time when it comes to itinerary optimization and generation as well as managing the itineraries of the user.
    \item \textbf{Compatibility}. This will evaluate the system’s compatibility with other external systems that the users might use. The questionnaire will verify how well the proposed system integrates and works alongside other mapping applications and systems.
    \item \textbf{Interaction Capability}. This will focus on the system’s usability, accessibility, and user satisfaction. Through this, the system will be assessed on how easy the system is to navigate, how intuitive and simple the user interface is for itinerary planning, and how effectively it guides the users. 
    \item \textbf{Reliability}. This will assess the system’s dependability and consistency in performing its intended functions. The criterion will assess how well the system can operate without failure, maintain stability, and recover from unexpected errors or interruptions. 
    \item \textbf{Security}.  This criterion will ensure that all collected data remains confidential, secure, and protected. Access to the data will only be available to the researchers and authorized users. No information will be shared with third parties without proper consent.
    \item \textbf{Maintainability}. This will assess the ability of the system to be updated, debugged, and improved. This criterion will focus on the ability of the system to adapt to changes and its efficiency during system updates. It will ensure that new features or fixes can be integrated without disrupting existing functionalities of the system.
    \item \textbf{Flexibility}. This will evaluate the system’s ability to adapt to changing user needs, technological changes, and requirement changes. The criterion will assess how easily the system can support new itinerary features, and integrate additional modules or tools. 
\end{enumerate}

\subsection{Population and Sample}

\subsection{Data Collection}

\subsection{Data Processing}

\section{Ethical Consideration}
Ethical considerations will play a crucial role in the conduct of this study, especially since it will involve the participation of real people, and government institutions and offices. The researchers will strictly follow ethical guidelines in compliance with the Data Privacy Act of 2012 (Republic Act No. 10173), to uphold integrity of the research and to ensure the privacy and security of all collected data. Obtaining informed consent, implementing data encryption, and enforcing strict access controls will be observed to safeguard the rights, and confidentiality of all individuals and institutions involved in the evaluation and development of the system.
\subsection{Informed Consent}
Participants of the study will be required to give an informed consent before responding to any personally identifying data or opinions that will be collected for the study, usability testing, and system evaluation. Consent materials will inform respondents about the study purpose, what data will be collected, how the data will be used, risks and benefits, and contact details for questions. The project will follow recognized Philippine guidance on ethical review and informed consent processes such as the Data Privacy Act of 2012 (Republic Act No. 10173).
In addition to individual respondents, the study will also involve data collection from government offices, particularly from Bulacan Provincial History, Arts, Culture, and Tourism Office (PHACTO) and other related agencies that maintain official records on local attractions, and tourism statistics. The researchers will formally address the Provincial Governor and the Head of the Provincial Tourism Department to request permission to access relevant datasets or documents. These data will include, but will not be limited to, lists of Points of Interest (POIs), and visitor statistics of tourism establishments.
The request letters will clearly outline the purpose of the data use, the scope of the requested information, and the intended outcomes of the study. Only publicly shareable or officially permitted data will be used, and the researchers will comply with all terms set by the concerned government offices. The use of such data will be limited strictly to academic and system development purposes.

\subsection{Confidentiality and Privacy}
The researchers will ensure that all personal or sensitive information collected during the study will remain confidential. Any identifiable data from participants or government records will be anonymized before analysis or publication. Access to raw data will be restricted to authorized research team members only, and all information will be securely stored in password-protected or encrypted digital repositories.

\subsection{Data Protection}
All collected data will be processed and stored in compliance with the Data Privacy Act of 2012 (Republic Act No. 10173) and its Implementing Rules and Regulations (IRR). The researchers will adopt the principles of transparency, legitimate purpose, and proportionality in handling personal and institutional data. Appropriate technical and organizational security measures, such as encrypted file storage, limited data access, and secure data transmission will be implemented to protect information from unauthorized access, alteration, or loss.

\subsection{Ethical Data Collection}
The data collection process will be guided by integrity and accountability. Only information relevant to the objectives of the system will be gathered. The researchers will not collect sensitive personal data unless necessary and permitted by the respondent.
For survey and usability testing activities, participation will be entirely voluntary, and responses will be treated as confidential. For institutional data, such as those obtained from PHACTO or other local government offices, the researchers will follow formal request procedures and uphold any data-sharing agreements or confidentiality clauses that will be imposed.


\subsection{Transparency and Integrity}
The researchers will maintain transparency throughout the conduct of the study. All research methods, procedures, data sources, and analytical processes will be accurately documented and reported. The team will strictly avoid misrepresentation of data, selective reporting of results, or exaggeration of the system’s capabilities.
Acknowledgment will be given to all sources of information, including government offices and online databases. Conflicts of interest, if any, will be disclosed openly. The researchers will ensure that all representations made to data providers, research participants, and readers will be honest and verifiable.
The study will uphold the principles of academic integrity, guided by the ethical standards outlined by the Data Privacy Act of 2012 (Republic Act No. 10173), Republic Act No. 8293 (Intellectual Property Code of the Philippines), and institutional research ethics guidelines of Bulacan State University (BulSU).
