\chapter{Research Methodology}
\thispagestyle{empty}

In this chapter the research design, model adoption, process of developing the system, system evaluation, population and sample, data collection, data processing, and ethical considerations are discussed. This ensures that the research methodology is well-defined and structured, providing a clear roadmap for the execution of the study. The research design outlines the approach and framework for conducting the research, while the model adoption section delves into the specific mathematical models and algorithms that are utilized in the development of LAKAD. The process of developing the system details the steps taken to create and implement the application, while the system evaluation focuses on assessing its performance and effectiveness. The population and sample section defines the target audience for the study, and the data collection and processing sections outlines how the data was gathered and analyzed. Finally, ethical considerations ensure that the research is conducted in a responsible and ethical manner, respecting the rights and privacy of participants.

\section{Research Design}

This study primarily focuses on the development and implementation of a mobile application that generates personalized itineraries for tourists. The research design is structured to ensure that the system is developed based on empirical evidence and is evaluated for its quality, standards, and acceptability among users and professionals in the tourism industry. The quantitative aspect of the research design allows for the collection and analysis of numerical data to assess the effectiveness of the system, while the applied research approach ensures that the development of the system is grounded in real-world applications and addresses the specific needs of its intended users.

\subsection{Quantitative Research}
The quantitative research, specifically, descriptive research design is one of the designs in this study. This aspect of the study measure, analyze, and describe the quality, standards, and acceptability of the system using quantitative methods by collecting user and professional feedback. The researchers assess the quality and standards using the ISO/IEC 25010:2023 standard by seeking professionals to assess the system. While the acceptability of the system for tourists was analyzed using the Technology Acceptable Model (TAM). The survey measures the user’s satisfaction with the system using a 10-point rating scale for ISO/IEC 25010:2023 and 5-point rating scale for TAM, providing insights to the system’s ease of use and usefulness. Incorporating quantitative approach for the study ensures an empirical and evidence-based support for the quality and acceptability of the developed system.

\subsection{Applied Research}
The applied research aspect of the study focuses on developing and implementing a system for generating personalized itineraries and optimizing existing itineraries for the interests of the user. The primary objective of this study is to create a mobile application that optimizes itineraries using the best Metaheuristic algorithm and generates personalized itineraries with travel budget using Adaptive Genetic Algorithm with Dynamic Mutation and Crossover Probabilities (AGAM) framework. The selection of the best metaheuristic algorithm is based on the performance of the algorithms in terms of solution quality and runtime efficiency. The development of the system involves the use of mathematical models such as the Traveling Salesman Problem (TSP) for route optimization and the Orienteering Problem (OP) for itinerary generation. The implementation of these models in the mobile application ensures that the system is grounded in real-world applications and addresses the specific needs of its intended users, providing a practical solution for personalized itinerary generation and optimization.

\section{Model Adoption}
This section discusses the optimization problems such as route optimization and itinerary generation and the steps in modeling each mathematically. Specifically, the mathematical models used in the study are the traveling salesman problem (TSP) and the orienteering problem (OP).

\subsection{Modeling Route Optimization with TSP}
Route optimization refers to the process of finding the optimal order in visiting a set of places. This is also another definition of the TSP, where the salesman must minimize the travel cost between traveling a set of cities. In the study, the POI represents the nodes and the edges connecting each node as the distance between the shortest route, a sample graph is shown in Figure \ref{tspgraph}. 

\begin{figure}[H]
	\centering
	\caption{Sample Graph}
	\label{tspgraph}
	\includegraphics[width=0.8\textwidth]{TSP.jpeg}
\end{figure}

The objective is to minimize the total travel distance by finding the best order to travel each POI from an itinerary. Using integer programming and given a graph $G$ = $(V, E)$ where \( V \) is the set of vertices (POI) and \( E \) is the set of edges (road distance), the objective function is as shown in Equation \ref{eq:tsp}:
\begin{equation}
	\label{eq:tsp}
	\text{Minimize } \sum_{i = 1}^{n} \sum_{j = 1, j \neq i} c_{ij} x_{ij}
\end{equation}
where:
\begin{itemize}
	\item $n$ is the total number of vertices (POI).
	\item $c_{ij}$ refers to the cost of traveling from vertex $i$ to vertex $j$, which in this case is the distance between the two POI.
	\item $x_{ij}$ refers to a binary value that indicates whether an edge directly connecting vertex $i$ to vertex $j$ exists in the solution (1 if it exists, 0 otherwise).
\end{itemize}
The function is also constrained to the following:
\begin{enumerate}
	\item Every city is entered and exited exactly once.
	\item The solution does not contain any sub-tours.
\end{enumerate}

\subsection{Candidate Algorithms for TSP}
The study considers several metaheuristic algorithms for route optimization for LAKAD, each with its unique approach and characteristics. These algorithms include:

\begin{enumerate}
	
\item \textbf{Genetic Algorithm (GA).} Inspired by genetics and evolution, the Genetic Algorithm is an optimization technique that works by analyzing potential solutions and generating new ones through the processes of selection, crossover, and mutation. This iterative process continues until a predefined termination requirement is met. GA is widely used across various domains because of its ability to navigate complex search spaces and identify optimal or near-optimal solutions. Algorithm \ref{GA} illustrates the step-by-step GA approach for solving the TSP \parencite{Hossain2024}.

\item \textbf{Ant Colony Optimization (ACO).} The Ant Colony Optimization (ACO) is an optimization algorithm that mimics the foraging behavior of ants. It utilizes artificial pheromone trails to guide the search for optimal solutions. Ants in the algorithm employ probabilistic pheromones and heuristic information to address problems. The quality of the solutions determines how the pheromone trails are updated. This mechanism encourages exploration and prevents premature convergence. ACO has proven effective for various optimization difficulties. Algorithm \ref{ACO} outlines the stepwise ACO approach for solving TSP \parencite{Hossain2024}.

\item \textbf{Simulated Annealing (SA).} The Simulated Annealing (SA) algorithm is a metaheuristic technique influenced by the annealing process in metallurgy. It begins with an initial solution and incorporates a mechanism that allows for the occasional acceptance of a "worse" solution based on and more thoroughly explore the solution space. The probability of accepting a worse solution is gradually reduced over time, simulating the cooling process. SA is applicable to a wide range of optimization problems. Algorithm \ref{SA} illustrates the stepwise SA algorithm for solving TSP \parencite{Hossain2024}.

\item \textbf{Particle Swarm Optimization (PSO).} The Particle Swarm Optimization (PSO) algorithm is inspired by the social behavior of bird flocks and fish schools. In PSO, the population is made up of numerous particles, each representing a potential solution within the search space. Every particle is characterized by a velocity and a fitness value. Conceptually, PSO is very straightforward, as it relies solely on elementary mathematical operations and does not require any derived information about the optimization function \parencite{Wadi2025}. Algorithm \ref{PSO} provides a step-by-step description of the PSO algorithm as applied to solving TSP.

\item \textbf{Elephant Herding Optimization (EHO).} The Elephant Herding Optimization (EHO) algorithm is a swarm-based metaheuristic search approach for addressing optimization problems that mimics the real herding behavior of elephants, which involves organizing elephant swarms into smaller clans, each led by a matriarch and consisting of multiple female elephants and their calves, while adult male calves are expelled from the clan \parencite{Wadi2025}. Algorithm \ref{EHO} provides a step-by-step description of the EHO algorithm as applied to solving TSP.

\item \textbf{Grey Wolf Optimizer (GWO).} The Grey Wolf Optimization (GWO) algorithm is a metaheuristic technique inspired by grey wolves, simulating their leadership and cooperative hunting behaviors to locate optimal solutions. The hierarchical structure of a wolf pack (Alpha, Beta, Delta, and Omega wolves) is shown in the algorithm's design. Wolves' positions are updated based on the Alpha's location and the calculated hunting distance. GWO leverages a balance of exploration and exploitation to discover the best possible solutions. This approach has been successfully applied to various optimization challenges. Algorithm \ref{GWO} provides a step-by-step description of the GWO algorithm as applied to solving the Traveling Salesperson Problem (TSP) \parencite{Hossain2024}. 

\item \textbf{Hovering Scouts and Foraging Flocks Pied Kingfisher Optimizer (HSFFPKO).} HSFFPKO enhances the traditional Pied Kingfisher Optimizer by introducing multiscale search mechanisms to improve optimization performance. By adding Hovering Scouts for fine-grained local adjustments and Foraging Flocks for parallel group-based exploitation, the algorithm maintains better population diversity than its predecessor. These additions allow HSFFPKO to mitigate local optima entrapment and solve complex, high-dimensional problems more efficiently \parencite{DelaCruz2025HSFFPKO}. Algorithm \ref{HSFFPKO} provides a step-by-step description of the HSFFPKO algorithm.   
\end{enumerate}

\subsection{TSP Algorithm Selection}

To determine the best algorithm to implement for LAKAD, a comparative analysis was conducted on several metaheuristic algorithms used for solving TSP. The analysis focuses on solution quality (accuracy) and runtime speed of each algorithm. The algorithms were tested on selected instances from the TSPLIB library, a standard benchmark for TSP and related problems. Table \ref{tsplib_instace} presents selected instances for benchmarking. 

\begin{table}[H]
	\centering
	\singlespacing
	\caption{TSPLIB Instances for Benchmarking}
	\label{tsplib_instace} \renewcommand{\arraystretch}{1.2}
	\begin{tabularx}{\linewidth}{XXX}
		\toprule
		\textbf{TSPLIB instance} & \textbf{Description} & \textbf{Optimal Solution} \\
		\midrule
		burma14 & Small instance with 14 cities & 3323\\ 
		bays29 & Medium instance with 29 cities & 2020\\ 
		eil51 & Large instance with 51 cities & 426\\ 
		berlin52 & Large instance with 52 cities & 7542\\
		\bottomrule
	\end{tabularx}
\end{table}

\textbf{Experimental Setup.} The experiments were implemented in Python because of its library support for TSPLIB, and for simplicity of the language. To ensure statistical reliability and mitigate the stochastic nature of heuristic algorithms, each algorithm was executed for 10 independent iterations on each TSPLIB instance. The results were averaged to get the baseline performance profile. 
Two primary metrics were defined to evaluate the algorithms:

\begin{enumerate}
	\item \textbf{Runtime Speed (\( T_{avg} \)).} It measures the average time it takes to find the solution across the 10 iterations, measured in seconds. This metric is crucial for assessing the efficiency of the algorithm, and is calculated using the formula shown in equation \ref{eq:runtime}: 
	\begin{equation}
		\label{eq:runtime}
		T_{avg} = \frac{1}{10} \sum_{i=1}^{10} T_i
	\end{equation}
	where $T _{i}$ is the runtime for the $i$-th iteration.

	\item \textbf{Solution Quality (\( Q_{avg} \)).} Since the best solution for each TSPLIB instance is known, the accuracy is measured by the average relative error of the solution found by the algorithm compared to the optimal solution. The calculation is shown in equation \ref{eq:quality}:
	\begin{equation}
		\label{eq:quality}
		Q_{avg} = \frac{1}{10} \sum_{i=1}^{10} \frac{C_i - C_{opt}}{C_{opt}} \times 100
	\end{equation}
	where:
	\begin{itemize}
		\item $C_i$ is the cost of the solution found in the $i$-th iteration.
		\item $C_{opt}$ is the cost of the known optimal solution for the instance.
	\end{itemize}
\end{enumerate}

\textbf{Algorithm Ranking.} To select the final algorithm, the trade-off between speed and accuracy was resolved using Simple Additive Weighting (SAW).SAW was chosen because of its simplicity and interpretability, allowing for a straightforward comparison of the algorithms based on the defined metrics. Since both $T _{avg}$ and $Q_{avg}$ are “cost” criteria, i.e., lower values represent better performance, it is normalized relative to the maximum (worst) observed value to scale all metrics between 0 and 1. The normalization formula used was shown in equation \ref{eq:norm}:
\begin{equation}
	\label{eq:norm}
	r _{ij} = \frac{x_{ij}}{\max(x_{ij})}
\end{equation}
where: 
\begin{itemize}
	\item $r _{ij}$ is the normalized value of the $j$-th criteria for the $i$-th algorithm.
	\item $x_{ij}$ is the original value of the $j$-th criteria for the $i$-th algorithm.
	\item $\max(x_{ij})$ is the maximum value observed for the $j$-th criteria across all algorithms.
\end{itemize}

The final score $S _{i}$ for each algorithm $i$ was calculated in equation \ref{eq:saw}: 
\begin{equation}
	\label{eq:saw}
	S _{i} = \sum_{j=1}^{n} w_j \cdot r _{ij}
\end{equation}
where:
\begin{itemize}
	\item $S _{i}$ is the final score for the $i$-th algorithm.
	\item $w_j$ is the weight assigned to the $j$-th criteria, such that $\sum_{j=1}^{n} w_j = 1$.
	\item $r _{ij}$ is the normalized value of the $j$-th criteria for the $i$-th algorithm, which was defined in equation \ref{eq:norm}.
\end{itemize}

For this study, since the weights can be subjectively assigned, equal weights were assigned to both criteria, i.e., $w _{1} = w _{2} = 0.5$. This gives equal importance of runtime speed and solution quality, the final score is shown in equation \ref{eq:finalscore}:
\begin{equation}
	\label{eq:finalscore}
	S _{i} = 0.5 \cdot r _{i1} + 0.5 \cdot r _{i2}
\end{equation}

The algorithm with the lowest score $S _{i}$ is considered the best choice for implementation for LAKAD. The results of the algorithm selection process are discussed in Chapter 4. 

\subsection{Modeling Itinerary Generation as Orienteering Problem}

Another core optimization problem in the study is itinerary generation, which can be modeled as an orienteering problem (OP). In the context of the study, an itinerary refers to a plan of the user to go to two or more places. An itinerary can be modeled mathematically using graph theory similar to TSP. This would involve setting the vertices of a graph as the places and the edges as the road distance between the connected edges. A big difference of OP from TSP is the fact that it takes the whole graph or all the possible POI and outputs a subgraph instead, where TSP would use the entire graph given. An example is shown in Figure \ref{opgraph} where 5 POI are given as input and only three were used to generate an itinerary.

\begin{figure}[H]
	\centering
	\caption{Subgraph result from OP}
	\label{opgraph}
	\includegraphics[width=0.8\textwidth]{subgraph.png}
\end{figure}

The objective of the OP is to maximize the available profit, in the case of the study: travel distance, to generate an itinerary for the user alongside other factors. Mathematically, the OP can be formulated as follows in Equation \ref{eq:op}.

\begin{equation}
	\label{eq:op}
	\text{Maximize } Z = \sum_{i = 2}^{n - 1} \sum_{j = 2}^{n}S _{i} x _{ij}  
\end{equation}
where:
\begin{itemize}
	\item $n$ is the total number of vertices (POI).
	\item $S _{i}$ is the score of vertex $i$, which can be a combination of factors such as travel distance, user preferability score, and popularity rating.
	\item $x _{ij}$ is a binary variable that indicates whether the edge from vertex $i$ to vertex $j$ is included in the itinerary (1 if included, 0 otherwise).
	\item $Z$ is the total score of the itinerary, which the algorithm aims to maximize.
\end{itemize}
The function is also constrained to the following:
\begin{enumerate}
	\item Each vertex must only be visited once.
	\item The total score must not exceed the given budget.
	\item The generated path must be connected.
\end{enumerate}

\subsection{Adaptive Genetic Algorithm with Dynamic Mutation and Crossover Probabilities}

The study also chose to adopt a modified variant of the genetic algorithm developed by \textcite{Yochum2020} called Adaptive Genetic Algorithm with Dynamic Mutation and Crossover Probabilities (AGAM) for the optimization of recommended itineraries defined in terms of TSP. AGAM addresses itinerary planning as a Multi-Objective Optimization Problem, an optimization technique rooting from orienteering problems. As discussed in their work, \textcite{Yochum2020} developed AGAM to generate optimal and personalized travel itineraries by balancing several objectives such as point of interest (POI) rating, and distance of POIs between the next one. Rooting from Genetic Algorithm (GA), AGAM simulates the process of natural evolution and utilizes genetic operators such as selection, crossover, and mutation to iteratively improve the quality of itineraries which, unlike traditional GA that use fixed mutation rates, also employs dynamic adjustments in parameters during the evolutionary process in a way that can maintain population diversity and therefore generate efficient and adaptive solutions.

In this implementation, the population set is defined as $P = \left\{ p _{1},\dots, p _{n} \right\}$, where each individual $p _{i}$ represents a potential itinerary. An itinerary is encoded as a sequence of POIs, $p _{i}= \left\{ q _{1}, q _{2}, \dots, q _{m} \right\}$, where each gene $q _{i}$ corresponds to a specific POI visit. The length of the sequence depends on the number of POIs selected, and it is constrained by a user-defined limit $MAXN$.

\textbf{Fitness Function.} The quality of each itinerary is evaluated by the fitness function given in Equation \ref{agamFitness}. Since factors such as popularity, rating, and interest are on different scales, a normalization process is applied to each factor so that its values is between 0 and 1. 

\begin{equation}
	\label{agamFitness}
	f(p_i) = w_{1}Ti(p_i) + w_{2}Tn(p_i) + w_{3}Tr(p_i) 
\end{equation}
Where $w _{j}$ for $j = 1, 2, 3$, represents the weight assigned to each factor, and it is defined as follows:
\begin{itemize}
    
    \item $Ti(p_i)$ is the sum of all interest values of each POI in the itinerary $p_i$. Each POI contains an interest value which quantifies how interested the user is in that specific POI category.
    
    \item $Tn (p _{i})$ is the total number of POIs in $p _{i}$, which is calculated in Equation \ref{Tn}.
    \begin{equation}
        \label{Tn}
        Tn ( p _{i}) = \frac{N _{op} ( p _{i} )}{NOP}, \quad \text{where } N _{op} ( p _{i} ) \le MAXN 
    \end{equation}
    where $N _{op} ( p _{i} )$ is the number of POIs in itinerary $p _{i}$, and $NOP$ is the total number of available POIs.
    
    \item $Tr( p _{i} )$ is the total rating of $p_i$ given by Equation \ref{Tr},
    \begin{equation}
        \label{Tr}
        Tr ( p _{i} ) = \frac{\sum_{q_i \in p_i} R(q_i)}{MAXR}
    \end{equation} 
    where $R(q_i)$ is the rating of a specific POI $q_i$ in itinerary $p_i$ obtained from Google Maps, and $MAXR$ is the best possible rating of all POIs defined as $MAXR = \text{Number of POIs} * \text{Maximum Rating}$.
\end{itemize}

\textbf{Adaptive Genetic Operators.} AGAM incorporates dynamic adjustments in mutation and crossover probabilities to balance exploration (searching new solutions) and exploitation (refining existing solutions). These are defined as follows:
\begin{itemize}
    \item \textbf{Adaptive Crossover $ ( P _{c} )$.} To preserve high-quality solutions, the crossover probability decreases as the fitness of the parents exceeds the population average $ ( f _{avg} )$. This is given by Equation \ref{crossover},
    \begin{equation}
        \label{crossover}
        PC = \begin{cases}
            pc_1 - \dfrac{(pc_1 - pc_2)(f' - f_{avg})}{f_{max} - f_{avg}}, & f' \ge f_{avg} \\
            pc_1, & f' < f_{avg}
        \end{cases}
    \end{equation}
    Here, $pc_1$ and $pc_2$ are the upper and lower bounds for crossover probability,$f_{max}$ is the largest valued fitness score among the set of populations $P$, $f'$ is the larger fitness score of the two parents being crossed, and $f_{avg}$ refers to the average fitness score of the population $P$. 

    \item \textbf{Adaptive Mutation $ ( P _{m} )$.} Similarly, mutation rates are adjusted to prevent destroying optimal solutions. This is defined in Equation \ref{mutation},
    \begin{equation}
        \label{mutation}
        PM = \begin{cases}
            pm_1 - \dfrac{(pm_1 - pm_2)(f - f_{avg})}{f_{max} - f_{avg}}, & f \ge f_{avg} \\
            pm_1, & f < f_{avg}
        \end{cases}
    \end{equation}
    where $pm_1$ and $pm_2$  are the bounds for mutation probability, and $f$ is the fitness score of the individual being mutated.
\end{itemize}

For the parameters of AGAM, the following values were used: $pc_1 = 0.9$, $pc_2 = 0.6$, $pm_1 = 0.1$, and $pm_2 = 0.01$. These values were chosen based on \textcite{Yochum2020} recommendation, which are commonly used in dynamic crossover and mutation probabilities.

The procedure of AGAM is shown in Algorithm \ref{alg:agam}

\begin{algorithm}[H]
	\singlespacing
	\footnotesize
	\caption{Adaptive Genetic Algorithm for Itinerary Optimization (AGAM)}
	\label{alg:agam}
	\begin{algorithmic}[1]

		\Require $MAXD$ (maximum travel distance), $MAXN$ (maximum number of POIs), $N$ (population size)
		\Require $pc_1, pc_2$ (crossover control parameters), $pm_1, pm_2$ (mutation control parameters)
		\Require $w_1, w_2, w_3, w_4$ (fitness weights), $\delta$ (number of generations)
		\Ensure Optimal itinerary $p_{opt}$

		\State Initialize population $P$ with $N$ randomly generated itineraries
		\For{each $p_i \in P$}
		\State Ensure $TotalDistanceCost(p_i) \le MAXD$
		\EndFor

		\For{iteration = 1 to $\delta$}

		\State Evaluate fitness $F(p_i)$ for each itinerary using:
		\[
			F = w_1 f_{\text{interest}} + w_2 f_{\text{totalPOI}} + w_3 f_{\text{rating}} + w_4 f_{\text{popularity}}
		\]

		\State Compute $f_{\max} = \max(F(p_i))$, $f_{\text{avg}} = average(F(p_i))$

		\State Initialize new population $P_i = \emptyset$

		\While{$|P_i| < N$}
		\State $(parent_1, parent_2) = TournamentSelection(P, k=2)$
		\State $PC = calculateCrossoverProbability(F(parent_1), F(parent_2), f_{\text{avg}}, f_{\max})$

		\If{$random(0,1) < PC$}
		\State $(child_1, child_2) = OnePointCrossover(parent_1, parent_2)$

		\If{$TotalDistanceCost(child_1) > MAXD$ \textbf{and} $NUMPOI(child_1) > MAXN$}
		\State Regenerate($child_1$)
		\EndIf
		\If{$TotalDistanceCost(child_2) > MAXD$ \textbf{and} $NUMPOI(child_2) > MAXN$}
		\State Regenerate($child_2$)
		\EndIf

		\If{$TotalDistanceCost(child_1) < MAXD$ \textbf{and} $NUMPOI(child_1) < MAXN$}
		\State InsertUnvisitedPOIs($child_1$)
		\EndIf
		\If{$TotalDistanceCost(child_2) < MAXD$ \textbf{and} $NUMPOI(child_2) < MAXN$}
		\State InsertUnvisitedPOIs($child_2$)
		\EndIf

		\State Add $child_1$ and $child_2$ to $P_i$

		\Else
		\State Add $parent_1$ and $parent_2$ to $P_i$
		\EndIf
		\EndWhile

		\For{each itinerary $p_i \in P_i$}
		\State $PM = calculateMutationProbability(F(p_i), f_{\text{avg}}, f_{\max})$

		\If{$random(0,1) < PM$}
		\State $Mutated\_p_i = ApplyMutation(p_i)$

		\If{$TotalDistanceCost(Mutated\_p_i) > MAXD$ \textbf{or} $NUMPOI(Mutated\_p_i) > MAXN$}
		\State Regenerate($Mutated\_p_i$)
		\EndIf

		\If{$TotalDistanceCost(Mutated\_p_i) < MAXD$ \textbf{and} $NUMPOI(Mutated\_p_i) < MAXN$}
		\State InsertUnvisitedPOIs($Mutated\_p_i$)
		\EndIf

		\State $p_i \gets Mutated\_p_i$
		\EndIf
		\EndFor

		\State Recalculate fitness of all itineraries in $P_i$
		\State $P \gets P_i$

		\EndFor

		\State \Return $p_{opt}$ = itinerary with the highest fitness in $P$

	\end{algorithmic}
\end{algorithm}


\section{Process of Developing the System}

The development of the LAKAD mobile application uses the AGILE methodology, with a specific focus on the Kanban method. This approach integrates all key software development phases. These phase include analysis, planning, design, development, testing, and review each of the phases contributes to the overall quality and functionality of the final product.

Kanban is a highly effective visual process management system designed to optimize the flow of work. It operates on the principle of Just In Time (JIT) delivery, as highlighted by \textcite{Alaidaros2021}, ensuring the tasks are completed and delivered precisely when needed to maximize efficiency. 

This methodology offers significant advantages for the LAKAD development team. It provides a clear, visual representation of the entire workflow, allowing all team members to easily track progress from start to finish. This transparency is crucial for quickly identifying potential bottlenecks, gaps in the process, or any mishaps that might arise. By making the workflow visible, the team can proactively address issues, rather than reacting to them after they have escalated. Kanban, being a type of Agile-based methodology, is an approach that promotes flexibility which can help in an environment of software development where problems and unforeseen challenges are inevitable. Kanban's adaptability allows the team to adjust priorities and reallocate resources as needed, ensuring that progress remains steady even when obstacles are encountered. This flexibility is key to maintaining momentum and delivering a high-quality software output \parencite{Risener2022}. The continuous flow and iterative nature of Kanban supports the team in delivering a robust and reliable system that meets user needs effectively. Figure \ref{kanban} shows a graphical representation of a Kanban workflow model.

\begin{figure}[H]
	\centering
	\caption{Kanban Software Development Life Cycle}
	\label{kanban}
	\framebox{    \includegraphics[width=0.8\textwidth]{kanban.png}}
\end{figure}

\subsection{Analysis}
The analysis phase of the development involves outlining the system’s scope, objectives, and requirements. This was done by reviewing the existing literature in itinerary recommendation and local tourism in the province of Bulacan. This phase establishes the key features of LAKAD, that integrates five major features designed to provide optimized, personalized, and efficient itinerary planning for tourists exploring Bulacan. Each feature combines techniques, verified local data, and user-centric design principles.

\begin{enumerate}
	\item \textbf{Personalized Itinerary Generation.}
		The main feature of LAKAD is the personalized itinerary generator focusing on the diverse cultural, historical and heritage tourist destinations located across the province of Bulacan. The system developed to adapt to user-specific factors that can affect the personalization aspect of an itinerary. This feature aims to cater the personal preferences that a user has while still promoting the exploration of the rich tourism culture of the region.

	\item \textbf{Itinerary Optimization.}
		The optimization feature of the system works in line with the personalized itinerary generator by employing the chosen best algorithm for route optimization. The optimization further helps the minimization of travel distance and constraints to visit every possible destination within the generated itinerary.

	\item \textbf{Tourist Spot Searching.}
		A tourist spot search feature allows the users to explore Bulacan’s tourist attractions specifically through an interface where they can search through a catalog of  tourist destinations that was verified by the Provincial History, Arts, Culture, and Tourism Office (PHACTO) of Bulacan. Furthermore, the system displays important and relevant information such as location and short description about the tourist spot.

	\item \textbf{Itinerary Management.}
		An Itinerary Management feature of the application allows users to manage and organize their itineraries. The users may add, edit, or remove destinations from their itinerary. It also allows the tracking of visited and unvisited locations so that users are informed of the status of their progress in the itinerary. Users are also able to save and reload itineraries for future use.

	\item \textbf{Itinerary Navigation.}
		This feature assist the user during travel by providing real-time navigation support. Users are able to track their current location and visualize routes similar to Google Maps. It displays direction, travel duration, and distances between tourist spots and supports live GPS tracking to help users follow their itinerary accurately.

\end{enumerate}
Furthermore, the application includes a user profile feature that allows users to create and manage their profiles, which can store their preferences, past itineraries, and other relevant information. This feature enhances the personalization aspect of the system by allowing it to learn from user behavior and preferences over time.

\subsection{Planning}
The Gantt chart provides a visual timeline of the schedule of the research activity that was conducted in the process of developing LAKAD. It outlines the sequence of tasks relating to Thesis I and Thesis II, including the flow of each phase of the research from documentation and planning until system development, evaluation, and final defense.

\begin{figure}[H]
	\centering
	\caption{Gantt Chart}
	\includegraphics[angle=0, width=1\linewidth]{GanttChart.png}
\end{figure}

The \textbf{Thesis I} output focuses on the research documentation and planning for development of LAKAD. This process begins with the Review of Related Studies (Chapter 2), in which local and foreign studies are gathered, evaluated, and organized to develop the theoretical and conceptual framework. By gathering information in Chapter 2, determining the best algorithm for the system, and the appropriate approach to execute the idea, the researchers were able to establish a solid foundation that would be the basis of the succeeding phases. After completing Chapter 1: Problem and its Background and Chapter 2, the researchers finalize the concepts, models, and methods that was utilized in Chapter 3: Methodology preparation. Chapter 3 defines the research design, the selected development model, and the desired process on creating the system, as well as the foundation on system evaluation and data gathering that was conducted on Thesis II.

For \textbf{Thesis II}, the researcher proceeds to the implementation and completion phase. The group developed and implemented the proposed system based on the design created in Thesis I. 

\textbf{System Development and Implementation}. This phase focuses on building and integrating the components of the system based on the planned design. In the development process, the construction of the front-end, database integration, and feature implementation following the chosen development model. Testing and debugging processes was carried out on a regular basis to ensure proper functionality and performance.

\textbf{System Testing and Evaluation}. This phase is for testing the reliability, usability, and general quality of the system. Various testing procedures was conducted to ensure that every function works as planned and meets user needs. Feedback was collected from users and a system analyst to identify areas that need improvement.

\textbf{Data Analysis and Interpretation}. This stage includes data processing and data analysis that are conducted from the system evaluation. Responses collected was organized, encoded, and statistically analyzed to determine user perceptions as well as the general performance of the system.

\subsection{Design}
The design phase focuses on creating the overall architecture and structure of the system. This includes defining the system components, their interactions, and the data flow within the application. The system architecture was designed to ensure scalability, maintainability, and efficiency. The researchers created diagrams and models to visualize the system’s structure and functionality.

\begin{figure}[H]
	\centering
	\caption{System Architecture}
	\label{SysArch}
	\includegraphics[width=\textwidth]{SysArch.png}
\end{figure}

Figure \ref{SysArch} illustrates the system architecture of LAKAD: A Personalized Mobile Itinerary Creator and how the different components of the application interact to provide personalized and optimized travel itineraries for users. At the core of the system is the mobile frontend, which serves as the main interface where users can search for tourist destinations, generate itineraries, and manage their travel plans. When a user interacts with the application, the frontend communicates with the Mapbox API to send requests for map data, location coordinates, and optimized routes. The API processes these requests and returns the corresponding responses, which are then displayed on the user interface in the form of maps and travel routes. In addition, the application stores user-generated itineraries and relevant information in a database, allowing users to access and manage their itineraries. 

\begin{figure}[H]
	\centering
	\caption{Entity Relationship Diagram}
	\label{erd}
	\includegraphics[width=0.8\textwidth]{erd.png}
\end{figure}

The Entity Relationship Diagram (ERD) of the proposed system, as shown in Figure \ref{erd}, consists of seven entities; User, Itinerary, Itinerary\_POI, Verified\_POI, Custom\_POI, Category, Preferences, each connected in a circular way with the exception of the Custom\_POI entity connected only from outside. A User entity can generate many itineraries of the entity Itinerary. An Itinerary entity can contain as many Itinerary\_POI, an entity that acts as a buffer entity for a POI before being added inside an Itinerary, this lets the system properly assign a POI inside an itinerary. Itinerary\_POI entities reference either a Verified\_POI or Custom\_POI, which differs as Verified\_POI entities are locations that are specifically confirmed by PHACTO to be a tourist destination in Bulacan, whereas a Custom\_POI is not necessarily one. By consequence, each Verified\_POI has at least one dedicated Category they belong to. A User entity may also have their own Preferences which are then classified by Category.

\begin{figure}
	\centering
	\caption{Data Flow Diagram}
	\label{dfd}
	\includegraphics[width=0.8\textwidth]{datFlow.png}
\end{figure}

Only one user exists for the whole system which is the tourist as illustrated in Figure \ref{dfd}. There are two main flows for the user actions: preferences flow and itinerary flow. The preferences flow is straightforward, the user submits the preferences per category and then is saved in the app to be used later. The itinerary flow starts from the user generating an itinerary through the AGAM process in which is saved to the database and is sent to the itinerary management process for the visualization through graphical user interface.

From itinerary management, there are two other flows: optimization and navigation flow. The optimization flow would use the best TSP algorithm selected from the analysis to optimize the itinerary visitation where the changes would be saved and flow back to the management. Navigation flow would start the navigation process, showing the tourist the directions and steps to go to the POI in the itinerary.


\subsection{Development}

This development phase pertains to the actual coding and implementation of the system’s features inside an android application. The researchers’ will utilize the AGILE methodology, particularly the Kanban method for the software development life cycle. This includes weekly meetings and development checks by the researchers to ensure the system is properly developed to its intended uses. LAKAD was developed using the React Native main development framework, which uses fast and cost-efficient cross-platform mobile development, allowing a single codebase to function on both iOS and Android devices. The system utilized the JavaScript programming language for implementing the application’s user interface and core functionalities due to its flexibility and compatibility with React Native framework. The main development environment for writing, editing, and debugging the program was Visual Studio Code, which provided lightweight performance and had a large selection of extensions appropriate for React Native development. Throughout the development process, Git and GitHub was used to ensure structured collaboration and version control. The list of tools used in the development is presented in Table \ref{listtools}.

Additionally, React Native extends the "Learn once, write anywhere" paradigm to mobile development, leveraging recent advancements like its new architecture and the JavaScript Interface (JSI). The framework's structure, represented by Fabric, aims to boost compatibility with modern React features such as concurrent rendering and synchronous data fetching. While JSI’s main element allows for direct calls between JavaScript and native code. This change features a smoother interaction with the host platform and significantly reduces performance bottlenecks. JSI also allows Fabric to create UI components on the native side more efficiently. The combined adoption of JSI and Fabric offers a significant enhancement in React Native's performance, making it a good choice for developers who desire the agility of mobile development and native platform functionalities \parencite{Zhou2024}.

This framework allows the researchers to develop the mobile application, LAKAD, as React Native’s cross-platform capabilities also allow for android development.

\begin{table}[H]
	\centering
	\singlespacing
	\caption{List of Tools for Development}
	\label{listtools}
	\renewcommand{\arraystretch}{1.2}
	\begin{tabularx}{\linewidth}{XX}
		\toprule 
		\textbf{Name} & \textbf{Description} \\
		\midrule
		Visual Studio Code  & A lightweight and powerful source code developed by Microsoft. It supports multiple programming languages and offers built-in debugging, version control integration, and extensions suitable for React Native development. \\
		React Native        & An open-source framework developed by Meta for building cross-platform mobile applications using JavaScript and React. \\
		JavaScript     & An Interpreted, high-level programming language for building user interfaces, interaction and application logic.\\
		Git                 & A software for versioning and managing software projects. \\
		GitHub              & Host for git repositories that allows online collaboration. \\
		\bottomrule
	\end{tabularx}
\end{table}


\subsection{Testing}
For the testing phase, LAKAD undergo rigorous evaluation to ensure the overall functionality and quality. The process includes system integration and user acceptance testing. System testing focuses on validating each module of the system by using test cases that evaluates the logic of the system which identifies and resolve issues affecting itinerary generation and recommendation features. Integration testing was conducted to ensure that all components of the system, such as itinerary generation and optimization, management, and user interface, operates cohesively to provide a seamless user experience.

\subsection{Review}
The review phase focuses on evaluating the overall functionality and usability of the system after development is completed. The researchers gather views and opinions of users, including a system analyst to determine whether the system runs efficiently and serves its desired purposes. Guided by their observation, necessary adjustments and enhancements done to refine the system.



\section{System Evaluation}
This section outlines the methods and criteria used to evaluate LAKAD, the developed mobile itinerary generator system. It details the instruments and standards for assessment, describes the target population and sampling approach, and explains the procedures for data collection and analysis. The goal is to ensure that the system meets user needs and quality benchmarks through structured evaluation using both user and expert feedback.

\subsection{Evaluation Instrument}
The evaluation instrument for the developed system uses Technology Acceptance Model (TAM) for the end-users and ISO / IEC 25010:2023 for IT professionals. This combination of evaluation methods allows for a comprehensive assessment of the system from both user experience and technical quality perspectives. 

\textbf{Technology Acceptance Model (TAM).} The developed system was evaluated using the Technology Acceptance Model (TAM), as it provides a theoretical framework for understanding the process by which users accept and adopt new technologies. TAM is widely applied in system development studies for calculating the level of user acceptance based on their perceptions and intentions for using a system. To evaluate the effectiveness and acceptance of the proposed system, the following key constructs was evaluated:
\begin{enumerate}
	\item \textbf{Perceived Usefulness}. This construct measures how much users would believe that system usage would improve their task performance. It looks at whether system functionalities like itinerary recommendation, generation, and optimal route provide concrete value and efficiency to the user.
	\item \textbf{Perceived Ease Of Use}. This measures the degree of which users find the system easy to learn and operate. It singles out features including interface design, feature clarity, and the convenience of performing tasks.
	\item \textbf{Attitude Toward Using}. This assesses the sentiment, satisfaction, and positive impression towards using the system. It measures how much the user enjoys interaction with the system and whether they find it useful and interesting.
	\item \textbf{Behavioral Intention To Use}. This criterion measures the tendency and likelihood of the user to continue using the system in the future. It measures the intentions of the people to recommend or repeatedly use the system as their preferred tool.
\end{enumerate}

The TAM questionnaire in this study was an adapted version of a TAM questionnaire, which was modified to fit the context of LAKAD. Additionally, a reliability test was conducted on the TAM questionnaire to ensure that the instrument is consistent and reliable for measuring user acceptance. The reliability test was performed using Cronbach's alpha, which assesses the internal consistency of the questionnaire items. A Cronbach's alpha value of 0.7 or higher is generally considered acceptable, indicating that the items in the questionnaire are measuring the same underlying construct effectively \parencite{Cronbach2025}.

\textbf{ISO/IEC 25010:2023.} The developed system was also evaluated by IT professionals using the ISO/IEC 25010:2023 questionnaire which evaluates the overall quality of LAKAD. The following criteria was used in the questionnaire and are based on the ISO/IEC 25010:2023 standards:

\begin{enumerate}
	\item \textbf{Functional Suitability}. This focused on how well the proposed system meets the user's needs. The questions were about the system's main features such as itinerary optimization, personalized itinerary generation, and itinerary management.
	\item \textbf{Performance Efficiency}. This assessed the system's ability to provide fast and efficient processing and its capability to handle large amounts of tasks. The questions assessed the system's response time when it comes to itinerary optimization and generation as well as managing the itineraries of the user.
	\item \textbf{Compatibility}. This evaluated the system's compatibility with other external systems that the users might use. The questionnaire verified how well the developed system integrates and works alongside other mapping applications and systems.
	\item \textbf{Interaction Capability}. This focused on the system's usability, accessibility, and user satisfaction. Through this, the system was assessed on how easy the system is to navigate, how intuitive and simple the user interface is for itinerary planning, and how effectively it guides the users.
	\item \textbf{Reliability}. This assessed the system's dependability and consistency in performing its intended functions. The criterion assessed how well the system can operate without failure, maintain stability, and recover from unexpected errors or interruptions.
	\item \textbf{Security}.  This criterion ensured that all collected data remains confidential, secure, and protected. Access to the data will only be available to the researchers and authorized users. No information will be shared with third parties without proper consent.
	\item \textbf{Maintainability}. This assessed the ability of the system to be updated, debugged, and improved. This criterion focused on the ability of the system to adapt to changes and its efficiency during system updates. It ensured that new features or fixes can be integrated without disrupting existing functionalities of the system.
	\item \textbf{Flexibility}. This evaluated the system's ability to adapt to changing user needs, technological changes, and requirement changes. The criterion assessed how easily the system can support new itinerary features, and integrate additional modules or tools.
	\item \textbf{Safety}. This focused on ensuring that the system does not cause any harm to its users. The criterion assessed how well the system protects users from potential risks, errors, or negative consequences that may arise from using the system.
\end{enumerate}

\subsection{Population and Sample}

The respondents of this study consist of the end-users or the tourists in Bulacan and IT professionals for evaluating LAKAD. The tourists are the primary users  of the application, while the IT professionals are the secondary users who will evaluate the technical aspects of the system. 

\textbf{End-Users.} The End-users or the tourists are the primary users of the application. The acceptance of the application, measured by TAM, were evaluated by the end-users. 

A multi-stage non-probability sampling technique was used to select the respondents for the end-users. First, municipalities in Bulacan were grouped based on their 2025 tourist arrivals (low, medium, high), see Table \ref{tourist_arrivals}. Selected municipalities were chosen from different tourist arrival groups to ensure a diverse representation of tourists. This approach helps to reduce sampling bias and ensures that the sample reflects the diversity of tourists visiting different areas of Bulacan.

\begin{table}[H]
	\centering
	\singlespacing
	\caption{Tourist Arrivals in Bulacan (2025)}
	\label{tourist_arrivals}
	\renewcommand{\arraystretch}{1.2}
	\begin{tabularx}{\linewidth}{XXX}
		\toprule 
		\textbf{Municipality} & \textbf{No. of Arrivals} & \textbf{Classification} \\
		\midrule
		Hagonoy 		& 1,105 	& Low \\
		Obando  		& 5,190 	& Low \\
		Guiguinto 		& 7,907 	& Low \\
		Plaridel 		& 25,323 	& Low \\ 
		San Ildefonso 	& 40,715 	& Low \\
		Baliwag 		& 42,775 	& Low \\
		Paombong 		& 43,510 	& Low \\
		Bulakan 		& 49,078 	& Low \\
		San Rafael 		& 59,044 	& Low \\
		Balagtas 		& 85,236 	& Low \\
		San Miguel 		& 104,990 	& Medium \\
		Santa Maria 	& 110,279 	& Medium \\
		Calumpit 		& 114,000 	& Medium \\
		Pulilan 		& 130,222 	& Medium \\
		Pandi 			& 200,667 	& Medium \\
		Angat 			& 233,255 	& Medium \\
		Meycauayan 		& 278,211 	& Medium \\
		Bustos 			& 295,682 	& Medium \\
		Drt 			& 750,233 	& High \\
		Malolos 		& 816,299 	& High \\
		Norzagaray 		& 1,048,633 & High \\
		Marilao 		& 1,116,989 & High \\
		Csjdm 			& 1,128,736 & High \\
		Bocaue 			& 1,335,539 & High \\
		\midrule
		\textbf{Total} & 8,023,618 &\\
		\bottomrule
	\end{tabularx}
\end{table}

Within the selected municipalities, respondents were allocated proportionately based on the tourist arrivals of each municipality, as provided by the data from the Provincial History, Arts, Culture, and Tourism Office (PHACTO) of Bulacan. This proportional allocation ensures that the sample size from each municipality is representative of the actual distribution of tourists across the region. 

A total of 150 end-users was considered to be appropriate for the study. This sample size is sufficient to provide meaningful insights into user acceptance while being manageable for data collection and analysis. This is supported by previous tourism technology research by \textcite{Raynera2025}, which utilized the TAM with 80 respondents in similar regional tourism contexts, indicating that a sample size of 150 is adequate for achieving reliable results in this study. Additionally, methodological recommendations suggest that survey sample sizes should be justified based on research objectives, analytical approach, and practical constraints, rather than relying solely on numerical rules \parencite{Memon2020}. 

The 150 end-users was selected using convenience sampling because volatility in tourist arrivals and the need for timely data collection. Convenience sampling allows for the selection of respondents who are readily available and willing to participate, which is practical in a tourism context where visitor numbers can fluctuate. This approach enables the researchers to gather data efficiently while still capturing a diverse range of tourist experiences and perceptions across different municipalities in Bulacan. The allocation of the sample size for end-users is shown in Table \ref{endusers}.

\begin{table}[H]
	\centering
	\singlespacing
	\caption{Allocation of Respondents for End-users}
	\label{endusers}
	\renewcommand{\arraystretch}{1.2}
	\begin{tabularx}{\linewidth}{XXX}
		\toprule 
		\textbf{Municipality} & \textbf{Proportion (\%)} & \textbf{No. of Respondents} \\
		\midrule
		Baliwag		& 1.08 		& 2 	\\
		Bocaue 		& 33.86 	& 51 	\\
		Bustos 		& 7.50 		& 11 	\\
		Calumpit 	& 2.89 		& 4 	\\ 
		Guiguinto 	& 0.20 		& 1 	\\
		Malolos 	& 20.70 	& 31 	\\
		Marilao 	& 28.32 	& 42 	\\
		Plaridel 	& 0.64 		& 1 	\\
		Pulilan 	& 3.30 		& 5 	\\
		San Rafael 	& 1.50 		& 1 	\\
		\midrule
		\textbf{Total} & 100.00 & 150 \\
		\bottomrule
	\end{tabularx}
\end{table}

	
\textbf{IT Professionals.} IT professionals evaluates the quality of LAKAD using the ISO/IEC 25010:2023. A total of 20 IT experts were purposively selected based on their qualification and relevant experiences. This includes but not limited to computer programmers, UI/UX, and selected faculty members from The College of Information and Communications Technology (CICT) at Bulacan State University (BulSU).


To summarize, the total sample size for the study will consist of 170 respondents, with 150 end-users and 20 IT professionals as shown in Table \ref{pop_sample}.

\begin{table}[H]
	\centering
	\singlespacing
	\caption{Summary of Population and Sample}
	\label{pop_sample}
	\renewcommand{\arraystretch}{1.2}
	\begin{tabularx}{\linewidth}{XXX}
		\toprule 
		\textbf{Respondent Group} & \textbf{Population} & \textbf{Sample Size} \\
		\midrule
		End-users (Tourists) & Tourists visiting Bulacan tourist spots & 150 \\
		IT Professionals & Software engineers, QA, UX professionals & 20 \\
		\midrule
		\textbf{Total} &  & 170 \\
		\bottomrule
	\end{tabularx}
\end{table}

% -----------------------------------------------------------------------

\subsection{Data Collection}

The data collection for this study will involve two primary instruments. First is the Technology Acceptance Model (TAM) questionnaire for end-users. The other instrument is the ISO/IEC 25010:2023 evaluation for IT professionals.  

The TAM will employ a 5-point Likert scale (1=Strongly Disagree, 5=Strongly Agree) to evaluate users’ responses across constructs such as Perceived Usefulness, Perceived Ease of Use, Attitude Towards Technology Use, and Behavioral Intention (to Use), with 3–5 items for each construct. The use of 5-point scale is justified by its balance between simplicity and reliability in measuring users’ perceptions, it also reduces cognitive load on the side of end-users.  

For IT professionals, the ISO/IEC 25010:2023 evaluation will use a 10-point Likert scale (1=Strongly Disagree, 10=Strongly Agree) to achieve greater precision and granularity in professional assessments. This broader scale will allow technical evaluators to make finer distinctions for the following software attributes: Functional Suitability, Performance Efficiency, Compatibility, Interaction Capability, Reliability, Security, Maintainability, Flexibility, and Safety, with 3–5 items per construct. Each of these attributes will be measured with 3-5 distinct survey items.  

The data collection process will begin with an orientation and the acquisition of informed consent for the respondents. Subsequently, participants will be given a brief explanation of the application before being allowed to explore it independently. Following this, the end-users will complete the TAM, while IT professionals will complete the ISO/IEC 25010:2023 and provide qualitative feedback. All responses will be transferred to a CSV file for analysis, with no personal data collected beyond optional contact information for follow-ups. Furthermore, all data will be stored on an encrypted drive, with access strictly limited to the researchers.


\subsection{Data Processing and Analysis}

The collected data will go through preprocessing and analysis to ensure accuracy in interpretation. Descriptive statistical methods will be used to summarize and interpret the collected data. The mean (Equation \ref{mean}) and standard deviation (Equation \ref{stddev}) will be computed to provide an overview of respondents’ perceptions and evaluations. These measures will help to determine how well the system meets the user expectations and software quality standards. The results will be presented in tables and figures, with verbal interpretation according to the defined scales for TAM and ISO/IEC 25010:2023.

The collected data will go through preprocessing and analysis to ensure accuracy in interpretation. The interpretation of data will be done with the help of descriptive statistics by computing the mean (Equation \ref{mean}) and standard deviation (Equation \ref{stddev}).
\begin{equation}
	\label{mean}
	\bar{x} = \frac{1}{n} \sum_{i=1}^{n} x_i
\end{equation}

\begin{equation}
	\label{stddev}
	\sigma = \sqrt{\frac{1}{n-1} \sum_{i=1}^{n} (x_i - \bar{x})^2}
\end{equation}
where $x_i$ is the score given by respondent $i$, and $n$ is the total number of valid responses.

With these, the respondents’ perceptions will be interpreted based on the scales defined for TAM and ISO/IEC 25010:2023, with corresponding verbal interpretation for each score. The verbal interpretation is shown in Table \ref{tam_interpret} and Table \ref{iso_interpret} for TAM and ISO 25010:2023, respectively. After that, the results will be presented in tables and figure to summarize and make the interpretation more comprehensible.  

% \newpage
\begin{table}[H]
	\centering
	\caption{Verbal interpretation Scale for TAM (5-point Likert Scale)}
	\label{tam_interpret} \renewcommand{\arraystretch}{0.6}
	\begin{tabularx}{\linewidth}{llX}
		\toprule
		\textbf{Range} & \textbf{Descriptive Rating} & \textbf{Verbal Interpretation}\\
		\midrule
		4.50 – 5.00 & Highly Acceptable & 	The user strongly perceives the system as useful, easy to use, and intends to use it.\\
		3.50 – 4.49	& Acceptable & 	The user generally perceives the system as useful and easy to use, with a positive intention to use it.\\
		2.50 – 3.49	& Fairly Acceptable & The user has mixed feelings about the system’s usefulness and ease of use, with uncertain intention to use it.\\
		1.50 – 2.49	& Slightly Acceptable & 	The user generally perceives the system as not useful or difficult to use, with a negative intention to use it.\\
		1.00 – 1.49	& Not Acceptable & 	The user strongly perceives the system as not useful, very difficult to use, and has no intention to use it.\\
		\bottomrule
	\end{tabularx}
\end{table}


\begin{table}[H]
	\centering
	\caption{Verbal Interpretation Scale for ISO/IEC 25010.2023 (10-point Likert Scale)}
	\label{iso_interpret} \renewcommand{\arraystretch}{0.6}
	\begin{tabularx}{\linewidth}{llX}
		\toprule
		\textbf{Range} & \textbf{Descriptive Rating} & \textbf{Verbal Interpretation}\\
		\midrule
		8.50 – 10.00 & 	Highly Acceptable & 	The system fully meets or exceeds requirements; very useful and effective.\\
		6.50 – 8.49	& Acceptable & 	The system meets requirements and is generally useful, with minor issues.\\
		4.50 – 6.49	& Moderately Acceptable	& The system is somewhat adequate but still needs refinements to be fully useful.\\
		2.50 – 4.49	& Slightly Acceptable & 	The system shows minimal adequacy but requires major improvements.\\
		1.00 – 2.49	& Not Acceptable & 	The system fails to meet requirements; not useful or practical.\\
		\bottomrule
	\end{tabularx}
\end{table}

\section{Ethical Considerations}
Ethical considerations will play a crucial role in the conduct of this study, especially since it will involve the participation of real people, and government institutions and offices. The ethical guidelines will strictly be followed in compliance with the Data Privacy Act of 2012 (Republic Act No. 10173) and Bulacan State University’s (BulSU) College of Science (CS) Ethical Committee, to uphold the integrity of the research and to ensure the privacy and security of all collected data. Obtaining informed consent, implementing data encryption, and enforcing strict access controls will be observed to safeguard the rights, and confidentiality of all individuals and institutions involved in the evaluation and development of the system. 

\subsection{Informed Consent}
Participants of the study will be required to give an informed consent before responding to any personally identifying data or opinions that will be collected for the study, usability testing, and system evaluation. Consent materials will inform respondents about the study purpose, what data will be collected, how the data will be used, risks and benefits, and contact details for further questions regarding the study. The project will follow recognized Philippine guidance on ethical review and informed consent processes such as the Data Privacy Act of 2012 (Republic Act No. 10173). Participants will also be given the freedom to withdraw their responses at will, at which point the collected data from them will be discarded and deleted.
In addition to individual respondents, the study will also involve data collection from government offices, particularly from Bulacan Provincial History, Arts, Culture, and Tourism Office (PHACTO) and other related agencies that maintain official records on local attractions, and tourism statistics. As per the request of the PHACTO, a letter required to be sent will formally address the Provincial Governor and the Head of the Provincial Tourism Department to request permission to access relevant datasets or documents. These data will include, but will not be limited to, lists of Points of Interest (POIs), and visitor statistics of tourism establishments.
The request letters will clearly outline the purpose of the data use, the scope of the requested information, and the intended outcomes of the study. Only publicly shareable or officially permitted data will be considered and will be used in compliance with all terms set by the concerned government offices. The use of such data will be limited strictly to academic and system development purposes.

\subsection{Confidentiality and Privacy}
All personal or sensitive information collected during the study will remain confidential. Any identifiable data from participants or government records such as names or addresses will be anonymized before analysis or publication by not digitizing said fields and replacing them with code or identifiers instead. Access to raw data will be restricted to authorized research team members only, and all information will be securely stored in digital repositories.

\subsection{Data Protection}
All collected data will be processed and stored in compliance with the Data Privacy Act of 2012 (Republic Act No. 10173) and its Implementing Rules and Regulations (IRR). The principles of transparency and legitimate purpose in handling personal and institutional data will be followed. Appropriate technical and organizational security measures, such as encrypted file storage and secure data transmission will be implemented to protect information from unauthorized access or loss. The collected physical data will be safely compiled in one clearbook folder which will be stored inside a vault for the duration of the study and will be disposed of one month after the final defense and digitally collected data will be encrypted in a spreadsheet file only accessible by the researchers, and will be deleted one month after the final defense.

\subsection{Ethical Data Collection}
The data collection process will be guided by integrity and accountability. Only information relevant to the objectives of the system will be gathered. The collection of data will not include sensitive personal information unless necessary and, if in any case is, will be collected under the permission of the respondent. In compliance with Philippine regulations, the consenting respondents that will be considered for the survey will strictly be 18 years old and above.
For survey and usability testing activities, participation will be entirely voluntary, and responses will be treated as confidential. For institutional data, such as those obtained from PHACTO or other local government offices, the researchers will follow the formal request procedures and uphold any data-sharing agreements or confidentiality clauses that will be imposed by the office.

\subsection{Transparency and Integrity}
The researchers will maintain transparency throughout the conduct of the study. All research methods, procedures, data sources, and analytical processes will be accurately documented and reported. The team will strictly avoid misrepresentation of data, selective reporting of results, or exaggeration of the system’s capabilities.
Acknowledgment will be given to all sources of information, including government offices and online databases. Conflicts of interest, if any, will be disclosed openly. All representations made to data providers, research participants, and readers will be honest and verifiable.
The study will uphold the principles of academic integrity, guided by the ethical standards outlined by the Data Privacy Act of 2012 (Republic Act No. 10173), Republic Act No. 8293 (Intellectual Property Code of the Philippines), and institutional research ethics guidelines of Bulacan State University (BulSU).
