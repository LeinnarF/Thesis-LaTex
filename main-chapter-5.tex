\chapter{Summary, Conclusions, and Recommendations}
\thispagestyle{empty}

\section{Summary of Findings}

This study developed a mobile-based application designed to provide optimized travel itineraries for tourism in the Province of Bulacan. The system utilizes the Adaptive Genetic Algorithm (AGAM) for generating travel itineraries and Simulated Annealing (SA) for optimizing routes. The findings of the study were derived from the development process, comparative analysis, and formal evaluations utilizing the Technology Acceptance Model and ISO/IEC 25010:2023 standards. From these procedures, the following findings were obtained.

\begin{enumerate}
	\item Algorithm Selection and Performance (TSP)\\
	Simulated Annealing (SA) was found to be the most suitable algorithm to be incorporated within the LAKAD application. Although the Genetic Algorithm (GA) performed the best in terms of solution quality, the execution efficiency of GA was found to be impractical for mobile application. In contrast, SA maintained a small average relative error and solved large problem sizes faster. When used through the Simple Additive Weighting (SAW) approach by giving equal weight to execution time and solution quality, SA was found to be the most preferred algorithm, followed by Elephant Herding Optimization (EHO). The findings show that SA provides the optimal balance of speed and accuracy, making it the best practical choice for real-time use in a mobile environment.
	
	\item Core System Features\\
	The system enables Tourist Spot Exploration that allows users to explore a list of tourist spots in Bulacan that have been validated by the Provincial History, Arts, Culture, and Tourism Office (PHACTO), complete with pictures, descriptions, and ratings. This data serves as the foundation for Personalized Itinerary Generation, where the system creates itineraries according to user-input parameters such as preferred districts or municipalities, landmark types, maximum travel distance, and the number of stopovers. To enhance efficiency, the Route Optimization feature rearranges the order of these stopovers to achieve the shortest possible total travel distance.
	
	For active travel, the Navigation component uses Mapbox and OSRM to deliver real-time GPS navigation, which includes driving, walking, and cycling as travel modes, while providing users with directions, estimated arrival and departure times per stopover, and the estimated total travel time of the itinerary. Complementing this is the Itinerary Management feature, which allows users to organize, edit, save, and track their progress through visited and unvisited locations. Finally, the Administrator Management side enables the scalability of the system through the management of user accounts, the creation or editing of tourist spot information, and the management of user-submitted reviews.
	
	\item Acceptability and User Experience\\
	The evaluation in Technology Acceptance Model (TAM) showed a highly favorable acceptance among the tourist participants, who found the system both useful and easy to operate. The findings supported that the users found the application to be an effective tool for searching tourist destinations in Bulacan, with a focus on their willingness to recommend the system to others. Moreover, the evaluative instrument proved its reliability through statistical validation, which demonstrated that the collected feedback accurately reflected the respondents' positive attitudes and behavioral intentions.
	
	\item Software Quality and Professional Validation\\
	The Technology field professionals evaluated the system in all software quality assessment criteria and found it to operate outstanding performance levels. The system achieved high accountability together with the functional performance that completely matched the professional requirements for its intended tasks. While experts noted minor variations in specialized technical areas like interoperability, they established a cohesive  agreement that the system is technically sound,secure, and ready for deployment.
	
\end{enumerate}

\section{Conclusion}
Researchers developed a mobile-based system which successfully solves both trip planning problems and the inconsistent promotion of tourist spots in Province of Bulacan. The researchers demonstrated that data-driven tools improve tourism accessibility for local areas when they use mathematical optimization models together with modern mobile technologies.

The Traveling Salesman Problem (TSP) algorithms comparison showed that Simulated Annealing (SA) works best for mobile implementation. The mobile environment could use exact methods and complex genetic algorithms to achieve higher precision but their computational demands make them impractical for real-time use. Users need fast solutions, and SA provides high solution quality and rapid execution. The Open Source Routing Machine (OSRM) became a crucial element for system scalability because it allowed the system to perform matrix calculations for personalized generation without any limits from commercial API restrictions.

The Adaptive Genetic Algorithm (AGAM) implementation shows that personalized recommendations become possible through balancing multiple objectives, such as interest, ratings, and distance. This approach moves beyond simple destination lists by creating meaningful and tailored itineraries that reflect the specific interest of the user while taking into consideration their physical and temporal constraints.

Finally, the highly favorable evaluations from both tourists and IT professionals validate that LAKAD is not only a technically sound system but also a socially relevant tool. The application serves as a promotional tool for the Provincial History Arts Culture and Tourism Office (PHACTO) who want to promote both popular tourist spots and the province's lesser-known attractions. The study demonstrates that LAKAD provides a strong user-centered solution for Bulacan sustainable tourism development through metaheuristic optimization and localized data combination.

\section{Recommendations}
To further enhance the capabilities and reach of Lakad, the following recommendations are proposed:

\begin{enumerate}
    \item The system needs to add more POIs compared to its current historical and cultural site base because its upcoming versions need to include different types of POIs. The application must include local hotels and dining options which will enable users to plan their travel needs through a single platform.
    
    \item The study used Simulated Annealing (SA) as its solution method, but researchers could investigate hybrid metaheuristic solutions and new algorithm designs to obtain better results for complex Traveling Salesman Problems (TSP).
    
    \item The system should extend its operations to additional provinces which will turn LAKAD into a comprehensive national tourism tool that serves all parts of the Philippines. The application needs an iOS version to reach more users because its current Android-only system prevents users who use iOS devices from accessing the app.
    
    \item The system must develop multi-day tour planning capabilities because its current system only supports single-day travel planning. The system needs to create overnight stay logic which enables users to plan their extended vacation stays in the province.
    
    \item The future version can use real-time traffic data and public transit direction systems to enhance its travel time estimation capabilities. The system will deliver better schedule information to users which reflects actual traffic conditions and different transportation options.
    
    \item The future versions of the system can incorporate an immersive exploration capability which uses Google Maps Street View or Virtual Reality technology. It allows users to virtually explore a destination and experience 360-degree views of Bulacan’s tourist spots which helps them travel with confidence while discovering new places.
    
    \item The system uses Supabase as its primary database solution, but it lacks the necessary capabilities to support large-scale database and user management tasks. The system aims to reduce its dependency on external APIs especially focusing on the OSRM Matrix API which has limited accessibility. OSRM is an open-source routing engine that can be deployed on a self-hosted server to provide distance matrix calculations with minimal limitations. The research proposes using a self-hosted server to achieve better scalability and system control while delivering dependable route calculations that operate independently of external services.
\end{enumerate}